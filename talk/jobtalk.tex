\documentclass[usenames,dvipsnames]{beamer}
\useoutertheme{infolines}

\usepackage{multirow}
\usepackage{tikz}
\usetikzlibrary{arrows,shapes,backgrounds}
\tikzstyle{every picture}+=[remember picture]
\tikzstyle{na} = [baseline=-.5ex]
\tikzstyle{background grid}=[draw, black!50,step=.5cm]
\usepackage{feynmp-auto}

\title[Charmed Searches]{Searches for Charmed Final States in ATLAS}
\author[dhg3]{Dan Guest}
\institute[Yale]{Yale University}


\usecolortheme{beaver}
\setbeamercolor{item projected}{bg=darkred}
\setbeamertemplate{enumerate items}[default]
\setbeamertemplate{navigation symbols}{}
\setbeamercovered{transparent}
\setbeamercolor{block title}{fg=darkred}
\setbeamercolor{local structure}{fg=darkred}
\usefonttheme{serif} % default family is serif

\definecolor{darkgreen}{rgb}{0,0.5,0}
\definecolor{aqua}{rgb}{0,1,1}

\newcommand{\feyninc}[2]{\scalebox{#1}{\input{../misc/feyngen/#2}}}

\newcommand{\lagr}{\mathscr{L}}
\newcommand{\Lc}{\mathrm{L}}
\newcommand{\Rc}{\mathrm{R}}
\newcommand{\cmenergy}{\sqrt{s} = 8\,\text{TeV}}
\newcommand{\lumi}{\mathcal{L}}
\newcommand{\invfb}{\text{fb}^{-1}}
\newcommand{\lumiunct}{20.34 \pm 0.57\, \invfb}
\newcommand{\atlas}{ATLAS}
\newcommand{\supp}[1]{\tilde{#1}}
\newcommand{\neut}{\supp{\chi}_1^0}
\newcommand{\cha}{\supp{\chi}_{1}^{\pm}}
\newcommand{\scharm}{\supp{c}}
\newcommand{\sctoc}{\supp{c} \to c \neut}
\newcommand{\sttoc}{\supp{t} \to c \neut}
\newcommand{\su}[1]{\mathrm{SU}(#1)}
\newcommand{\alphas}{\alpha_{\mathrm{s}}}
\newcommand{\ckmrow}[1]{V_{#1 d} & V_{#1 s} & V_{#1 b}}
\newcommand{\ckmrowabs}[1]{|V_{#1 d}| & |V_{#1 s}| & |V_{#1 b}|}
\newcommand{\cutoff}{\Lambda_{\mathrm{UV}}}
\newcommand{\smcosmo}{\Lambda\text{CDM}}
\newcommand{\T}{\mathrm{T}}
\newcommand{\met}{E_{\rm T}^{\rm miss}}
\newcommand{\vmet}{\mathbf{p}_{\rm T}^{\rm miss}}
\newcommand{\subvmet}[1]{\mathbf{p}_{\mathrm{T}\,\text{#1}}^{\text{miss}}}
\newcommand{\pt}{p_{\rm T}}
\newcommand{\vpt}{\mathbf{p}_\T}
\newcommand{\mct}{m_{\rm CT}}
\newcommand{\mcc}{m_{cc}}
\newcommand{\mll}{m_{\ell \ell}}
\newcommand{\mt}{m_{\rm T}}
\newcommand{\metdphi}[1]{\Delta \phi(\vmet, #1)}
\newcommand{\etmisslep}{|\vmet + \vpt^{\text{2 lepton}}|}
\newcommand{\meteff}{\met/(\pt^{\text{2 jet}} + \met)}
\newcommand{\meteffshort}{E^{\text{miss}}_{\T, \text{eff}}}
\newcommand{\wjets}{W + \text{jets}}
\newcommand{\zjets}{Z + \text{jets}}
\newcommand{\topbg}{top}
\newcommand{\ttbar}{t \bar{t}}
\newcommand{\cls}{\mathrm{CL}_{\mathrm{s}}}
\newcommand{\clb}{\mathrm{CL}_{\mathrm{b}}}
\newcommand{\clsb}{\mathrm{CL}_{\mathrm{s+b}}}
\newcommand{\mev}{\text{MeV}}
\newcommand{\gev}{\text{GeV}}
\newcommand{\tev}{\text{TeV}}

\newcommand{\crw}{CRW}
\newcommand{\crz}{CRZ}
\newcommand{\crt}{CRT}
\newcommand{\vrmcc}{VR $m_{cc}$}
\newcommand{\vrmct}{VR $m_{\rm CT}$}
\newcommand{\mctgt}[1]{\mct > #1\ \text{GeV}}
\newcommand{\meff}{m_{\mathrm{eff}}}

\newcommand{\nvxp}{n^{\mathrm{vx}}_{\text{5trk}}}

%% JetFitterCharm stuff
\newcommand{\catpt}{15, 25, 35, 50, 80, 120, 200, $\infty$}
\newcommand{\cateta}{0, 0.7, 1.5, 2.5}

%% styles
\newcommand{\plt}[1]{\textsf{#1}}
\newcommand{\vect}[1]{\bm{#1}}

%% jet cleaning quantities
\newcommand{\emf}{f_{\text{EM}}}
\newcommand{\chf}{f_{\text{CH}}}
\newcommand{\fmax}{f_\text{max}^{\text{layer}}}
\newcommand{\nege}{E_{\text{neg}}}
\newcommand{\hecf}{f_{\text{HEC}}}
\newcommand{\hecq}{f^{\text{HEC}}_Q}
\newcommand{\larq}{f^{\text{LAr}}_Q}
\newcommand{\qmean}{\langle Q_{\text{cell}}^{\mathrm{LAr}}\rangle}
%% \newcommand{\chargedfrac}{f_{\text{charged}}}
\newcommand{\jvf}{f_{\text{vx}}^{\text{jet}}}
\newcommand{\logicor}{\quad OR \quad}
\newcommand{\veryloose}{\textsc{VeryLoose}}

%% ckm matrix
\newcommand{\ckmmatrix}{\begin{pmatrix}
  0.97428 \pm 0.00015 & 0.2253 \pm 0.0007 & 0.00347^{+0.00016}_{-0.00012} \\
  0.2252 \pm 0.0007 & 0.97345^{+0.00015}_{-0.00016} & 0.0410^{+0.0011}_{-0.0007} \\
  0.00862_{-0.00020}^{+0.00026} & 0.0403_{-0.0007}^{+0.0011} & 0.999152^{+0.000030}_{-0.000045}
\end{pmatrix}}
\newcommand{\ckmmatrixapprox}{\begin{pmatrix}
  0.974 & 0.225 & 0.003 \\
  0.225 & 0.973 & 0.041 \\
  0.009 & 0.040 & 0.999
\end{pmatrix}}
\newcommand{\ckmparticle}{\begin{pmatrix}
  1 & \epsilon & 0 \\
  \epsilon & 1 & \epsilon \\
  0 & \epsilon & 1 \\
  \end{pmatrix}
}


\newcommand{\backupbegin}{
   \newcounter{framenumberappendix}
   \setcounter{framenumberappendix}{\value{framenumber}}
}
\newcommand{\backupend}{
   \addtocounter{framenumberappendix}{-\value{framenumber}}
   \addtocounter{framenumber}{\value{framenumberappendix}} 
}

\newcommand{\link}[2]{\underline{\href{#2}{#1}}}
\begin{document}

\maketitle

\begin{frame}
  \frametitle{Motivation: Why SUSY?}
  \begin{columns}
    \begin{column}{0.6\textwidth}
      \begin{itemize}
      \item Gives a DM candidate ($\neut$)
      \item Useful for GUT unification
      \item ``solves'' hierarchy  problem \\
        \ldots but only if superpartners are light
      \begin{tiny}
        \begin{equation*}
          \Delta m_{h}^2 = m_{\text{soft}}^2 \left[ \frac{\Gamma}{16 \pi^2} \ln(\cutoff / m_{\text{soft}}) + \ldots \right]
        \end{equation*}
      \end{tiny}
      \item simple SUSY models: visible at LHC
      \end{itemize}
    \end{column}
    \begin{column}{0.4\textwidth}
      \includegraphics[width=\textwidth]{%
figures/external/bulletcluster_comp_f2048.jpg} \\
    \end{column}
  \end{columns}
  \begin{center}
      \includegraphics[width=0.6\textwidth]{%
figures/external/unification-ugly.jpg}
  \end{center}
\end{frame}

\begin{frame}
  \frametitle{Motivation: $c$-tagged SUSY}
  \begin{columns}
    \begin{column}{0.6\textwidth}
      \begin{itemize}
      \item ``compressed'': $m_{\tilde{t}} - m_{\neut} < m_W$
      \item Forces $\tilde{t}$ decay via $c$-jets
      \item Signature: boosted, $c$-jets + $\met$
      \item See \link{arXiv 1501.01325}{http://arxiv.org/abs/1501.01325}
      %% \item \textcolor{red}{Not discussed in detail here (ADD LINK)}
      \end{itemize}
    \end{column}
    \begin{column}{0.4\textwidth}
      \includegraphics[width=\textwidth]{%
figures/external/feyn-sttoc-isr.pdf}
    \end{column}
  \end{columns}
  \hrule
  \begin{columns}
    \begin{column}{0.4\textwidth}
      \includegraphics[width=\textwidth]{%
figures/external/feyn-sctoc.pdf}
    \end{column}
    \begin{column}{0.6\textwidth}
      \begin{itemize}
      \item Same general signature as $\sttoc$, $m_{\tilde{t}} - m_{\neut} \nless m_W$
      \item Most searches: degenerate ``light'' ($udcs$) $\tilde{q}$ masses, not needed though (see \link{arXiv 1212.3328}{http://arxiv.org/abs/1212.3328})
      \item \textcolor{red}{More on this in this talk}
      \end{itemize}
    \end{column}
  \end{columns}
\end{frame}

\begin{frame}
  \frametitle{Charm Tagging}
  \begin{block}{Soft Lepton}
  \begin{itemize}
  \item Soft lepton: branching ratio to muons $<$ 7\% (18\%) for $D^0$ ($D^\pm$)
    \begin{itemize}
    \item Unacceptably low efficiency for most analyses
    \end{itemize}
  \end{itemize}
  \end{block}
  \begin{columns}
    \begin{column}{0.5\textwidth}
      \begin{block}{Lifetime Based}
      \begin{itemize}
      \item Shorter decay lengths vs $b$-jets
        \begin{center}
          \begin{tabular}{c | c c c}
            Meson  &  $B$ & $D^0$ & $D^{\pm}$ \\ \hline
            $c\tau$ [$\mu$m] & 492 & 312 & 123 \\
          \end{tabular}
        \end{center}
      \item Fewer tracks than $b$-jets
      \item $c$-jets are ``between'' light and $b$ in many distributions
      \item \textbf{NOTE:} Vertices inside inner pixels, tracks extrapolated
      \end{itemize}
      \end{block}
    \end{column}
    \begin{column}{0.5\textwidth}
      \includegraphics[width=\textwidth]{figures/external/b-jet.pdf}
    \end{column}
  \end{columns}
\end{frame}

\begin{frame}
  \frametitle{Tagging Algorithms}
  \begin{columns}[t]
    \begin{column}{0.5\textwidth}
      \begin{center}
        Impact Parameter (IP)
      \includegraphics[width=\textwidth]{figures/external/b-jet-ip.pdf}
      \end{center}
      \begin{itemize}
      \item Extrapolate tracks to perigee
      \item Combine IPs
      \end{itemize}
    \end{column}
    \vline{}
    \begin{column}{0.5\textwidth}
      \begin{center}
        Secondary Vertex (SV)
      \includegraphics[width=0.8\textwidth]{figures/external/b-jet-sv.pdf}
        \end{center}
      \begin{itemize}
      \item Fit secondary vertex to tracks
      \item Use SV properties (mass, displacement from PV, etc)
      \end{itemize}
    \end{column}
  \end{columns}
\end{frame}

\begin{frame}
  \frametitle{Tagging Algorithms (continued)}
  \begin{columns}
    \begin{column}{0.5\textwidth}
      JetFitter: the complicated tagger
      \begin{enumerate}
      \item Fits a ``Flight Line'' from the PV along jet axis
      \item Line updated to cross tracks
      \item Vertices clustered
      \item Vertex properties extracted
      \end{enumerate}
      \begin{itemize}
      \item Can resolve any number of SVs
      \item Retuned for $c$-tagging:
        \begin{itemize}
        \item Looser track selection
        \item Fewer tracks assigned to PV
        \end{itemize}
      \end{itemize}
    \end{column}
    \begin{column}{0.5\textwidth}
        \begin{center}
          JetFitter
      \includegraphics[width=\textwidth]{figures/external/b-jet-jf.pdf}
        \end{center}
    \end{column}
  \end{columns}
\end{frame}

%% \begin{frame}
%%   \frametitle{Charm Tagging}
%%   \begin{itemize}
%%   \item Two main approaches: ``soft lepton'' and ``lifetime''
%%   \item Soft lepton: branching ratio to muons $<$ 7\% (18\%) for $D^0$ ($D^\pm$)
%%     \begin{itemize}
%%     \item Unacceptably low efficiency for most analyses
%%     \end{itemize}
%%   \item Lifetime: limited mainly by detector resolution, but physically more difficult than $b$-tagging
%%     \begin{itemize}
%%     \item Shorter decay lengths for $D$ mesons
%%     \item Fewer tracks than $b$-jets, hard to resolve SV
%%     \item $c$-jets are ``between'' light and $b$ in many distributions
%%     \item \textcolor{red}{should replace this with Will's figure}
%%     \item \textcolor{red}{focus on how difficult it is to distinguish}
%%     \end{itemize}
%%     \begin{center}
%%       \begin{tabular}{c | c c c}
%%         Meson  &  $B$ & $D^0$ & $D^{\pm}$ \\ \hline
%%         $c\tau$ [$\mu$m] & 492 & 312 & 123 \\
%%       \end{tabular}
%%     \end{center}
%%   \item May get better with the IBL
%%   \item \link{JetFitterCharm}{%
%%     https://atlas.web.cern.ch/Atlas/GROUPS/PHYSICS/PUBNOTES/ATL-PHYS-PUB-2015-001/} used in two public results:
%%     \link{$\tilde{c} \to c\chi_1^0$}{http://arxiv.org/abs/1501.01325} and
%%     \link{$\tilde{t} \to c\chi_1^0$}{http://arxiv.org/abs/1407.0608}.
%%   \end{itemize}
%% \end{frame}

\begin{frame}
  \frametitle{JetFitterCharm Overview}
  \begin{columns}
    \begin{column}{0.6\textwidth}
      \begin{itemize}
      \item Combine all three approaches
      \item IP based: IP3D
        \begin{itemize}
        \item {2D IP} for each track
        \item Lookup \fcolorbox{black}{aqua}{$\mathcal{L}_b$}, \fcolorbox{black}{aqua}{$\mathcal{L}_{\text{light}}$} for each track (based on simulation)
        \item Calculate $\sum_{\text{trk}} \ln (\mathcal{L}_b / \mathcal{L}_{\text{light}})$
        \end{itemize}
      \item Add {vertex variables}, e.g.
        \begin{itemize}
        \item Vertex significance
        \item Vertex mass
        \item Fraction of jet energy in vertices
        \end{itemize}
      \item Feed all variables into {a NN}
      \item Cut on neural net outputs: \\
        $P_{b}$, $P_{c}$, and $P_{u}$ ($P_{\rm light}$)
      \end{itemize}
    \end{column}
    \begin{column}{0.4\textwidth}
      \includegraphics[width=\textwidth]{figures/jfc/dot/simple-arch.pdf}
    \end{column}
  \end{columns}
  \vspace{1em}
    \begin{tiny}
      PUB Note: \texttt{\url{https://atlas.web.cern.ch/Atlas/GROUPS/PHYSICS/PUBNOTES/ATL-PHYS-PUB-2015-001/}}
    \end{tiny}
\end{frame}


%% NOTE: should make sure I mention that there's no clean cut to isolate
%% c-jets or b-jets
\begin{frame}
  \frametitle{Tagging Discriminants}
  \begin{itemize}
  \item Define two discriminants based on NN outputs:
  \begin{align*}
    \text{anti-}b & \equiv \frac{P_c}{P_b} &
    \text{anti-light} & \equiv \frac{P_c}{P_{\text{light}}}
  \end{align*}
  \item Can trade $b$-rejection for light-rejection
  \end{itemize}
  \begin{columns}
    \begin{column}{0.5\textwidth}
      \includegraphics[width=\textwidth]{figures/jfc/rejrej-simple.pdf}
    \end{column}
    \begin{column}{0.5\textwidth}
      \includegraphics[width=\textwidth]{figures/jfc/2d-cut.pdf}
    \end{column}
  \end{columns}
\end{frame}

\begin{frame}
  \frametitle{Charm tagging calibrations}
  \begin{columns}
    \begin{column}{0.6\textwidth}
      \begin{itemize}
      \item Standard $b$-tagging calibrations:
        \begin{itemize}
        \item[light] Negative tags \tikz[na] \coordinate(neg-tag);
        \item[$b$] dileptonic $t \bar{t}$ \tikz[na] \coordinate(dilep-tt);
        \item[$c$] $D^*$ reconstruction \tikz[na] \coordinate(dzero);
        \end{itemize}
      \end{itemize}
      \includegraphics[width=\textwidth]{%
figures/external/dstar-pion-mass.pdf}\\
    \end{column}
    \begin{column}{0.34\textwidth}
      \includegraphics[width=\textwidth]{%
figures/external/negative-tag-sv0.pdf}\\
      \includegraphics[width=\textwidth]{%
figures/external/giacinto-mll.pdf}
    \end{column}
  \end{columns}
\end{frame}

\begin{frame}
  \frametitle{Charm-Tagging Scale Factors}
  \begin{columns}
    \begin{column}{0.6\textwidth}
  \begin{itemize}
  \item Standard $b$-tagging calibrations:
    \begin{itemize}
    \item[light] Negative tags \tikz[na] \coordinate(neg-tag);
    \item[$b$] dileptonic $t \bar{t}$ \tikz[na] \coordinate(dilep-tt);
    \item[$c$] $D^*$ reconstruction \tikz[na] \coordinate(dzero);
    \end{itemize}
  \end{itemize}
      \includegraphics[width=\textwidth]{%
figures/external/sf-ctag-c-medium.pdf}
    \end{column}
    \begin{column}{0.4\textwidth}
      \includegraphics[width=\textwidth]{%
figures/external/sf-ctag-u-medium.pdf} \\
      \includegraphics[width=\textwidth]{%
figures/external/sf-ctag-b-medium.pdf} \\
    \end{column}
  \end{columns}
\end{frame}

%% \begin{frame}
%%   \frametitle{Interlude: Deep Learning}
%%   \begin{columns}
%%     \begin{column}{0.5\textwidth}
%%     \end{column}
%%     \begin{column}{0.4\textwidth}
%%       \includegraphics[width=\textwidth]{figures/external/autoencoder-layer.png}
%%     \end{column}
%%   \end{columns}
%%   \begin{tiny}
%%     figure credit: \url{http://ufldl.stanford.edu/wiki/index.php/Stacked_Autoencoders}
%%   \end{tiny}
%% \end{frame}

\begin{frame}
  \frametitle{Interlude: Deep Learning}
  \begin{columns}
    \begin{column}{0.6\textwidth}
      \begin{center}
        Flavor-Tagging Framework (Run 1)
        \includegraphics[width=\textwidth]{%
figures/tagging-graphs/tagging-full-nogaia.pdf} \\
        \includegraphics[width=0.8\textwidth]{%
figures/tagging-graphs/tagging-leg.pdf}
      \end{center}
    \end{column}
    \vline
    \begin{column}{0.4\textwidth}
      \begin{center}
        Goal for Run 2 \\[0.3cm]
        \includegraphics[width=\textwidth]{%
figures/tagging-graphs/tagging-gaia-future.pdf}
      \end{center}
    \end{column}
  \end{columns}
  \begin{itemize}
  \item Framework complicated (left) because training NNs is hard
  \item We can fix this (right) by training NNs with autoencoders
  \end{itemize}
\end{frame}

%% \begin{frame}
%%   \frametitle{Deep Learning for Flavor Tagging}
%%   \begin{itemize}
%%   \item find some autoencoder picture(s) on the internets.
%%   \item maybe mention how awesome training on data could be
%%   \item talk about denoising?
%%   \end{itemize}
%% \end{frame}

\begin{frame}
  \frametitle{Deep Learning: Performance}
  \begin{columns}
    \begin{column}{0.5\textwidth}
      \begin{center}
        $b$-tagging
       \includegraphics[width=\textwidth]{%
figures/external/uRejRoc.pdf}
      \end{center}
    \end{column}
    \begin{column}{0.5\textwidth}
      \begin{center}
        $c$-tagging
      \includegraphics[width=\textwidth]{%
figures/external/ctag-2d-gaia-vs-jfc.pdf}
      \end{center}
    \end{column}
  \end{columns}
  \begin{itemize}
  \item Autoencoders simplify the framework \emph{and} perform better
    \begin{itemize}
    \item For $b$-tagging \textcolor{red}{autoencoders} are always above \textcolor{darkgreen}{M1}
    \item More efficient $c$-tagger for any choice of $b$ or light rejection
    \end{itemize}
  \item Being ported to Athena as a qualification task
  \end{itemize}
\end{frame}

\begin{frame}[fragile=singleslide]
  \frametitle{Back to Searches: $\sctoc$}
  \begin{columns}
    \begin{column}{0.5\textwidth}
      \begin{itemize}
      %% \item \link{arXiv 1212.3328}{http://arxiv.org/abs/1212.3328}
      \item Signature: \textcolor{darkgreen}{$c$-jets} + \textcolor{red}{$\met$}
      \item Trigger on $\met$ (\verb|EF_xe80_tclcw_tight|)
      \item $c$-tag two leading jets
      \item Veto leptons
      \end{itemize}
    \end{column}
    \begin{column}{0.5\textwidth}
      \feyninc{1.0}{scsc-ccN1N1}      %% \includegraphics[width=\textwidth]{figures/external/feyn-sctoc.pdf}
    \end{column}
  \end{columns}
  \begin{itemize}
  \item Problems for $\Delta m \equiv m_{\scharm} - m_{\neut} \lesssim 80\,\gev$:
    \begin{itemize}
    \item $\met$ trigger won't fire
    \item $c$-jets harder to resolve
    \end{itemize}
  \item Solution: add an \textcolor{orange}{ISR boosted} signal region
    \begin{itemize}
    \item This SR already exists in the $\sttoc$ search
    \item Even better: simplified SUSY models mean $\sigma_{\tilde{t}\tilde{t}} \approx \sigma_{\tilde{c}\tilde{c}}$
    \item We can take $\cls$ values directly from $\sttoc$
    \end{itemize}
  \item Boosted region isn't discussed further here
  \end{itemize}
\end{frame}

\newcommand{\feynincstd}[1]{\feyninc{1.0}{#1}}
\begin{frame}
  \frametitle{Backgrounds: Multijet QCD}
  \begin{columns}
    \begin{column}{0.5\textwidth}
    \feyninc{0.9}{multijet} \\[0.2cm]
      \begin{itemize}
        \item \textcolor{red}{one mismeasured jet} can result in large $\met$
      \end{itemize}
    \end{column}
    \begin{column}{0.5\textwidth}
    \includegraphics[width=\textwidth]{%
int/figures/stackplots/dans/preselection/jetmet_dphi.pdf}
    \end{column}
  \end{columns}
  \begin{itemize}
  \item Require $\Delta \phi(j_{\{1,2,3\}},\met) > 0.4$
  \item Remaining QCD mostly light flavor $\to$ removed by $c$-tagging
  \item Estimated with ``jet smearing'' (see \url{https://cdsweb.cern.ch/record/1423310}), small after all cuts
  \end{itemize}
\end{frame}

\begin{frame}
  \frametitle{Backgrounds: $W$/$Z$ + jets}
  \begin{columns}
    \begin{column}{0.5\textwidth}
    \feynincstd{vjets} \\[0.2cm]
      \begin{itemize}
      \item After anti-QCD cuts, backgrounds are $W$/$Z$ + jets
      \item Real $\met$ from $Z \to \nu \nu$ and $W \to \tau \nu$
      \end{itemize}
    \end{column}
    \begin{column}{0.5\textwidth}
      \begin{center}
        After QCD cuts
      \includegraphics[width=\textwidth]{%
int/figures/stackplots/dans/preselection/j0_flavor_truth_label.pdf}
      \end{center}
    \end{column}
  \end{columns}
  \begin{itemize}
    \item Jets are dominantly light flavored, removed by two $c$-tag requirement (each tag: $1/\epsilon_c \approx 150$)
  \end{itemize}
\end{frame}

\begin{frame}
  \frametitle{Backgrounds: $t\bar{t}$}
  \begin{columns}
    \begin{column}{0.6\textwidth}
      \begin{center}
        \feynincstd{ttbar} \\[0.2cm]
      \end{center}
    \begin{itemize}
    \item After $c$-tagging cuts, large $t\bar{t}$ background remains
    \item Remove with two kinematic cuts:
      \begin{itemize}
      \item[$\mcc$]: also removes QCD
      \item[$\mct$]: $t\bar{t}$ below $135\,\gev$
      \end{itemize}
    \end{itemize}
    \begin{tiny}
 $\mct \equiv \{[ E_\T(v_{1}) + E_\T(v_{2}) ]^2 -  [  \vpt(v_{1}) - \vpt(v_{2}) ]^2\}^{1/2}$
    \end{tiny}
    \end{column}
    \begin{column}{0.4\textwidth}
      \includegraphics[width=\textwidth]{%
int/figures/stackplots/dans/signal_mct150/mass_cc_blinded.pdf}\\
      \includegraphics[width=\textwidth]{%
int/figures/stackplots/dans/signal_mct150/mass_ct_blinded.pdf}
    \end{column}
  \end{columns}
\end{frame}

%% \begin{frame}
%%   \frametitle{Discriminants}
%%   \begin{itemize}
%%   \item $\mct$
%%   \item $c$-tagging distributions?
%%   \end{itemize}
%% \end{frame}

\begin{frame}
  \frametitle{Control Regions}
  \begin{columns}
    \begin{column}{0.6\textwidth}
      \begin{itemize}
      \item $Z$ + jets, $W$ + jets, and $t\bar{t}$ can't be completely removed
      \item Instead, constrained with 1 and 2 lepton control regions
        \begin{itemize}
        \item 
        \end{itemize}
      \end{itemize}
    \end{column}
    \begin{column}{0.4\textwidth}
      \includegraphics[width=\textwidth]{%
int/figures/stackplots/dans/cr_z/mass_ll.pdf} \\
      \includegraphics[width=\textwidth]{%
int/figures/stackplots/dans/cr_w/mass_t.pdf}
    \end{column}
  \end{columns}
\end{frame}

\begin{frame}
  \frametitle{Systematics}
  \begin{itemize}
  \item probably show Will's table?
  \end{itemize}
\end{frame}

\begin{frame}
  \frametitle{Results}
  \begin{itemize}
  \item show the unblinded SR in $\mct$
  \end{itemize}
\end{frame}

\begin{frame}
  \frametitle{Exclusion Fit}
  \begin{itemize}
  \item Show combined exclusion for $\scharm$ regions
  \end{itemize}
\end{frame}

\begin{frame}
  \frametitle{Near the Diagonal: $\sttoc$ search}
  \begin{itemize}
    \item show that exclusion
    \item same cross-section, same final state!
  \end{itemize}
\end{frame}

\begin{frame}
  \frametitle{Combined exclusion in the $\scharm$--$\neut$ plane}
  \begin{itemize}
  \item Show the dotted one, point out where the regions work well
  \end{itemize}
\end{frame}

\begin{frame}
  \frametitle{Comparison with inclusive $\tilde{q}$ search}
  \begin{itemize}
  \item Show inclusive comparison
  \item Point out the increase
  \item Mention the gained sensitivity to flavor
  \end{itemize}
\end{frame}

\begin{frame}
  \frametitle{Conclusions}
  \begin{itemize}
  \item $c$-tagged SUSY: traditionally too hard
  \item First-Generation $c$-tagging:
    \begin{itemize}
    \item Has been used in several searches (link to papers again)
    \end{itemize}
  \item The next generation:
    \begin{itemize}
    \item Use autoencoders
    \item Improved $c$ and $b$ tagging
    \end{itemize}
  \end{itemize}
\end{frame}

\begin{frame}
  \frametitle{Outlook}
  \begin{itemize}
  \item $H \to c \bar{c}$ (will be hard but awesome)
  \item more general taggers (include $\tau$, $g \to b \bar{b}$)
  \item Better $b$-tagging means $H \to HH \to bbXX$ (include this diagram)
  \item xAOD migration \emph{could} be promising
  \end{itemize}
\end{frame}

\backupbegin
\begin{frame}
  \begin{Huge}
    \begin{center}
      BONUS \\ SLIDES
    \end{center}
  \end{Huge}
\end{frame}

\begin{frame}
  \frametitle{Comparison Between COMBNN and JFC}
\end{frame}

\begin{frame}
  \frametitle{Comparison Between JFC and MVx}
\end{frame}

\begin{frame}
  \frametitle{Comparison Between GAIA and JFC}
\end{frame}

\begin{frame}
  \frametitle{List of Tagging Variables}
  \begin{itemize}
  \item should probably make some pictures of some of them
  \end{itemize}
\end{frame}

\backupend

\end{document}
