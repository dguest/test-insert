\documentclass[usenames,dvipsnames]{beamer}
\useoutertheme{infolines}

\usepackage{multirow}

\title[Charmed Searches]{Searches for Charmed Final States in ATLAS}
\author[dhg3]{Dan Guest}
\institute[Yale]{Yale University}


\usecolortheme{beaver}
\setbeamercolor{item projected}{bg=darkred}
\setbeamertemplate{enumerate items}[default]
\setbeamertemplate{navigation symbols}{}
\setbeamercovered{transparent}
\setbeamercolor{block title}{fg=darkred}
\setbeamercolor{local structure}{fg=darkred}
\usefonttheme{serif} % default family is serif

\newcommand{\lagr}{\mathscr{L}}
\newcommand{\Lc}{\mathrm{L}}
\newcommand{\Rc}{\mathrm{R}}
\newcommand{\cmenergy}{\sqrt{s} = 8\,\text{TeV}}
\newcommand{\lumi}{\mathcal{L}}
\newcommand{\invfb}{\text{fb}^{-1}}
\newcommand{\lumiunct}{20.34 \pm 0.57\, \invfb}
\newcommand{\atlas}{ATLAS}
\newcommand{\supp}[1]{\tilde{#1}}
\newcommand{\neut}{\supp{\chi}_1^0}
\newcommand{\cha}{\supp{\chi}_{1}^{\pm}}
\newcommand{\scharm}{\supp{c}}
\newcommand{\sctoc}{\supp{c} \to c \neut}
\newcommand{\sttoc}{\supp{t} \to c \neut}
\newcommand{\su}[1]{\mathrm{SU}(#1)}
\newcommand{\alphas}{\alpha_{\mathrm{s}}}
\newcommand{\ckmrow}[1]{V_{#1 d} & V_{#1 s} & V_{#1 b}}
\newcommand{\ckmrowabs}[1]{|V_{#1 d}| & |V_{#1 s}| & |V_{#1 b}|}
\newcommand{\cutoff}{\Lambda_{\mathrm{UV}}}
\newcommand{\smcosmo}{\Lambda\text{CDM}}
\newcommand{\T}{\mathrm{T}}
\newcommand{\met}{E_{\rm T}^{\rm miss}}
\newcommand{\vmet}{\mathbf{p}_{\rm T}^{\rm miss}}
\newcommand{\subvmet}[1]{\mathbf{p}_{\mathrm{T}\,\text{#1}}^{\text{miss}}}
\newcommand{\pt}{p_{\rm T}}
\newcommand{\vpt}{\mathbf{p}_\T}
\newcommand{\mct}{m_{\rm CT}}
\newcommand{\mcc}{m_{cc}}
\newcommand{\mll}{m_{\ell \ell}}
\newcommand{\mt}{m_{\rm T}}
\newcommand{\metdphi}[1]{\Delta \phi(\vmet, #1)}
\newcommand{\etmisslep}{|\vmet + \vpt^{\text{2 lepton}}|}
\newcommand{\meteff}{\met/(\pt^{\text{2 jet}} + \met)}
\newcommand{\meteffshort}{E^{\text{miss}}_{\T, \text{eff}}}
\newcommand{\wjets}{W + \text{jets}}
\newcommand{\zjets}{Z + \text{jets}}
\newcommand{\topbg}{top}
\newcommand{\ttbar}{t \bar{t}}
\newcommand{\cls}{\mathrm{CL}_{\mathrm{s}}}
\newcommand{\clb}{\mathrm{CL}_{\mathrm{b}}}
\newcommand{\clsb}{\mathrm{CL}_{\mathrm{s+b}}}
\newcommand{\mev}{\text{MeV}}
\newcommand{\gev}{\text{GeV}}
\newcommand{\tev}{\text{TeV}}

\newcommand{\crw}{CRW}
\newcommand{\crz}{CRZ}
\newcommand{\crt}{CRT}
\newcommand{\vrmcc}{VR $m_{cc}$}
\newcommand{\vrmct}{VR $m_{\rm CT}$}
\newcommand{\mctgt}[1]{\mct > #1\ \text{GeV}}
\newcommand{\meff}{m_{\mathrm{eff}}}

\newcommand{\nvxp}{n^{\mathrm{vx}}_{\text{5trk}}}

%% JetFitterCharm stuff
\newcommand{\catpt}{15, 25, 35, 50, 80, 120, 200, $\infty$}
\newcommand{\cateta}{0, 0.7, 1.5, 2.5}

%% styles
\newcommand{\plt}[1]{\textsf{#1}}
\newcommand{\vect}[1]{\bm{#1}}

%% jet cleaning quantities
\newcommand{\emf}{f_{\text{EM}}}
\newcommand{\chf}{f_{\text{CH}}}
\newcommand{\fmax}{f_\text{max}^{\text{layer}}}
\newcommand{\nege}{E_{\text{neg}}}
\newcommand{\hecf}{f_{\text{HEC}}}
\newcommand{\hecq}{f^{\text{HEC}}_Q}
\newcommand{\larq}{f^{\text{LAr}}_Q}
\newcommand{\qmean}{\langle Q_{\text{cell}}^{\mathrm{LAr}}\rangle}
%% \newcommand{\chargedfrac}{f_{\text{charged}}}
\newcommand{\jvf}{f_{\text{vx}}^{\text{jet}}}
\newcommand{\logicor}{\quad OR \quad}
\newcommand{\veryloose}{\textsc{VeryLoose}}

%% ckm matrix
\newcommand{\ckmmatrix}{\begin{pmatrix}
  0.97428 \pm 0.00015 & 0.2253 \pm 0.0007 & 0.00347^{+0.00016}_{-0.00012} \\
  0.2252 \pm 0.0007 & 0.97345^{+0.00015}_{-0.00016} & 0.0410^{+0.0011}_{-0.0007} \\
  0.00862_{-0.00020}^{+0.00026} & 0.0403_{-0.0007}^{+0.0011} & 0.999152^{+0.000030}_{-0.000045}
\end{pmatrix}}
\newcommand{\ckmmatrixapprox}{\begin{pmatrix}
  0.974 & 0.225 & 0.003 \\
  0.225 & 0.973 & 0.041 \\
  0.009 & 0.040 & 0.999
\end{pmatrix}}
\newcommand{\ckmparticle}{\begin{pmatrix}
  1 & \epsilon & 0 \\
  \epsilon & 1 & \epsilon \\
  0 & \epsilon & 1 \\
  \end{pmatrix}
}


\newcommand{\backupbegin}{
   \newcounter{framenumberappendix}
   \setcounter{framenumberappendix}{\value{framenumber}}
}
\newcommand{\backupend}{
   \addtocounter{framenumberappendix}{-\value{framenumber}}
   \addtocounter{framenumber}{\value{framenumberappendix}} 
}

\newcommand{\link}[2]{\underline{\href{#2}{#1}}}
\begin{document}

\maketitle

\begin{frame}
  \frametitle{Motivation: Why SUSY?}
  \begin{itemize}
  \item show some pictures of susy stuff
  \item naturalness equations
  \end{itemize}
\end{frame}

\begin{frame}
  \frametitle{Motivation: Why Charm?}
  \begin{itemize}
  \item show stop--lsp plane
  \item feynman diagrams for both
  %% \item Link to papers about how light scharm is possible
  \end{itemize}
\end{frame}

\begin{frame}
  \frametitle{Charm Tagging}
  \begin{itemize}
  \item Two main approaches: ``soft lepton'' and ``lifetime''
  \item Soft lepton: branching ratio to muons $<$ 7\% (18\%) for $D^0$ ($D^\pm$)
    \begin{itemize}
    \item Unacceptably low efficiency for most analyses
    \end{itemize}
  \item Lifetime: limited mainly by detector resolution, but physically more difficult than $b$-tagging
    \begin{itemize}
    \item Shorter decay lengths for $D$ mesons
    \item Fewer tracks than $b$-jets, hard to resolve SV
    \item $c$-jets are ``between'' light and $b$ in many distributions
    \item \textcolor{red}{should replace this with Will's figure}
    \item \textcolor{red}{focus on how difficult it is to distinguish}
    \end{itemize}
    \begin{center}
      \begin{tabular}{c | c c c}
        Meson  &  $B$ & $D^0$ & $D^{\pm}$ \\ \hline
        $c\tau$ [$\mu$m] & 492 & 312 & 123 \\
      \end{tabular}
    \end{center}
  \item May get better with the IBL
  \item \link{JetFitterCharm}{%
    https://atlas.web.cern.ch/Atlas/GROUPS/PHYSICS/PUBNOTES/ATL-PHYS-PUB-2015-001/} used in two public results:
    \link{$\tilde{c} \to c\chi_1^0$}{http://arxiv.org/abs/1501.01325} and
    \link{$\tilde{t} \to c\chi_1^0$}{http://arxiv.org/abs/1407.0608}.
  \end{itemize}
\end{frame}

\begin{frame}
  \frametitle{JetFitterCharm Overview}
  \begin{columns}
    \begin{column}{0.6\textwidth}
      \begin{itemize}
      \item Start with $b$-tagging tools
      \item Loosen JetFitter:
        \begin{itemize}
        \item Looser track selection
        \item Assign more tracks to SV
        \end{itemize}
      \item Add JF Variables:
        \begin{itemize}
        \item $L_{xy}^1$, $L_{xy}^2$ (secondary and tertiary vertex displacement)
        \item $\varphi_{\textrm{trk}}$ (track rapidity along jet axis)
        \end{itemize}
      \item Feed all variables to a neural net
        \begin{itemize}
        \item 19-20-10-3 topology
        \item Trained with JETNET on $t\bar{t}$
        \end{itemize}
      \item Cut on neural net outputs: \\
        $P_{b}$, $P_{c}$, and $P_{u}$ ($P_{\rm light}$)
      \end{itemize}
    \end{column}
    \begin{column}{0.4\textwidth}
      \includegraphics[width=\textwidth]{figures/jfc/dot/simple-arch.pdf}
    \end{column}
  \end{columns}
  \vspace{1em}
    \begin{tiny}
      PUB Note: \texttt{\url{https://atlas.web.cern.ch/Atlas/GROUPS/PHYSICS/PUBNOTES/ATL-PHYS-PUB-2015-001/}}
    \end{tiny}
\end{frame}


%% NOTE: should make sure I mention that there's no clean cut to isolate
%% c-jets or b-jets
\begin{frame}
  \frametitle{Output Probabilities}
  \begin{itemize}
  \item Define two discriminants based on NN outputs:
  \begin{align*}
    \text{anti-}b & \equiv \frac{P_c}{P_b} &
    \text{anti-light} & \equiv \frac{P_c}{P_{\text{light}}}
  \end{align*}
  \item In all plots that follow, cut on these two variables
  \end{itemize}
  \begin{columns}
    \begin{column}{0.5\textwidth}
      \includegraphics[width=\textwidth]{%
figures/jfc/anti-bottom-discriminant.pdf}
    \end{column}
    \begin{column}{0.5\textwidth}
      \includegraphics[width=\textwidth]{%
figures/jfc/anti-light-discriminant.pdf}
    \end{column}
  \end{columns}
\end{frame}

%% NOTE: should probably combine this and the previous slide
\begin{frame}
  \frametitle{Operating Points}
  \begin{columns}
    \begin{column}{0.5\textwidth}
      \begin{itemize}
      \item Prototype analysis: \link{$\tilde{t} \to c \chi_1^0$}{http://arxiv.org/abs/1407.0608}
        \begin{itemize}
        \item Backgrounds: $W/Z + \text{ Jets}$, $\sim 10\%$ top
        \end{itemize}
      \item Defined two operating points
        \begin{itemize}
        \item loose: reject $b$-jets
        \item medium: reject light-, $b$-jets
        \end{itemize}
      \end{itemize}
    \end{column}
    \begin{column}{0.5\textwidth}
      \includegraphics[width=\textwidth]{figures/jfc/2d-cut.pdf}
    \end{column}
  \end{columns}
  \begin{tabular}{c|c c | c c c }
    Operating Point & $\log P_c / P_b$ & $\log P_c / P_{\text{light}}$ & $\epsilon_c$ & $1/\epsilon_b$ & $1/\epsilon_{\text{light}}$ \\ \hline
    loose & $> -0.9$ & -- & 0.95 & 2.5 & 1.0 \\
    medium & $> -0.9$ & $> 0.95$ & 0.20 & 8.0 & 200 \\
  \end{tabular}
\end{frame}

\begin{frame}
  \frametitle{Performance: Other Potential Operating Points}
  \begin{columns}
    \begin{column}{0.52\textwidth}
      \includegraphics[width=\textwidth]{figures/jfc/jfc-ctag-roc.pdf}
    \end{column}
    \begin{column}{0.48\textwidth}
      \includegraphics[width=\textwidth]{figures/jfc/rejrej-simple.pdf}
    \end{column}
  \end{columns}
  \begin{itemize}
  \item Plots above: possible combinations of $1/\epsilon_{b}$ and $1/\epsilon_{\rm light}$ vs $\epsilon_c$
  \item In all cases, use ``rectangular'' anti-$b$ and anti-light cut combination
  \item Big trade-offs between light and $b$ rejection are possible
  \end{itemize}
\end{frame}

\begin{frame}
  \frametitle{Interlude: Calibration}
  \begin{itemize}
  \item Show some calibration plots
  \item Mention that this took \emph{forever}
  \item Say that we're going to keep talking about flavor tagging, but JFC is what we used
  \end{itemize}
\end{frame}

\begin{frame}
  \frametitle{JetFitterCharm in the Flavor-Tagging Framework}
  \begin{itemize}
  \item show the mess of taggers that existed
  \end{itemize}
\end{frame}

\begin{frame}
  \frametitle{Deep Learning for Flavor Tagging}
  \begin{itemize}
  \item find some autoencoder picture(s) on the internets.
  \item maybe mention how awesome training on data could be
  \item talk about denoising?
  \end{itemize}
\end{frame}

\begin{frame}
  \frametitle{Deep Learning Performance}
  \begin{itemize}
  \item Show so many plots here
  \item Point out how much more important efficiency gain is vs rejection
  \item Mention that we have people working on this!
  \end{itemize}
\end{frame}

\begin{frame}
  \frametitle{Back to Searches: Scharm to Charm}
  \begin{itemize}
  \item show feynman diagram again
  \item talk about general strategy: $\met$ + $c$-tagging
  \end{itemize}
\end{frame}

\begin{frame}
  \frametitle{Backgrounds}
  \begin{itemize}
  \item show backgrounds, color code met
  \end{itemize}
\end{frame}

\begin{frame}
  \frametitle{Discriminants}
  \begin{itemize}
  \item $\mct$
  \item $c$-tagging distributions?
  \end{itemize}
\end{frame}

\begin{frame}
  \frametitle{Control Regions}
  \begin{itemize}
  \item not sure if this should be one slide or several\ldots
  \end{itemize}
\end{frame}

\begin{frame}
  \frametitle{Systematics}
  \begin{itemize}
  \item probably show Will's table?
  \end{itemize}
\end{frame}

\begin{frame}
  \frametitle{Results}
  \begin{itemize}
  \item show the unblinded SR in $\mct$
  \end{itemize}
\end{frame}

\begin{frame}
  \frametitle{Exclusion Fit}
  \begin{itemize}
  \item Show combined exclusion for $\scharm$ regions
  \end{itemize}
\end{frame}

\begin{frame}
  \frametitle{Near the Diagonal: $\sttoc$ search}
  \begin{itemize}
    \item show that exclusion
    \item same cross-section, same final state!
  \end{itemize}
\end{frame}

\begin{frame}
  \frametitle{Combined exclusion in the $\scharm$--$\neut$ plane}
  \begin{itemize}
  \item Show the dotted one, point out where the regions work well
  \end{itemize}
\end{frame}

\begin{frame}
  \frametitle{Comparison with inclusive $\tilde{q}$ search}
  \begin{itemize}
  \item Show inclusive comparison
  \item Point out the increase
  \item Mention the gained sensitivity to flavor
  \end{itemize}
\end{frame}

\begin{frame}
  \frametitle{Conclusions}
  \begin{itemize}
  \item $c$-tagged SUSY: traditionally too hard
  \item First-Generation $c$-tagging:
    \begin{itemize}
    \item Has been used in several searches (link to papers again)
    \end{itemize}
  \item The next generation:
    \begin{itemize}
    \item Use autoencoders
    \item Improved $c$ and $b$ tagging
    \end{itemize}
  \end{itemize}
\end{frame}

\begin{frame}
  \frametitle{Outlook}
  \begin{itemize}
  \item $H \to c \bar{c}$ (will be hard but awesome)
  \item more general taggers (include $\tau$, $g \to b \bar{b}$)
  \item Better $b$-tagging means $H \to HH \to bbXX$ (include this diagram)
  \end{itemize}
\end{frame}

\backupbegin
\begin{frame}
  \begin{Huge}
    \begin{center}
      BONUS \\ SLIDES
    \end{center}
  \end{Huge}
\end{frame}

\begin{frame}
  \frametitle{Comparison Between COMBNN and JFC}
\end{frame}

\begin{frame}
  \frametitle{Comparison Between JFC and MVx}
\end{frame}

\begin{frame}
  \frametitle{Comparison Between GAIA and JFC}
\end{frame}

\backupend

\end{document}
