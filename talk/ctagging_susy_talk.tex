\documentclass[usenames,dvipsnames]{beamer}
\geometry{papersize={12.8cm,9.6cm}}
\useoutertheme{infolines}

\usepackage{multirow}
\usepackage[normalem]{ulem}
\usepackage{cancel}
\usepackage{tikz}
\usepackage{colortbl}
\usetikzlibrary{arrows,shapes,backgrounds}
\tikzstyle{every picture}+=[remember picture]
\tikzstyle{na} = [baseline=-.5ex]
\tikzstyle{background grid}=[draw, black!50,step=.5cm]
\usepackage{feynmp-auto}
\usepackage{bbold}
\usepackage{bm}
\usepackage{booktabs}
\usepackage{siunitx}

\title[Charmed SUSY]{Charm jet identification in searches for new physics with the ATLAS detector}
\author[dhg3]{Dan Guest}
\institute[Yale]{Yale University}


\usecolortheme{beaver}
\setbeamercolor{item projected}{bg=darkred}
\setbeamertemplate{enumerate items}[default]
\setbeamertemplate{navigation symbols}{}
\setbeamercovered{transparent}
\setbeamercolor{block title}{fg=darkred}
\setbeamercolor{local structure}{fg=darkred}
\usefonttheme{serif} % default family is serif

\newcommand{\feyninc}[2]{\scalebox{#1}{\input{../misc/feyngen/#2}}}
\newcommand{\widegraphic}[1]{\includegraphics[width=\textwidth]{#1}}

\newcommand{\lagr}{\mathscr{L}}
\newcommand{\Lc}{\mathrm{L}}
\newcommand{\Rc}{\mathrm{R}}
\newcommand{\cmenergy}{\sqrt{s} = 8\,\text{TeV}}
\newcommand{\lumi}{\mathcal{L}}
\newcommand{\invfb}{\text{fb}^{-1}}
\newcommand{\lumiunct}{20.34 \pm 0.57\, \invfb}
\newcommand{\atlas}{ATLAS}
\newcommand{\supp}[1]{\tilde{#1}}
\newcommand{\neut}{\supp{\chi}_1^0}
\newcommand{\cha}{\supp{\chi}_{1}^{\pm}}
\newcommand{\scharm}{\supp{c}}
\newcommand{\sctoc}{\supp{c} \to c \neut}
\newcommand{\sttoc}{\supp{t} \to c \neut}
\newcommand{\su}[1]{\mathrm{SU}(#1)}
\newcommand{\alphas}{\alpha_{\mathrm{s}}}
\newcommand{\cutoff}{\Lambda_{\mathrm{UV}}}
\newcommand{\smcosmo}{\Lambda\text{CDM}}
\newcommand{\T}{\mathrm{T}}
\newcommand{\met}{E_{\rm T}^{\rm miss}}
\newcommand{\vmet}{\mathbf{E}_{\rm T}^{\rm miss}}
\newcommand{\pt}{p_{\rm T}}
\newcommand{\vpt}{\mathbf{p}_\T}
\newcommand{\mct}{m_{\rm CT}}
\newcommand{\mcc}{m_{cc}}
\newcommand{\mll}{m_{\ell \ell}}
\newcommand{\mt}{m_{\rm T}}
\newcommand{\metdphi}[1]{\Delta \phi(\met, #1)}
\newcommand{\wjets}{W + \text{jets}}
\newcommand{\zjets}{Z + \text{jets}}
\newcommand{\topbg}{top}
\newcommand{\ttbar}{t \bar{t}}
\newcommand{\cls}{\mathrm{CL}_{\mathrm{s}}}
\newcommand{\clb}{\mathrm{CL}_{\mathrm{b}}}
\newcommand{\clsb}{\mathrm{CL}_{\mathrm{s+b}}}
\newcommand{\gev}{\text{GeV}}
\newcommand{\tev}{\text{TeV}}

\newcommand{\crw}{CRW}
\newcommand{\crz}{CRZ}
\newcommand{\crt}{CRT}
\newcommand{\vrmcc}{VR $m_{cc}$}
\newcommand{\vrmct}{VR $m_{\rm CT}$}
\newcommand{\mctgt}[1]{\mct > #1\ \text{GeV}}
\newcommand{\meff}{m_{\mathrm{eff}}}

\definecolor{darkgreen}{rgb}{0,0.5,0}
\definecolor{aqua}{rgb}{0,1,1}
\definecolor{lightgray}{rgb}{0.95,0.95,0.95}


\newcommand{\backupbegin}{
   \newcounter{framenumberappendix}
   \setcounter{framenumberappendix}{\value{framenumber}}
}
\newcommand{\backupend}{
   \addtocounter{framenumberappendix}{-\value{framenumber}}
   \addtocounter{framenumber}{\value{framenumberappendix}}
}

\newcommand{\link}[2]{\underline{\href{#2}{#1}}}
\begin{document}

\section*{Intro}

\maketitle
\begin{frame}
  \frametitle{Contents}
  \tableofcontents
\end{frame}

\AtBeginSection[]
{
  \begin{frame}
  \frametitle{Contents}
  \tableofcontents[currentsection]
  \end{frame}
}

\section{Motivation}

\subsection{SUSY}
\begin{frame}
  \frametitle{Why SUSY?}
  \begin{columns}
    \begin{column}{0.5\textwidth}
      \begin{enumerate}
      \item Start with standard model
        \begin{itemize}
        \item \emph{huge} $\Delta m_h$ corrections
        \end{itemize}
      \item Add ``superpartners''
        \begin{itemize}
          \item bosons $\leftrightarrow$ fermions
        \end{itemize}
      %% \item Add a second $H$
      %% \item (more than) double the particles!
      \item Problem solved\ldots almost\ldots
      \end{enumerate}
    \end{column}
    \begin{column}{0.5\textwidth}
      \widegraphic{figures/external/standard-model.pdf}
    \end{column}
  \end{columns}
  \begin{itemize}
  \item In the simplest case, identical masses to SM
    \begin{itemize}
    \item Required to cancel Higgs corrections
    \end{itemize}
  \item Clearly not physical $\to$ add \emph{soft} terms to raise sparticle mass
      \begin{small}
      \begin{equation*}
        \Delta m_{h}^2 = m_{\text{soft}}^2 \left[ \frac{\Gamma}{16 \pi^2} \ln(\cutoff / m_{\text{soft}}) + \ldots \right]
      \end{equation*}
      \end{small}
  \end{itemize}
\end{frame}

%% \begin{frame}
%%   \frametitle{So\ldots 19 new particles? Isn't that ugly?}
%%   \begin{columns}
%%     \begin{column}{0.6\textwidth}
%%       Yes, but\ldots
%%       \begin{itemize}
%%       \item Gives a dark matter candidate ($\neut$)
%%       \item Useful for GUT unification
%%       \item In short, it fixes more than one problem
%%       %% \item Minimal SUSY models: visible at LHC
%%       %% \item ``solves'' hierarchy  problem \\
%%       %%   \ldots but only if superpartners are light
%%       \end{itemize}
%%     \end{column}
%%     \begin{column}{0.4\textwidth}
%%       \widegraphic{%
%% figures/external/bulletcluster_comp_f2048.jpg} \\
%%     \end{column}
%%   \end{columns}
%%   \begin{center}
%%     \includegraphics[width=0.8\textwidth]{%
%%       figures/external/unification-ugly.jpg}
%%   \end{center}
%% \end{frame}

\begin{frame}
  \frametitle{Motivation: LHC SUSY}
  \begin{itemize}
    %% \item SUSY ``solves'' the hierarchy problem
    %% \item Hierarchy only solved with light particles
    %%   \begin{small}
    %%   \begin{equation*}
    %%     \Delta m_{h}^2 = m_{\text{soft}}^2 \left[ \frac{\Gamma}{16 \pi^2} \ln(\cutoff / m_{\text{soft}}) + \ldots \right]
    %%   \end{equation*}
    %%   \end{small}
    %% \item The largest Yukawa coupling ($\Gamma$) is to $t$
    %%   \begin{itemize}
    %%   \item needs to be canceled by a relatively light $\tilde{t}$
    %%   \end{itemize}
    \item If natural SUSY exists, we could see it at the LHC
      %% \begin{itemize}
      %% \item \ldots but the easy searches have come back negative
      %% %% \item All the more reason to look at data in new ways
      %% \end{itemize}
    \item Focus on ``simplified'' models: only one or two low-mass sparticles (e.g. $\tilde{t}$ + $\neut$)
      %% \begin{itemize}
      %%   \item 
      %% \end{itemize}
    \item In our case, assume $R$ parity, and that the $\neut$ is the lightest SUSY particle (LSP)
  \end{itemize}
  \begin{center}
    %% \includegraphics[width=0.5\textwidth]{%
    %%   figures/external/bulletcluster_comp_f2048.jpg}
  \end{center}
\end{frame}

\subsection{Charmed Searches}
\begin{frame}
  \frametitle{Charm-tagged Searches: $\sttoc$}
  \begin{columns}
    \begin{column}{0.4\textwidth}
      \begin{itemize}
      \item Many $\tilde{t}$ decays
      \item Depend on $m_{\tilde{t}}$ and $m_{\neut}$
      \end{itemize}
      \widegraphic{figures/external/feyn-sttoc-isr.pdf}\\
      \begin{itemize}
      %% \item Assume $m_{\tilde{t}} > m_{\neut}$
      %% \item All other sparticles heavy
      %% \item ``compressed'': $m_{\tilde{t}} - m_{\neut} < m_W$
      \item $m_{\tilde{t}} - m_{\neut} < m_W$ forces $\tilde{t}$ decay via $c$-jets
      %% \item Signature: boosted, $c$-jets + $\met$
      %% \item See \link{arXiv 1407.0608}{http://arxiv.org/abs/1407.0608}
      %% \item \textcolor{red}{Not discussed in detail here (ADD LINK)}
      \end{itemize}
    \end{column}
    \begin{column}{0.6\textwidth}
      \begin{center}
        \includegraphics[width=\textwidth]{%
          figures/external/stop.pdf}
        %% \includegraphics[width=\textwidth]{%
        %%   figures/external/blocked-stop-lsp.pdf}
      \end{center}
    \end{column}
  \end{columns}
\end{frame}


\begin{frame}
  \frametitle{Scalar charm ($\sctoc$)}
  \begin{center}
  \includegraphics[width=0.3\textwidth]{figures/external/feyn-sctoc.pdf}
  \begin{itemize}
  \item Same general signature as $\sttoc$, $m_{\scharm} - m_{\neut} \nless m_W$
  \item ``Light'' ($udcs$) $\tilde{q}$ masses needn't be degenerate (see \link{arXiv 1212.3328}{http://arxiv.org/abs/1212.3328})
  \item \textbf{Present talk focuses on this search} (also \link{arXiv 1501.01325}{http://arxiv.org/abs/1501.01325})
  \end{itemize}
  \end{center}
\end{frame}

\section{The ATLAS Experiment}
\begin{frame}
  \frametitle{The LHC / ATLAS}
  \begin{itemize}
  \item Big experiment (27 km long), $\sim 3000$ authors on ATLAS
  \item High energy: $7,8,13\,\tev$\ldots
  \item ``as much power as a city'' ($\approx$ 50 MW, 4\% of Geneva, pop 470k)
  \item More importantly, it collides \textbf{a lot of protons}
  \end{itemize}
  \begin{columns}
    \begin{column}{0.5\textwidth}
      \begin{itemize}
      \item bunch crossing: 20--40 MHz
      \item collisions / crossing (ATLAS):
        \begin{description}[Run 2:]
        \item[Run 1:] 20
        \item[Run 2:] $\sim 100$
        \end{description}
      \item Fills take 2 hours, last 1 day
      \item $\sim 1000$ bunches / fill
      \end{itemize}
    \end{column}
    \begin{column}{0.5\textwidth}
      \widegraphic{figures/external/cern-from-air.jpg}
    \end{column}
  \end{columns}
  %% \begin{itemize}
  %% \item Taking data requires a three-level trigger
  %% \end{itemize}
\end{frame}

\begin{frame}
  \frametitle{The interaction point}
  \includegraphics[width=\textwidth]{figures/external/pileup-vertices.png}
  \begin{itemize}
  \item In 2012, $\sim 20$ vertices per bunch crossing
  \item Vertices inflated to 20 times actual size
  \item Inner pixel layer (gray) is approximately 5 cm from the beam
  \item For a $10\,\gev$ track, transverse resolution is $\sim 30\,\mathrm{\mu m}$
  \item \textbf{Flavor-tagging relies on displaced vertices, can travel $\bm{\mu}$m--cm}
  \end{itemize}
\end{frame}

\begin{frame}
  \frametitle{Heavy Quark Decays}
  \begin{itemize}
  \item Quark mixing angles measured to extraordinary precision
  \item For the sake of flavor-tagging, we make some approximations:
    \begin{equation*}
      %% V_{\text{CKM}} =
      \begin{pmatrix}
        \ckmrowabs{u} \\ \ckmrowabs{c} \\ \ckmrowabs{t}
      \end{pmatrix} = \ckmmatrixapprox \approx \ckmparticle
    \end{equation*}
  \end{itemize}
  \begin{columns}
    \begin{column}{0.5\textwidth}
      \begin{center}
        \begin{tabular}{l S[table-format=3.0e1] c} % [table-format=1.3e1]
          Decay & $c\tau$ [$\mu$m] & Hadron \\ \hline
          $t \to b$ & 150E-12 & --- \\
          $b \to c$ & 492  & $B$ \\
          $c \to s$ & 312  & $D^0$ \\
          $s \to u$ & 27E3  & $K_{\mathrm{s}}$
        \end{tabular}
      \end{center}
    \end{column}
    \begin{column}{0.5\textwidth}
      \begin{center}
        \includegraphics[width=\textwidth]{figures/external/sm-t-decay.pdf}
      \end{center}
    \end{column}
  \end{columns}
\end{frame}

\begin{frame}
  \frametitle{The Calorimeter (Dark Matter and Neutrino Detector)}
  \widegraphic{figures/external/atlas-cal.jpg}
\end{frame}

\begin{frame}
  \frametitle{The ATLAS Experiment}
  \widegraphic{figures/external/atlas-medium.jpeg}
  %% \begin{itemize}
  %% \item Start outside, move inward
  %% \item mention calo as ``neutrino'' detector
  %% \item Introduce $\pt$, $\met$
  %% \item end with pixels and tracking / magnetic field
  %% \end{itemize}
\end{frame}

\begin{frame}
  \frametitle{The Standard Model to ATLAS}
  %% \begin{itemize}
  %% \item Many particles are constructed from a few basic pieces
  %% \end{itemize}
  \begin{columns}
    \begin{column}{0.35\textwidth}
      \begin{itemize}
      \item \textcolor{darkgreen}{\textbf{Some are stable}}
        %% \begin{itemize}
        %% \item $\mu$ spectrometer
        %% \item Calorimeter
        %% \end{itemize}
      \item \textcolor{red}{\textbf{Some form a secondary vx}}
        %% \begin{itemize}
        %% \item Can form secondary 
        %% %% \item Easily confused
        %% \end{itemize}
      \item \colorbox{aqua}{Neutrinos $\to \met$}
        %% \begin{itemize}
        %% \item $\met$
        %% \item Symmetries
        %% \end{itemize}
      %% \item Light jets ($u$, $d$, $s$) aren't distinguished
      \end{itemize}
    \end{column}
    \begin{column}{0.65\textwidth}
      \widegraphic{figures/external/sm-detectable.pdf}
    \end{column}
  \end{columns}
  \begin{itemize}
      \item \fcolorbox{black}{lightgray}{The rest decay instantly, reconstructed from decay products}
  \end{itemize}
\end{frame}


\section{Flavor Tagging}

\begin{frame}
  \frametitle{Flavor Tagging Basics}
  \begin{columns}
    \begin{column}{0.5\textwidth}
      \begin{itemize}
      \item Secondary vertex based
      \item Must separate \textcolor{red}{$b$}, \textcolor{darkgreen}{$c$}, and light
      %% \item All decay within beam-pipe
      \item Approximate decay distance  \\
        $\beta \gamma c \tau \approx 6.4\,\mathrm{mm}$, $B$ @ $70\,\gev$
      \item Requires both tracking resolution and luck
      %% \item \textcolor{darkgreen}{$c$-decays} are part of \textcolor{red}{$b$-decays}
      \end{itemize}
      \begin{center}
        \includegraphics[width=\textwidth]{figures/external/mv1.pdf}
      \end{center}
    \end{column}
    \begin{column}{0.4\textwidth}
      \begin{center}
        \widegraphic{figures/external/b-jet.pdf}\\[0.1cm]
        \includegraphics[width=\textwidth]{figures/external/sm-b-decay.pdf}
      \end{center}
    \end{column}
  \end{columns}
  %% \begin{tiny}
  %%   Figure credit: (lower right) Wikipedia ``Standard Model''
  %% \end{tiny}
\end{frame}

\subsection{Discriminants}

\begin{frame}
  \frametitle{Turning Jets into Discriminants}
  \begin{columns}[t]
    \begin{column}{0.5\textwidth}
      \begin{center}
        Impact Parameter (IP)
      \widegraphic{figures/external/b-jet-ip.pdf}
      \end{center}
      \begin{itemize}
      \item Extrapolate tracks to perigee
      \item Combine IPs
      \end{itemize}
    \end{column}
    \vline{}
    \begin{column}{0.5\textwidth}
      \begin{center}
        Secondary Vertex (SV)
      \includegraphics[width=0.8\textwidth]{figures/external/b-jet-sv.pdf}
        \end{center}
      \begin{itemize}
      \item Fit secondary vertex to tracks
      \item Use SV properties (mass, displacement from PV, etc)
      \end{itemize}
    \end{column}
  \end{columns}
\end{frame}



\begin{frame}
  \frametitle{More Discriminants}
  \begin{columns}
    \begin{column}{0.5\textwidth}
      JetFitter:
      \begin{enumerate}
      \item Fits a ``Flight Line'' from the PV along jet axis
      \item Line updated to cross tracks
      \item Vertices clustered
      \item Vertex properties extracted
      \end{enumerate}
      Useful for $c$-tagging
      \begin{itemize}
      \item Can resolve any number of SVs
      \item Retuned for $c$-tagging:
        \begin{itemize}
        \item Looser track selection
        \item Fewer tracks assigned to PV
        \end{itemize}
      \end{itemize}
    \end{column}
    \begin{column}{0.5\textwidth}
        \begin{center}
          JetFitter
      \widegraphic{figures/external/b-jet-jf.pdf}
        \end{center}
    \end{column}
  \end{columns}
\end{frame}

\begin{frame}
  \frametitle{List of Tagging Variables}
  \resizebox{\textwidth}{!}{\begin{tabular}{c | c | c }
  Algorithm             & Variable Name & Description \\
  \hline
  \multirow{4}{*}{Kinematic}
  & \multirow{2}{*}{$p_{T}^{\rm cat}$} & $\pt$ category of	 jet, divisions [GeV]: \\
  &                & \catpt \\
  & \multirow{2}{*}{$\eta^{\rm cat}$}  & $|\eta|$ category of jet, divisions: \\
  &                &  \cateta \\
  \hline
  \multirow{1}{*}{IP3D} & $\log (\mathcal{L}_{b}/ \mathcal{L}_{\rm light})$ & log ratio between $b$-jet and light-jet likelihood value \\
  \hline
  \multirow{4}{*}{SV1}               & $n_{\rm trk}^{{\rm SV1}}$     & Number of tracks matched to the vertex \\
  & $n_{\rm 2t}$      & Number of two-track vertices found in the jet
  \\
  & $m_{\rm vx}$      & Secondary vertex mass \\
  & $L / \sigma_{L}$ & Secondary vertex flight-length significance \\
  \hline

  \multirow{10}{*}{JetFitter}
  & $m_{\rm chain}$      & Invariant mass of decay products \\
  & $S_d^{\rm JF}$ & Total vertex flight-length significance \\
  & $n_{\rm vx}$      & Number of reconstructed vertices with  $\ge 2$ tracks \\
  & $n_{\rm trk}^{\rm JF}$     & Number of tracks matched to vertices with $\ge 2$ tracks \\
  & $n_{\rm 1t}$      & Number of single-track vertices\\
  & $L_{xy}^{1}$ & Transverse displacement of the secondary vertex \\
  & $L_{xy}^{2}$ & Transverse displacement of the tertiary vertex \\
  & $\min \varphi_{\rm trk}$ & Minimum track rapidity along jet axis \\
  & $\langle \varphi_{\rm trk} \rangle$ & Mean track rapidity along jet axis \\
  & $\max \varphi_{\rm trk} $ & Maximum track rapidity along jet axis \\
  \hline

  %% DG: the secondary vertex mass and flight length significance
  %%     are listed separately for JF and SV1, since they aren't
  %%     computed identically.

  %% & $m_{\rm vx}$      & Secondary vertex mass \\
  SV1, JetFitter& \multirow{2}{*}{$E_{\rm vx}/E_{\rm jet}$}  & Ratio of the vertex track energy sum \\
  (variables input from both)         &                  & to the jet track energy sum \\
  %% & $d / \sigma_{d}$ & Secondary vertex flight-length significance \\

\end{tabular}
}
\end{frame}

\subsection{Neural Nets}

\begin{frame}
  \frametitle{Neural Nets}
  \begin{columns}
    \begin{column}{0.5\textwidth}
      \begin{itemize}
      \item No single good discriminant
      %% \item Name is biologically inspired
      %% \item Have \textbf{absolutely nothing} to do with biology
      \item Good modeling in simulation
      \item Inputs: jet characteristics
      \item Outputs: flavor probabilities
      %% \item Layers composed of \emph{nodes}
      %% \item Each node
      %% \item Output layer: anything you want
      %% \item hidden layers: no obvious physical meaning
      \end{itemize}
    \end{column}
    \begin{column}{0.5\textwidth}
      \widegraphic{figures/tagging-graphs/simple-nn_gen.pdf} \\
    \end{column}
  \end{columns}
  %% \begin{columns}
  %%   \begin{column}{0.5\textwidth}
  \begin{itemize}
    \item Simple mathematical form: given \fcolorbox{black}{green}{inputs $\vect{x}$} and \fcolorbox{black}{orange}{outputs $\vect{y}$}
      \begin{equation*}
          \vect{y} = \mathcal{N}(\vect{x}) = A_{N-1} f( \cdots  A_2 \underbrace{f(A_1 \vect{x} + \vect{b}_1)}_{\vect{z_2}} + \vect{b}_2 \cdots  ) + \vect{b}_{N-1}
      \end{equation*}

  \item $f$ is the element-wise \emph{activation function}: $f(x) = \frac{1}{1 + e^{-x}}$
  \item The problem: what are $A_i$ and $\vect{b}_i$?
  \item Selected at random, ``trained'' using \textbf{simulated} data
  %% \item Trained via backpropagation (steepest descent)
  \end{itemize}
  %%   \end{column}
  %%   \begin{column}{0.5\textwidth}
  %%     \widegraphic{figures/external/logistic-curve.pdf}
  %%   \end{column}
  %% \end{columns}
\end{frame}

\begin{frame}
  \frametitle{Backpropagation}
  \begin{itemize}
  \item Draw inputs $\vect{x}$ and target values $\vect{t}$ from a training sample
  \end{itemize}
  \begin{columns}
    \begin{column}{0.5\textwidth}
      \begin{itemize}
      \item Calculate outputs $\vect{y}$
      \item Calculate error \textcolor{red}{$\vect{\xi}_{N} \equiv (\vect{y} - \vect{t})$}
      \item Backpropagate
        \[\textcolor{red}{\vect{\xi}_i} \equiv (A_i^{\intercal} \textcolor{red}{\vect{\xi}_{(i+1)}}) \circ f'(\vect{z}_i)\]
      \end{itemize}
    \end{column}
    \begin{column}{0.5\textwidth}
      \widegraphic{figures/tagging-graphs/backprop_gen.pdf} \\
    \end{column}
  \end{columns}
  \begin{itemize}
  \item Update \begin{align*} A_i &\to A_i - \gamma \vect{\xi}_{(i+1)}g(\vect{z}_i)^{\intercal} & \vect{b}_i &\to \vect{b}_i - \gamma \vect{\xi}_{(i+1)}
  \end{align*}
    where $\gamma$ is the \emph{learning rate}, $g \equiv f$ when $i \neq N-1$
  \item \textbf{This has problems for large NNs} (more on this later)
  \end{itemize}
\end{frame}

\subsection{Charm Tagging}

\begin{frame}
  \frametitle{Charm Tagger Overview}
  \begin{columns}
    \begin{column}{0.6\textwidth}
      \begin{itemize}
      \item Combine all three vertex finders
      \item IP based: IP3D
        \begin{itemize}
        \item {2D IP} for each track
        \item Lookup \fcolorbox{black}{aqua}{$\mathcal{L}_b$}, \fcolorbox{black}{aqua}{$\mathcal{L}_{\text{light}}$} for each track (based on simulation)
        \item Calculate $\sum_{\text{trk}} \ln (\mathcal{L}_b / \mathcal{L}_{\text{light}})$
        \end{itemize}
      \item Add {vertex variables}, e.g.
        \begin{itemize}
        \item Vertex significance
        \item Vertex mass
        \item Fraction of jet energy in vertices
        \end{itemize}
      \item Feed all variables into {a NN}
      \item Cut on neural net outputs: \\
        $P_{b}$, $P_{c}$, and $P_{u}$ ($P_{\rm light}$)
      \end{itemize}
    \end{column}
    \begin{column}{0.4\textwidth}
      \widegraphic{figures/jfc/dot/simple-arch.pdf}
    \end{column}
  \end{columns}
  \vspace{1em}
    \begin{tiny}
      PUB Note: \texttt{\url{https://atlas.web.cern.ch/Atlas/GROUPS/PHYSICS/PUBNOTES/ATL-PHYS-PUB-2015-001/}}
    \end{tiny}
\end{frame}

%% \begin{frame}
%%   \frametitle{Backpropagation}
%%   \begin{itemize}
%%   \item Explain backprop
%%   \end{itemize}
%% \end{frame}

%% \begin{frame}
%%   \frametitle{The ATLAS Flavor-tagging framework}
%%   \begin{columns}
%%     \begin{column}{0.8\textwidth}
%%     \widegraphic{%
%% figures/tagging-graphs/tagging-full-nogaia.pdf} \\
%% %%     \includegraphics[width=0.8\textwidth]{%
%% %% figures/tagging-graphs/tagging-leg.pdf}
%%     \end{column}
%%     \begin{column}{0.2\textwidth}
%%       \widegraphic{figures/tagging-graphs/tagging-leg-vert.pdf}
%%     \end{column}
%%   \end{columns}
%%   \begin{itemize}
%%   \item MV1 was the standard $b$-tagger for run 1, note the number of upstream algorithms
%%   \item For a deep NN, training is \textbf{really} hard
%%   \end{itemize}
%% \end{frame}

%% NOTE: should make sure I mention that there's no clean cut to isolate
%% c-jets or b-jets
\begin{frame}
  \frametitle{Operating Point ($c$-jet definition)}
  \begin{itemize}
  \item Define two discriminants based on NN outputs:
  \begin{align*}
    \text{anti-}b & \equiv \frac{P_c}{P_b} &
    \text{anti-light} & \equiv \frac{P_c}{P_{\text{light}}}
  \end{align*}
  \item Can trade $b$-rejection for light-rejection
  \item Defined operating point with $\sttoc$ search
  \end{itemize}
  \begin{columns}
    \begin{column}{0.5\textwidth}
      \widegraphic{figures/jfc/2d-cut.pdf}
    \end{column}
    \begin{column}{0.5\textwidth}
      \widegraphic{figures/external/rejrej-simple.pdf}
    \end{column}
  \end{columns}
\end{frame}

\newcommand{\calnote}{https://atlas.web.cern.ch/Atlas/GROUPS/PHYSICS/CONFNOTES/ATLAS-CONF-2014-046/}
\begin{frame}
  \frametitle{Charm-Tagging Scale Factors}
  \begin{columns}
    \begin{column}{0.6\textwidth}
      \begin{itemize}
      \item Standard $b$-tagging calibrations:
        \begin{itemize}
        \item[light] \link{Negative tags}{\calnote} (right)\tikz[na] \coordinate(neg-tag);
        \item[$b$] \link{dileptonic $t \bar{t}$}{https://atlas.web.cern.ch/Atlas/GROUPS/PHYSICS/CONFNOTES/ATLAS-CONF-2014-004/} (bottom right)\tikz[na] \coordinate(dilep-tt);
        \item[$c$] \link{$D^*$ reconstruction}{\calnote} (bottom)\tikz[na] \coordinate(dzero);
        \end{itemize}
      \end{itemize}
      \widegraphic{%
figures/external/sf-ctag-c-medium.pdf}
    \end{column}
    \begin{column}{0.4\textwidth}
      \widegraphic{%
figures/external/sf-ctag-u-medium.pdf} \\
      \widegraphic{%
figures/external/sf-ctag-b-medium.pdf} \\
    \end{column}
  \end{columns}
\end{frame}

%% \begin{frame}
%%   \frametitle{Interlude: Deep Learning}
%%   \begin{columns}
%%     \begin{column}{0.5\textwidth}
%%     \end{column}
%%     \begin{column}{0.4\textwidth}
%%       \widegraphic{figures/external/autoencoder-layer.png}
%%     \end{column}
%%   \end{columns}
%%   \begin{tiny}
%%     figure credit: \url{http://ufldl.stanford.edu/wiki/index.php/Stacked_Autoencoders}
%%   \end{tiny}
%% \end{frame}


\section{The $\sctoc$ Search}

\begin{frame}[fragile=singleslide]
  \frametitle{Back to Searches: $\sctoc$}
  \begin{columns}
    \begin{column}{0.5\textwidth}
      \begin{itemize}
      %% \item \link{arXiv 1212.3328}{http://arxiv.org/abs/1212.3328}
      \item Signature: \textcolor{darkgreen}{$c$-jets} + \textcolor{red}{$\met$}
      \item Trigger on $\met$ (\verb|EF_xe80_tclcw_tight|)
      %% \item $c$-tag two leading jets
      \item Veto leptons
        \begin{itemize}
        \item[$\mu$:] $\pt > 6\,\gev$
        \item[$e$:] $\pt > 7 \,\gev$
        \end{itemize}
      \end{itemize}
    \end{column}
    \begin{column}{0.5\textwidth}
      \feyninc{1.0}{scsc-ccN1N1}      %% \widegraphic{figures/external/feyn-sctoc.pdf}
    \end{column}
  \end{columns}
  \begin{itemize}
  \item Difficulties:
    \begin{itemize}
    \item As with all 0L searches, large multijet background
    \item $W$, $Z$, and $t$ decaying to $\nu$
    \end{itemize}
  \end{itemize}
\end{frame}

\subsection{Backgrounds}

\newcommand{\feynincstd}[1]{\feyninc{1.0}{#1}}
\begin{frame}
  \frametitle{Backgrounds: Multijet QCD}
  \begin{columns}
    \begin{column}{0.5\textwidth}
      \feyninc{0.9}{multijet} \\[0.2cm]
      \begin{itemize}
      \item Most removed by $\met > 150\,\gev$
        \item \textcolor{red}{one mismeasured jet} can result in large $\met$
      \end{itemize}
    \end{column}
    \begin{column}{0.5\textwidth}
    \widegraphic{%
int/figures/stackplots/dans/preselection/jetmet_dphi.pdf}
    \end{column}
  \end{columns}
  \begin{itemize}
  \item Require $\Delta \phi(j_{\{1,2,3\}},\met) > 0.4$
  \item Remaining QCD mostly light flavor $\to$ removed by $c$-tagging
  \item Estimated with ``jet smearing'' (see \url{https://cdsweb.cern.ch/record/1423310}), small after all cuts
  \end{itemize}
\end{frame}

\begin{frame}
  \frametitle{Backgrounds: $W$/$Z$ + jets}
  \begin{columns}
    \begin{column}{0.5\textwidth}
    \feynincstd{vjets} \\[0.2cm]
      \begin{itemize}
      %% \item After anti-QCD cuts, backgrounds are $W$/$Z$ + jets
      \item \textcolor{red}{Real $\met$} from $Z \to \nu \nu$ and $W \to \tau \nu$
      \end{itemize}
    \end{column}
    \begin{column}{0.5\textwidth}
      \begin{center}
        After QCD cuts
      \widegraphic{%
int/figures/stackplots/dans/preselection/j0_flavor_truth_label.pdf}
      \end{center}
    \end{column}
  \end{columns}
  \begin{itemize}
    \item Jets are dominantly light flavored, removed by two $c$-tag requirement (each tag: $1/\epsilon_{\text{light}} \approx 200$)
  \end{itemize}
\end{frame}

\begin{frame}
  \frametitle{Backgrounds: $t\bar{t}$}
  \begin{columns}
    \begin{column}{0.6\textwidth}
      \begin{center}
        \feynincstd{ttbar} \\[0.2cm]
      \end{center}
    \begin{itemize}
    \item Large $t\bar{t}$ background remains
    \item Kinematic cuts on leading two jets:
      \begin{itemize}
      \item[$\mcc$]: also removes QCD
      \item[$\mct$]: if $b$-jets are leading, $\mct \lesssim 135\,\gev$
      \end{itemize}
    \end{itemize}
    \begin{tiny}
 $\mct \equiv \{[ E_\T(v_{1}) + E_\T(v_{2}) ]^2 -  [  \vpt(v_{1}) - \vpt(v_{2}) ]^2\}^{1/2}$
    \end{tiny}
    \end{column}
    \begin{column}{0.4\textwidth}
      \widegraphic{%
int/figures/stackplots/dans/signal_mct150/mass_cc_blinded.pdf}\\
      \widegraphic{%
int/figures/stackplots/dans/signal_mct150/mass_ct_blinded.pdf}
    \end{column}
  \end{columns}
\end{frame}

%% \begin{frame}
%%   \frametitle{Discriminants}
%%   \begin{itemize}
%%   \item $\mct$
%%   \item $c$-tagging distributions?
%%   \end{itemize}
%% \end{frame}

\subsection{Control Regions}

\begin{frame}
  \frametitle{Control Regions: $Z$ + jets, $W$ + jets, $t\bar{t}$}
  \begin{columns}
    \begin{column}{0.6\textwidth}
      \begin{itemize}
      \item All regions use same $c$-tagging
      \item All CR use single-lepton triggers
      \item Different offline lepton requirements
      \item[$Z$]: 2 $e$ or 2 $\mu$
        \begin{itemize}
        \item Opposite sign, in $Z$-peak
        \item Leptons added to $\met$ to mimic $Z \to \nu\nu$
        \end{itemize}
      \item[$W$]: $e$ or $\mu$, $\pt > 50\,\gev$
        \begin{itemize}
        \item $\mct$ window, $\mct > 150$
        \item largely $t\bar{t}$
          \begin{itemize}
          \item  constrained by $t\bar{t}$ CR
          \end{itemize}
        \end{itemize}
      \item[$t\bar{t}$]: 1 $e$, 1 $\mu$
        \begin{itemize}
        \item Opposite sign
        \item Very pure, decent stats ($>100$)
        \end{itemize}
      \end{itemize}
    \end{column}
    \begin{column}{0.4\textwidth}
      \widegraphic{int/figures/stackplots/dans/cr_z/mass_ll.pdf} \\
      \widegraphic{int/figures/stackplots/dans/cr_w/mass_t.pdf}
    \end{column}
  \end{columns}
\end{frame}


\subsection{Results}

\begin{frame}
  \frametitle{Results}
  \begin{columns}
    \begin{column}{0.5\textwidth}
\widegraphic{int/figures/stackplots/dans/signal_mct150/mass_ct_afterFit.pdf}
    \end{column}
    \begin{column}{0.5\textwidth}
      \widegraphic{int/figures/stackplots/dans/signal_mct150/mass_cc_afterFit.pdf}
    \end{column}
  \end{columns}
  \begin{itemize}
  \item After unblinding, no excess found
  \item \textbf{Three signal regions:} $\mct > 150,\ 200,\ 250\,\gev$
  \item Roughly $1.5\sigma$ deficit
  \item Move on to setting limits
  \end{itemize}
\end{frame}

\begin{frame}
  \frametitle{Systematics}
  \begin{columns}
    \begin{column}{0.5\textwidth}
    \resizebox{\textwidth}{!}{\begin{tabular}{lccc}
\noalign{\smallskip}\hline\noalign{\smallskip}
{\bf Uncertainty of channel} & $m_{\rm CT} > 150$ & $m_{\rm CT} > 200$ & $m_{\rm CT} > 250$\\ 

\noalign{\smallskip}\hline\noalign{\smallskip}
%%
Total background expectation             &  $29.60$        &  $16.30$        &  $8.21$       \\
%% \\
\noalign{\smallskip}\hline\noalign{\smallskip}
%%
Total statistical $(\sqrt{N_{\rm exp}})$              & $\pm 5.44$        & $\pm 4.04$        & $\pm 2.87$       \\
%%
Total background systematic               & $\pm 5.57\ [18.82\%] $        & $\pm 3.46\ [21.22\%] $        & $\pm 1.94\ [23.61\%] $             \\
\noalign{\smallskip}\hline\noalign{\smallskip}
\noalign{\smallskip}\hline\noalign{\smallskip}
%%
\rowcolor{red}
$\epsilon_{\text{tag}}$ for light-jets & $\pm 5.59$  & $\pm 3.38$  & $\pm 1.95$       \\
%%
\rowcolor{green} $W$ + jets normalization &  $\pm 5.31$           &  $\pm 3.03$           &  $\pm 1.39$       \\
%%
\rowcolor{aqua} jet energy scale &  $\pm 3.26$           &  $\pm 1.84$           &  $\pm 1.12$       \\
%%
\rowcolor{green} $Z$ + jets normalization &  $\pm 3.21$           &  $\pm 1.73$           &  $\pm 0.97$       \\
%%
\rowcolor{red} $\epsilon_{\text{tag}}$ for $c$-jets &  $\pm 2.46$           &  $\pm 1.33$           &  $\pm 0.67$       \\
%%
combined top theory &  $\pm 2.13$           &  $\pm 1.32$           &  $\pm 0.57$       \\
%%
$m_{\rm CT} > 150$ stat error &  $\pm 1.86$          & $\pm 0.00$          & $\pm 0.00$       \\

%%
\rowcolor{red}
$\epsilon_{\text{tag}}$ for $b$-jets &  $\pm 1.69$           &  $\pm 0.98$           &  $\pm 0.51$       \\
%%
\rowcolor{green} top normalization &  $\pm 1.12$           &  $\pm 0.58$           &  $\pm 0.24$       \\
%%
$E_{\rm T}^{\rm miss}$ resolution &  $\pm 0.96$           &  $\pm 0.52$           &  $\pm 0.27$       \\
%%
Lumi         & $\pm 0.82$          & $\pm 0.45$          & $\pm 0.23$       \\
%%
combined $W$ theory &  $\pm 0.82$           &  $\pm 0.69$           &  $\pm 0.35$       \\
%%
combined $Z$ theory &  $\pm 0.80$           &  $\pm 0.69$           &  $\pm 0.26$       \\
%%
$E_{\rm T}^{\rm miss}$ trigger SF &  $\pm 0.59$           &  $\pm 0.33$           &  $\pm 0.16$       \\
%%
flat ``other'' systematic &  $\pm 0.22$           &  $\pm 0.22$           &  $\pm 0.22$       \\
%%
jet energy resolution &  $\pm 0.22$           &  $\pm 0.03$           &  $\pm 0.14$       \\
%%
$E_{\rm T}^{\rm miss}$ scale &  $\pm 0.11$           &  $\pm 0.10$           &  $\pm 0.19$       \\
%%
$\epsilon_{\text{tag}}$ for $\tau$-jets &  $\pm 0.06$           &  $\pm 0.06$           &  $\pm 0.04$       \\
%%
pileup &  $\pm 0.05$           &  $\pm 0.14$           &  $\pm 0.02$       \\
%%
$e$ resolution from $Z \to ee$ &  $\pm 0.01$           &  $\pm 0.00$           &  $\pm 0.00$       \\
%%
jet vertex fraction &  $\pm 0.00$           &  $\pm 0.00$           &  $\pm 0.00$       \\
%%
$e$ ID SF &  $\pm 0.00$           &  $\pm 0.00$           &  $\pm 0.00$       \\
%%
%%
%%
$\mu$ energy scale &  $\pm 0.00$           &  $\pm 0.00$           &  $\pm 0.00$       \\
%%
$e$ low $p_{\rm T}$ &  $\pm 0.00$           &  $\pm 0.00$           &  $\pm 0.00$       \\
%%
$m_{\rm CT} > 250$ stat error &  $\pm 0.00$          & $\pm 0.00$          & $\pm 0.85$       \\

%%
$m_{\rm CT} > 200$ stat error &  $\pm 0.00$          & $\pm 1.38$          & $\pm 0.00$       \\

%%
%%
%%
$\mu$ ID SF &  $\pm 0.00$           &  $\pm 0.00$           &  $\pm 0.00$       \\
%%
\noalign{\smallskip}\hline\noalign{\smallskip}
\end{tabular}
%%
}
    \end{column}
    \begin{column}{0.5\textwidth}
      \begin{itemize}
      \item Major backgrounds estimated from global fit
      \item Main causes of systematic errors:
        \begin{enumerate}
        \item \fcolorbox{red}{red}{Tagging}
        \item \fcolorbox{green}{green}{Limited CR statistics}
        \item \fcolorbox{aqua}{aqua}{Jet Energy Scale}
        \end{enumerate}
      \item Common tagging requirement mitigates systematic \ldots
      \item \ldots but limits CR stats
      \end{itemize}
    \end{column}
  \end{columns}
\end{frame}


\begin{frame}
  \frametitle{Exclusion Plane}
  \begin{itemize}
  \item Showing combined exclusion for all three regions
  \item Region with ``best'' expected $\cls$ taken for each point
  \end{itemize}
  \begin{columns}
    \begin{column}{0.5\textwidth}
      \widegraphic{int/figures/limit_tree/full_exclusion/upper_limits/scharm_combined.pdf}
    \end{column}
    \begin{column}{0.5\textwidth}
      \begin{itemize}
      \item Excludes up to $m_{\neut} = 200\,\gev$, $m_{\scharm} = 490\,\gev$
      \item Region where $m_{\scharm} < m_{\neut}$ is kinematically forbidden
      \item What's happening where $m_{\scharm} - m_{\neut} \lesssim 150\,\gev$?
        \begin{itemize}
        \item Lower $\Delta m$
        \item $c$ jets aren't resolved
        \item $\met$ is small
        \end{itemize}
      \end{itemize}
    \end{column}
  \end{columns}
\end{frame}

\begin{frame}
  \frametitle{Near the Diagonal: $\scharm$ Needs a Boost}
  \begin{itemize}
  %% \item Problems for $\Delta m \equiv m_{\scharm} - m_{\neut} \lesssim 150\,\gev$:
  %%   \begin{itemize}
  %%   \item Less $\met$, trigger won't fire
  %%   \item Less energetic $c$-jets, below $50\,\gev$ threshold
  %%   \end{itemize}
  \item We need a \textcolor{orange}{boosted} signal region
    \begin{itemize}
    \item Already exists in the $\sttoc$ search
    \item Even better: simplified SUSY models mean $\sigma_{\tilde{t}\tilde{t}} \approx \sigma_{\tilde{c}\tilde{c}}$
    \item We can use a simple reinterpretation of $\sttoc$
    \end{itemize}
  \end{itemize}
  \begin{columns}
    \begin{column}{0.5\textwidth}
      \begin{center}
        \vspace{0.1cm}
        \feyninc{1.0}{scsc-ccN1N1-boost}
      \end{center}
    \end{column}
    \begin{column}{0.5\textwidth}
    \widegraphic{misc/cc/c1c2-exclusion.pdf}
    \end{column}
  \end{columns}
\end{frame}

\begin{frame}
  \frametitle{Combined exclusion in the $\scharm$--$\neut$ plane}
  \begin{columns}
    \begin{column}{0.6\textwidth}
      \widegraphic{int/figures/limit_tree/full_exclusion/exclusion_best.pdf}
    \end{column}
    \begin{column}{0.4\textwidth}
      \begin{itemize}
      \item Dots indicate best expected $\cls$
      \item C1 and C2 regions cover $\Delta m \lesssim 100\,\gev$
      \item Larger $\Delta m$ covered by larger $\mct$ regions
      \end{itemize}
    \end{column}
  \end{columns}
\end{frame}

\begin{frame}
  \frametitle{Comparison with $\tilde{q} \to \mathrm{0L} + \text{jets}$ search}
  \begin{columns}
    \begin{column}{0.6\textwidth}
      \widegraphic{int/figures/limit_tree/full_exclusion/exclusion_inclusive.pdf}
    \end{column}
    \begin{column}{0.4\textwidth}
      \begin{itemize}
      \item Extends $\tilde{q}$ limits (non-degenerate $\tilde{q}$ masses)
      \item Also flavor-sensitive
      \item Wedge near $\Delta m = 0$ is covered by monojet
      \end{itemize}
    \end{column}
  \end{columns}
\end{frame}


\section{Tagging Improvements}


\begin{frame}
  \frametitle{The ATLAS Flavor-Tagging Stack}
  \begin{columns}
    \begin{column}{0.6\textwidth}
      \begin{center}
        Flavor-Tagging Framework (Run 1)
        \widegraphic{%
figures/tagging-graphs/tagging-full-nogaia.pdf} \\
        \includegraphics[width=0.8\textwidth]{%
figures/tagging-graphs/tagging-leg.pdf}
      \end{center}
    \end{column}
    \vline
    \begin{column}{0.4\textwidth}
      \begin{center}
        Goal for Run 2 \\[0.1cm]
        \widegraphic{%
figures/tagging-graphs/tagging-gaia-future.pdf}
      \end{center}
    \end{column}
  \end{columns}
  \begin{itemize}
  \item Framework complicated (left) because training NNs is hard
  %% \item We can fix this (right) by training NNs with autoencoders
  \end{itemize}
\end{frame}

\begin{frame}
  \frametitle{Neural Nets Revisited}
  \begin{itemize}
  \item The problem: A \emph{deep} (many layer) neural net converges slowly
  \end{itemize}
  \begin{columns}
    \begin{column}{0.5\textwidth}
      \begin{itemize}
      \item 27 inputs for flavor-tagging
        %% \item Takes many hours to train
      \item \textbf{Many local minima}
        %% \item Performance not reproducible between training
        %% \item This is an applied math problem, but\ldots
      \end{itemize}
    \end{column}
    \begin{column}{0.5\textwidth}
      \widegraphic{figures/tagging-graphs/complicated-train_gen.pdf}
    \end{column}
  \end{columns}
  \begin{itemize}
  \item This has been a known problem for decades
  \item \textbf{This is a solved problem} (In applied math)
  %% \item We needed to talk to experts
  \end{itemize}
\end{frame}

% \[\vect{\xi}_i \equiv (A_i^{\intercal} \vect{\xi}_{(i+1)}) \circ f'(\vect{z}_i)\]

\begin{frame}[t]
  \frametitle{The Solution: Autoencoders}
  \begin{columns}[T]
    \begin{column}{0.5\textwidth}
      \begin{itemize}
      \item Train two layers at a time
      \item Targets \fcolorbox{black}{orange}{$\vect{t} \equiv \vect{z}_{N-2}$}
      \item Each layer $\to$ \emph{feature space}
      \item<2-> Last layer discarded with each step
      %% \item Faster, more robust
      \item<4-> Unsupervised until last step
      \item<5-> Final step: fine-tuning
      \end{itemize}
      \hspace{4in}
    \end{column}
    \begin{column}{0.5\textwidth}
      \only<1>{\widegraphic{figures/tagging-graphs/autoencoder1_gen.pdf}}
      \only<2>{\widegraphic{figures/tagging-graphs/autoencoder1-crop_gen.pdf}}
      \only<3>{\widegraphic{figures/tagging-graphs/autoencoder2_gen.pdf}}
      \only<4>{\widegraphic{figures/tagging-graphs/autoencoder3_gen.pdf}}
      \only<5>{\widegraphic{figures/tagging-graphs/autoencoder3-fine_gen.pdf}}
      \begin{center}
        \only<2>{\textbf{freeze first layer \\ crop outputs}}
        \only<1,3-5>{\fcolorbox{black}{orange}{$\vect{t} \equiv
          \only<1>{\text{inputs}}
          \only<3-4>{\vect{z}_{\only<3>{2} \only<4>{3}}}
          \only<5>{\text{true flavor}}$}}
      \end{center}
    \end{column}
  \end{columns}
  %% \begin{itemize}
  %% \item Trains much more rapidly, more reliably, also \textbf{unsupervised}
  %% \item As a last step ``fine tune'' using full supervised backpropagation
  %% \end{itemize}
\end{frame}

  %%   \begin{equation*}
  %%     \tilde{\vect{x}} = D \underbrace{f(A_1 \vect{x} + \vect{b}_1)}_{\text{encoder}} + \vect{c}
  %%   \end{equation*}
  %% \item The \emph{encoder} compresses $\vect{x}$ into a lower dimensional space, $D$ and $\vect{c}$ decode back to the original input
  %% \item Throw away $D$, $\vect{c}$, repeat with $\vect{x} \to \vect{z_2}$, $i 


\begin{frame}
  \frametitle{Deep NN: Performance}
  \begin{columns}
    \begin{column}{0.5\textwidth}
      \begin{center}
        $b$-tagging
       \widegraphic{%
figures/external/uRejRoc.pdf}
      \end{center}
    \end{column}
    \begin{column}{0.5\textwidth}
      \begin{center}
        $c$-tagging
      \widegraphic{figures/external/ctag-2d-gaia-vs-jfc.pdf}
      \end{center}
    \end{column}
  \end{columns}
  \begin{itemize}
  \item Autoencoders simplify the framework \emph{and} perform better
    \begin{itemize}
    \item For $b$-tagging \textcolor{red}{autoencoders} are always above \textcolor{darkgreen}{MV1}
    \item More efficient $c$-tagger for any choice of $b$ or light rejection
    \end{itemize}
  \item Still an active area of development in ATLAS
  \end{itemize}
\end{frame}


\section{Summary}

\begin{frame}
  \frametitle{Summary}
  \begin{itemize}
  \item First iteration of $c$-taggers has been developed
    \begin{itemize}
    \item Applied to several run 1 SUSY searches
    \end{itemize}
  \item Charm-tagging continues to evolve
    \begin{itemize}
    \item Machine learning, simulation evolve with it
    \item Could improve $b$ tagging as well
    \end{itemize}
  \item Still no SUSY, but new ways to look in run 2
  \item Starting to see interest in $c$-tagging outside SUSY group
    \begin{itemize}
      \item $H \to c\bar{c}$, $H^{+} \to cX$, SM $\gamma + c$
    \end{itemize}
  \item Run 2 will be exciting:
    \begin{itemize}
      \item Higher energy, higher luminosity
      \item New inner pixel layer (Insertable B-Layer),
      %% \item Better neural nets
      %% \item Improved simulation
    \end{itemize}
  \end{itemize}
\end{frame}

%% \begin{frame}
%%   \frametitle{Outlook}
%%   \begin{itemize}
%%   \item $H \to c \bar{c}$ (will be hard but awesome)
%%   \item more general taggers (include $\tau$, $g \to b \bar{b}$)
%%   \item Better $b$-tagging means $H \to HH \to bbXX$ (include this diagram)
%%   \item xAOD migration \emph{could} be promising
%%   \end{itemize}
%% \end{frame}

\section*{Backup}

\backupbegin
\begin{frame}
  \begin{Huge}
    \begin{center}
      BONUS \\ SLIDES
    \end{center}
  \end{Huge}
\end{frame}

\begin{frame}
  \frametitle{Region Summary}
  \resizebox{\textwidth}{!}{

% For inclusion elsewhere: uses the following upstream includes / definitions:
%\RequirePackage{color} % at top of file (\usepackage{color} broke something else)
%\definecolor{orange}{rgb}{1.0, 0.49, 0.0}
%\definecolor{midgreen}{rgb}{0.01, 0.75, 0.24}
%\newcolumntype{C}[1]{>{\centering\let\newline\\\arraybackslash\hspace{0pt}}m{#1}}

\newcommand{\nocut}{---}
\begin{tabular}{|c|l| c | c | c | c | }
\hline
\multirow{2}{*}{Cut} &\multirow{2}{*}{Description}    & Signal regions  &\multicolumn{3}{c|}{Control regions}  \\
\cline{3-6}
 &  & SR & \crz & \crt & \crw \\
\hline
\hline
1 & Trigger & $\met$  & \multicolumn{3}{c|}{single lepton} \\
\hline
2 & Event cleaning &\multicolumn{4}{c|}{ Common to all SR and CR } \\
\hline

\multirow{2}{*}{3} & \multirow{2}{*}{Lepton selection} & \nocut & 2 SF OS & 2 DF OS & 1 \\
\cline{3-6}
                                                    &  & \multicolumn{4}{c|}{No further $e$/$\mu$ after overlap removal with $\pt > 7(6)\,\gev$ for $e$($\mu$)}  \\ \hline

\multirow{2}{*}{4} & $\met$ & $ > 150\,\gev$ & \nocut  & $>50\,\gev$ & $>100\,\gev$ \\ \cline{2-6}
 & $\etmisslep$ & \nocut & $>100\,\gev$ & \multicolumn{2}{c|}{\nocut}   \\ \hline

5 &Leading jet $\pt$ & $> 130\,\gev$ & \multicolumn{2}{c|}{$> 50\,\gev$}  & $> 130\,\gev$  \\ \hline
6 &Second jet $\pt$ & $>100\,\gev$  & \multicolumn{3}{c|}{$> 50\,\gev$}  \\ \hline

7 & $c$-tagging & \multicolumn{4}{c|}{leading 2 jets ($\pt > 50\,\gev$, $|\eta| < 2.5$)}  \\ \hline

8 &$\metdphi{\text{3 jets}}$ & $> 0.4$ & \multicolumn{3}{c|}{\nocut} \\ \hline
9 &$\meteff$ & $> 0.25$ & \multicolumn{3}{c|}{\nocut} \\ \hline

10 & Leading lepton $p_{T}$ & \nocut & $>70\,\gev$ & $ > 25\,\gev$  & $ > 50\,\gev $ \\ \hline
11 & $\mll$ & \nocut & $90 \pm 15\,\gev$ &  $>50\,\gev$ & \nocut \\ \hline
12 & $\mt$ &  \multicolumn{3}{c|}{\nocut} & 40--100$\,\gev$ \\ \hline

13 &$\mcc$ & $>200\,\gev$ & \multicolumn{3}{c|}{\nocut} \\ \hline
14 &$ \mct $ & $>$ 150, 200, 250 \gev & \multicolumn{2}{c|}{\nocut} & $>150\,\gev$ \\ \hline

\end{tabular}
}
\end{frame}

\begin{frame}
  \frametitle{Useful Links (mostly restricted to ATLAS)}
  \begin{itemize}
  \item Scharm analysis twiki: \url{https://twiki.cern.ch/twiki/bin/viewauth/AtlasProtected/DirectScharmAnalysisTwiki}
    \begin{itemize}
    \item INT note: \url{https://cds.cern.ch/record/1698012}
    \item Paper (public): \url{https://atlas.web.cern.ch/Atlas/GROUPS/PHYSICS/PAPERS/SUSY-2014-03/}
    \end{itemize}
  \item JetFitterCharm public: \url{https://atlas.web.cern.ch/Atlas/GROUPS/PHYSICS/PUBNOTES/ATL-PHYS-PUB-2015-001/}
  \item $\sttoc$ paper: \url{http://arxiv.org/abs/1407.0608}
  \end{itemize}
\end{frame}

\begin{frame}
  \frametitle{Background Tradeoffs: $b$ vs $c$ Tagging}
  \begin{columns}
    \begin{column}{0.5\textwidth}
      \begin{center}
        $c$-tagging
      \widegraphic{figures/external/rejrej-cprob.pdf}
      \end{center}
    \end{column}
    \begin{column}{0.5\textwidth}
      \begin{center}
        $b$-tagging
      \widegraphic{figures/external/rejrej-btag.pdf}
      \end{center}
    \end{column}
  \end{columns}
  \begin{itemize}
  \item A 1D discriminant works for $c$-tagging, but we loose the ability to trade backgrounds.
  \item With $b$-tagging this isn't a big deal, one cut fits all
  \end{itemize}
\end{frame}

\begin{frame}
  \frametitle{ATLAS $\tilde{t}$ exclusions}
  \includegraphics[width=0.8\textwidth]{%
    figures/external/blocked-stop-lsp.pdf}
\end{frame}


\begin{frame}
  \frametitle{JetFitterCharm relative performance}
  \begin{itemize}
  \item Several other discriminants have been suggested, compared to JetFitterCharm
  \end{itemize}
  \begin{columns}
    \begin{column}{0.5\textwidth}
      \begin{center}
        vs JetFitterCOMBNN
      \widegraphic{figures/external/ctag-2d-jfc-vs-jfit.pdf}
      \end{center}
    \end{column}
    \begin{column}{0.5\textwidth}
      \begin{center}
        vs MV1 + MV1c
      \widegraphic{figures/external/ctag-2d-jfc-vs-mv.pdf}
      \end{center}
    \end{column}
  \end{columns}
\end{frame}


\begin{frame}
  \frametitle{Comparison Between Deep Learning and MVx}
  \begin{columns}
    \begin{column}{0.5\textwidth}
      \widegraphic{figures/external/ctag-2d-gaia-vs-mv.pdf}
    \end{column}
    \begin{column}{0.5\textwidth}
      \begin{itemize}
      \item Showing Deep NN vs Run 1 ``baseline'' taggers
      \item Never seriously considered
      \end{itemize}
    \end{column}
  \end{columns}
\end{frame}

\begin{frame}
  \begin{columns}
    \begin{column}{0.45\textwidth}
      \widegraphic{int/figures/stackplots/dans/signal_mct150/met.pdf} \\
      \widegraphic{int/figures/stackplots/dans/signal_mct150/j0_pt.pdf}
    \end{column}
    \begin{column}{0.45\textwidth}
      \widegraphic{int/figures/stackplots/dans/signal_mct150/j1_pt.pdf} \\
      \widegraphic{int/figures/stackplots/dans/signal_mct150/j2_pt.pdf}
    \end{column}
  \end{columns}
\end{frame}


\begin{frame}
  \frametitle{Charm tagging calibrations}
  \begin{columns}
    \begin{column}{0.6\textwidth}
      \begin{itemize}
        \item Standard $b$-tagging calibrations:
        \begin{itemize}
        \item[light] \link{Negative tags}{\calnote} (right)\tikz[na] \coordinate(neg-tag);
        \item[$b$] \link{dileptonic $t \bar{t}$}{https://atlas.web.cern.ch/Atlas/GROUPS/PHYSICS/CONFNOTES/ATLAS-CONF-2014-004/} (bottom right)\tikz[na] \coordinate(dilep-tt);
        \item[$c$] \link{$D^*$ reconstruction}{\calnote} (bottom)\tikz[na] \coordinate(dzero);
        \end{itemize}
      \end{itemize}
      \widegraphic{figures/external/dstar-pion-mass.pdf}\\
    \end{column}
    \begin{column}{0.34\textwidth}
      \widegraphic{figures/external/negative-tag-sv0.pdf}\\
      \widegraphic{figures/external/giacinto-mll.pdf}
    \end{column}
  \end{columns}
\end{frame}

\begin{frame}
  \frametitle{Flavor Composition}
  \begin{columns}
    \begin{column}{0.5\textwidth}
      \widegraphic{int/figures/stackplots/will/jetsID_SRA_mCT_gt_150}\\
      \widegraphic{int/figures/stackplots/will/jetsID_CRZ_final.pdf}
    \end{column}
    \begin{column}{0.5\textwidth}
      \widegraphic{int/figures/stackplots/will/jetsID_CRW_final.pdf}\\
      \widegraphic{int/figures/stackplots/will/jetsID_CRT_final.pdf}
    \end{column}
  \end{columns}
\end{frame}


\begin{frame}
  \frametitle{Electron Definition}
  \resizebox{\textwidth}{!}{\begin{tabular}{|l|c|}
\hline
Requirement            & Value \\
\hline
\hline
\multicolumn{2}{|c|}{Preselected Electron}\\
\hline
Author      &  1 or 3 \\
\hline
Acceptance     & $E_\T > 7~\gev, |\eta^\mathrm{clust}| < 2.47$         \\
\hline
Quality & \textsc{Medium++} \\
\hline
Cleaning & $(\mathtt{el\_OQ \& 1446}) = 0$  \\
\hline
Overlap      & $\Delta{}R(e,j)<0.2$ OR $>0.4$, $j \in$ preselected jets \\
\hline
\hline
\multicolumn{2}{|c|}{Signal Electron}\\
\hline
Quality & \textsc{Tight++} \\
%\hline
%Acceptance     & $|\eta^\mathrm{clust}| < 2.47$, ($E_T > 20~\gev$)        \\%the \pt requirement is driven by the trigger thresholds. For leading lepton in any lepton-triggered region it is 25 GeV, for sub-leading leptons use the preselected lepton acceptance cuts.
\hline
%\multicolumn{2}{|c|}{Old IsSignalElecton}  \\
%\hline
%Isolation   & ptCone20/\pt\ $<$ 0.1\\
%\hline
%\hline
%\multicolumn{2}{|c|}{New IsSignalElectronExp} \\
% _etcut(25000.), _id_isocut(0.16), _calo_isocut(0.18), _d0sigcut(5.), _z0cut(0.4), _pt_isoMax(0.) 
Track Isolation   & $\mathtt{ptCone30} < 0.16 \pt$\\
\hline
Calorimeter Isolation & $\mathtt{topoEtcone30\_corrected} - k \nvxp < 0.18 \pt$\\
\hline
Longitudinal IP & $z_0 \sin(\theta) < 0.4\,\text{mm}$\\
\hline
Transverse IP signifiance & $d_0/\sigma_{d_0} < 5$\\
\hline
\end{tabular}
}
\end{frame}

\begin{frame}
  \frametitle{Muon Definition}
  \begin{center}
  \resizebox{0.8\textwidth}{!}{\begin{tabular}{|l|c|}
\hline
Requirement            & Value \\
\hline
\hline
\multicolumn{2}{|c|}{Preselected muon}\\
\hline
Algorithm      & \textsc{staco}, combined or segment-tagged muon \\
\hline
Acceptance     & $\pt > 6~\gev, |\eta| < 2.4$          \\
\hline
Quality        & \textsc{Loose}    \\
\hline
\multirow{6}{*}{ID track quality} & $\geq 1$ b-layer hit when it can be expected \\
                 & $\geq 1$ pixel hit or crossed dead pixel sensor \\
                 & $\geq 5$ SCT hits or crossed dead SCT sensor\\
                 & pixel holes + SCT holes $< 3$\\
                 & If $0.1 < |\eta| < 1.9$: $n_{\mathrm{TRT}} \geq 6$ or $n_{\mathrm{TRT}}^{\mathrm{outliers}} < 0.9 n_{\mathrm{TRT}}$ \\
               & If $|\eta| \geq 1.9$ and $n_{\mathrm{TRT}} \geq 6$: $n_{\mathrm{TRT}}^{\mathrm{outliers}} < 0.9 n_{\mathrm{TRT}}$ \\
\hline
Overlap     & $\Delta{}R(\mu,j) > 0.4$, $j \in $ preselected jets\\
\hline
\hline
\multicolumn{2}{|c|}{Signal muon}\\
\hline
%Acceptance     & $|\eta| < 2.4$ ($\pt > 20~\gev$)          \\%the \pt requirement is driven by the trigger thresholds. For leading lepton in any lepton-triggered region it is 25 GeV, for sub-leading leptons use the preselected lepton acceptance cuts.
%\hline
%% Quality        & Loose    \\
%% \hline
Cosmics Veto   & %% $|z_{\mu} - z_{\mathrm{PV}}| < 1\,\mathrm{mm}$
$d_0 < 0.2\,\text{mm}$          \\
\hline
%\hline
%\multicolumn{2}{|c|}{Old IsSignalMuon} \\
%\hline
%Isolation & ptCone20 $<$ 1.8 GeV \\
%\hline
%\hline
%\multicolumn{2}{|c|}{New IsSignalMuonExp} \\
%_ptcut(25000.), _id_isocut(0.12), _calo_isocut(0.12), _d0sigcut(3.), _z0cut(0.4), _pt_isoMax(0.) {}
%\hline
Track Isolation & $\mathtt{ptcone30\_trkelstyle} < 0.12 \pt$ \\
\hline
Calorimeter Isolation & $\mathtt{etcone30} - k_1 \nvxp - k_2 (\nvxp)^2 < 0.12 \pt$ \\
\hline
Longitudinal IP & $z_0 < 0.4\,\text{mm}$ \\
\hline
Transverse IP signifiance & $d_0/\sigma_{d_0} < 3$\\
\hline
\end{tabular}
}
  \end{center}
\end{frame}

\begin{frame}
  \frametitle{Trigger Definition}
  \begin{center}
  \resizebox{0.8\textwidth}{!}{\begin{tabular}{ | l | l | l | l | }
\hline
Channel & Trigger chain & Offline Threshold used \\ \hline
$\met$ & \texttt{EF\_xe80\_tclcw\_tight} & $\met > 150\,\gev$ \\
1-electron & \texttt{EF\_e24vhi\_medium1} OR \texttt{EF\_e60\_medium1} & $\pt(e)>25\,\gev$ \\
1-muon  & \texttt{EF\_mu24i\_tight} OR \texttt{EF\_mu36\_tight} & $\pt(\mu)>25\,\gev$ \\
\hline
%% 2-electron & \texttt{EF\_e24vhi\_medium1} OR \texttt{EF\_e60\_medium1} & $\pt(e)>(25,7)\,\gev$ \\
%% \hline
%% 2-muon & \texttt{EF\_mu24i\_tight} OR \texttt{EF\_mu36\_tight} & $\pt(\mu)>(25,6)\,\gev$ \\
%% \hline
%% 1-e, 1-mu & same as single $e/\mu$ triggers (overlap removed) & $\pt(e/\mu)>(25,7/6)\,\gev$ \\
%% \hline
\end{tabular}
}
  \end{center}
\end{frame}

\begin{frame}
  \frametitle{Outlook}
  \begin{block}{$\tilde{c}$--$\tilde{t}$ mixing}
    \begin{itemize}
    \item \link{arXiv 1302.7232}{http://arxiv.org/abs/1302.7232}
    \item Helps fine-tuning (a little)
    \item Predicts decays like $\tilde{t} + \tilde{c} \to (t \bar{c} + c \bar{t}) \neut$
    \item Not investigated (so far)
    \end{itemize}
  \end{block}
  \begin{block}{Higgs Couplings}
  \end{block}
\end{frame}



\backupend

\end{document}
\begin{frame}
  \frametitle{Charm Tagging}
  \begin{block}{Soft Lepton}
  \begin{itemize}
  \item Branching ratio to muons $<$ 7\% (18\%) for $D^0$ ($D^\pm$)
    \begin{itemize}
    \item Unacceptably low efficiency for most analyses
    \end{itemize}
  \end{itemize}
  \end{block}
  \begin{columns}
    \begin{column}{0.5\textwidth}
      \begin{block}{Lifetime Based}
      \begin{itemize}
      \item Shorter decay lengths vs $b$-jets
        \begin{center}
          \begin{tabular}{c | c c c}
            Meson  &  $B$ & $D^0$ & $D^{\pm}$ \\ \hline
            $c\tau$ [$\mu$m] & 492 & 312 & 123 \\
          \end{tabular}
        \end{center}
      \item Fewer tracks than $b$-jets
      \item $c$-jets are ``between'' light and $b$ in many distributions
      \item \textbf{NOTE:} Vertices inside inner pixels, tracks extrapolated
      \end{itemize}
      \end{block}
    \end{column}
    \begin{column}{0.5\textwidth}
      \widegraphic{figures/external/b-jet.pdf}
    \end{column}
  \end{columns}
\end{frame}

\begin{frame}
  \frametitle{Motivation: $\sttoc$ and $\sctoc$}
  \begin{columns}
    \begin{column}{0.6\textwidth}
      \begin{itemize}
      \item Assume $m_{\tilde{t}} > m_{\neut}$
      \item All other sparticles heavy
      \item ``compressed'': $m_{\tilde{t}} - m_{\neut} < m_W$
      \item Forces $\tilde{t}$ decay via $c$-jets
      \item Signature: boosted, $c$-jets + $\met$
      \item See \link{arXiv 1407.0608}{http://arxiv.org/abs/1407.0608}
      %% \item \textcolor{red}{Not discussed in detail here (ADD LINK)}
      \end{itemize}
    \end{column}
    \begin{column}{0.4\textwidth}
      \widegraphic{figures/external/feyn-sttoc-isr.pdf}
    \end{column}
  \end{columns}
  \hrule
  \begin{columns}
    \begin{column}{0.4\textwidth}
      \widegraphic{figures/external/feyn-sctoc.pdf}
    \end{column}
    \begin{column}{0.6\textwidth}
      \begin{itemize}
      \item Same general signature as $\sttoc$, $m_{\scharm} - m_{\neut} \nless m_W$
      \item ``Light'' ($udcs$) $\tilde{q}$ masses needn't be degenerate (see \link{arXiv 1212.3328}{http://arxiv.org/abs/1212.3328})
      \item \textbf{Focus of this talk} (also \link{arXiv 1501.01325}{http://arxiv.org/abs/1501.01325})
      \end{itemize}
    \end{column}
  \end{columns}
\end{frame}
\end{comment}
