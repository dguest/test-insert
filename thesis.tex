\documentclass{article}
\usepackage{xspace}
\usepackage{amsmath}

\usepackage{geometry}
\geometry{
  letterpaper,%
  textwidth=16cm,%
  textheight=23.2cm,%
  marginparsep=7pt,%
  marginparwidth=2.5cm%
}

\usepackage{hyperref}
\hypersetup{
  pdfauthor={Daniel Guest},
  pdftitle={Search for Scalar Charm Production with the ATLAS Detector},
  pdfsubject={Measure Zero (to really good precision)},
  colorlinks,
  urlcolor=blue,
  linkcolor=blue,
  citecolor=red
}


\title{Serches for Scalar Heavy Quarks Decaying to Charmed Final States in the ATLAS Detector}
\author{Daniel Guest}
\begin{document}


\newcommand{\cmenergy}{\ensuremath{\sqrt{s} = 8\,\text{TeV}}\xspace}

\maketitle

\begin{abstract}
I present the results from two ATLAS searches for pair-produced supersymmetric quarks in all avalible \cmenergy $pp$ collision data as of 2013. The first search targets pair production of scalar top quarks decaying to neutrilinos and charm quarks in a senario where other decays are kinematically suppressed, while the second search focuses on scalar charm quarks decaying to the same final state. Both searches make heavy use of a novel lifetime-based ``charm tagging'' algorithm and only consider a subset of events with large missing transverse energy and no leptons. In both cases, the number of observed events is consistent with the standard model: no evidence for supersymmetry is found over the range in which the searches have sensitivity.
\end{abstract}



\end{document}
