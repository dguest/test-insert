Uncertainties in this search arise from several sources.
Some are fairly obvious: the number of events populating the signal region is inherently Poissonian, as is the number of events in the control regions.
The far more difficult systematic uncertainties are described here.
As a convention, systematic uncertainties are applied to simulation and not to data.

These are roughly divided into \emph{experimental} (associated with the detector response) and \emph{theoretical} (associated with parameters of event generation and hadronization).
It's important to note that this division is only approximate.
For example, the flavor-tagging systematics are are entered as a set of several experimental uncertainties, but represent a reduced set of uncertainties corresponding to several analyses, each of which inherits uncertainty from limited statistics, uncertain detector response, and inexact modeling.
The same is true of other systematic uncertainties.

\subsection{Experimental}
\label{sec:sys_experimental}
Experimental uncertainties are applied to every simulated sample.
In general these systematics will be correlated between the signal and control regions, and may thus cancel to some extent in the final fit.
These systematics are implemented in two separate ways: normalization-based systematics are applied eventwise by multiplying the event normalization by an additional systematic factor; the remaining systematics involve recalculating object or event parameters with a perturbation applied, which will propagate to the event yield.

\subsubsection{Jet Related Uncertainties}
\paragraph{Jet Energy Scale} There is considerable uncertainty in the measurement and calibration of jet energy.
For the purposes of this analysis this uncertainty is condensed into a single pair of variations; all jet energies are scaled up or down according to the recommendations of the Jet/$\met$ group~\cite{JES,alt-jes,jes-twiki}.
\paragraph{Jet Energy Resolution} Jets are smeared in $\pt$ as a function of their $\pt$ and $\eta$, to account for overly optimistic jet resolution in simulation~\cite{jer}.
The up and down yield variations are defined to be $n_{\text{nom}} \pm (n_{\text{nom}} - n_{\text{smear}})/2$ where $n_{\text{nom}}$ and $n_{\text{smear}}$ are the nominal yield and the yield with jet smearing, respectively.
\paragraph{Jet Vertex Fraction} To reduce backgrounds from pileup, signal jets are required $\jvf > 0.5$ as defined in \cref{sec:jets}. An upward and downward yield uncertainty is derived by requiring $\jvf > 0.53$ and $\jvf > 0.47$, respectively.
\paragraph{Flavor Tagging} As discussed in \cref{sec:calib}, flavor-tagging uncertainties are derived from several designated analyses, separately for $b$, $c$, light, and $\tau$ jets. Each systematic is implemented as a normalization scale-factor.

\subsubsection{Lepton Uncertainties}
While not directly required in the search, leptons play an important role in the control regions.
Various systematics associated with lepton identification and calibration propagate into the background prediction in the signal region.
The systematic variations listed below are calculated from measurements of $Z \to \ell \ell$ and $J/\psi \to \ell \ell$ decays in data and simulation~\cite{el-gamma,muon}.
\paragraph{Lepton Identification} A normalization scale-factor is applied to leptons to account for discrepancies in the identification efficiency between simulation and data.
\paragraph{Lepton Energy and Momentum Resolution} Lepton energies are known within a few percent. To account for the uncertainty, the momentum of leptons is scaled in several ways:
\begin{itemize}
\item The low $\pt$ electron spectrum is shifted up or down to account for mismodeling.
\item The spectrum of higher $\pt$ electrons is shifted up or down to account for mismodeling measured via the $Z \to ee$ calibration.
\item The muon energy is shifted up or down to account for any muon energy mismodeling.
\end{itemize}

\subsubsection{Other Eventwise Uncertainties}
\paragraph{$\mathbb{\met}$ Resolution} Since $\met$ is computed partially using calibrated jets, leptons, and photons, the majority of $\met$ uncertainty is accounted for through the object uncertainties listed above.
As discussed in \cref{sec:met}, the final component of $\met$ is the $\subvmet{CellOut}$ term associated to the remaining calorimeter clusters.
This component is smeared to produce a corresponding event yield variation~\cite{met-tool}.
The up and down yield variations are defined to be $n_{\text{nom}} \pm (n_{\text{nom}} - n_{\text{smear}})/2$ where $n_{\text{nom}}$ and $n_{\text{smear}}$ are the nominal yield and the yield with $\met$ smearing, respectively.

\paragraph{Pileup} To account for differences between simulation and data in the number of primary vertices per bunch crossing $n_{\textrm{vxp}}$, simulated events are reweighted such the $n_{\textrm{vxp}}$ distributions match.
This reweighting also scales up the overall $n_{\textrm{vxp}}$ in simulation by $1.09 \pm 0.04$. The yields on either end of this range form another systematic variation.

\paragraph{Luminosity} The integrated luminosity using the $\met$ trigger is $\lumiunct$ in 2012, corresponding to a 2.8\% uncertainty, as measured through several techniques~\cite{atlas-lumi}.
This uncertainty is propagated to the fit, but since the major backgrounds are normalized by control regions it technically only influences the normalization of the signal and minor backgrounds.

\paragraph{Trigger} Uncertainties in the lepton triggers are negligible given the offline lepton $\pt$ requirement. To account for uncertainty in the $\met$ trigger, an overall $2\%$ uncertainty is applied in all $\met$ triggered regions.

\subsubsection{Minor Backgrounds}

\paragraph{Electroweak Processes} A flat 50\% uncertainty is applied to the minor single $t$, $\ttbar V$, and $VV$ backgrounds. The actual uncertainty is expected to be much smaller, but this inflated uncertainty has a negligible effect on the analysis given the minuscule number of events from these processes in the signal region.

\paragraph{QCD Multijet} A conservative 100\% uncertainty is applied to the QCD background, as discussed in \cref{sec:qcd-background}.

\subsection{Theoretical}
\subsubsection{Signal}
\subsubsection{Background}
\label{sec:sys_theorybkg}
