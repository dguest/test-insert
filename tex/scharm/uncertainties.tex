Uncertainties in this search arise from several sources.
Some are fairly obvious: the number of events populating the signal region is inherently Poissonian, as is the number of events in the control regions.
Far more difficult are the systematic uncertainties, which are described here.
As a convention, systematic uncertainties are applied to simulation and not to data.

These are roughly divided into \emph{experimental} (associated with the detector response) and \emph{theoretical} (associated with parameters of event generation and hadronization).
It's important to note that this division is only approximate.
For example, the flavor-tagging systematics are entered as a set of several experimental uncertainties, but represent a reduced set of uncertainties corresponding to several analyses, each of which inherits uncertainty from limited statistics, uncertain detector response, and inexact modeling.
The same is true of other systematic uncertainties.

\subsection{Experimental}
\label{sec:sys_experimental}
Experimental uncertainties are applied to every simulated sample.
In general these systematics will be correlated between the signal and control regions, and may thus cancel to some extent in the final fit.
These systematics are implemented in two separate ways. Normalization-based systematics are applied eventwise by multiplying the event normalization by an additional systematic factor. The remaining systematics involve recalculating object or event parameters with a perturbation applied, which will propagate to the event yield.

\subsubsection{Jet Related Uncertainties}
\paragraph{Jet Energy Scale} There is considerable uncertainty in the measurement and calibration of jet energy.
For the purposes of this analysis this uncertainty is condensed into a single pair of variations; all jet energies are scaled up or down according to the recommendations of the Jet/$\met$ group~\cite{JES,alt-jes,jes-twiki}.
\paragraph{Jet Energy Resolution} Jets are smeared in $\pt$ as a function of their $\pt$ and $\eta$, to account for overly optimistic jet resolution in simulation~\cite{jer}.
The up and down yield variations are defined to be $n_{\text{nom}} \pm (n_{\text{nom}} - n_{\text{smear}})/2$ where $n_{\text{nom}}$ and $n_{\text{smear}}$ are the nominal yield and the yield with jet smearing, respectively.
\paragraph{Jet Vertex Fraction} To reduce backgrounds from pileup, signal jets are required $\jvf > 0.5$ as defined in \cref{sec:jets}. An upward and downward yield uncertainty is derived by requiring $\jvf > 0.53$ and $\jvf > 0.47$, respectively.
\paragraph{Flavor Tagging} As discussed in \cref{sec:calib}, flavor-tagging uncertainties are derived from several designated analyses, separately for $b$, $c$, light, and $\tau$ jets. Each systematic is implemented as a normalization scale-factor.

\subsubsection{Lepton Uncertainties}
While not directly required in the search, leptons play an important role in the control regions.
Various systematics associated with lepton identification and calibration propagate into the background prediction in the signal region.
The systematic variations listed below are calculated from measurements of $Z \to \ell \ell$ and $J/\psi \to \ell \ell$ decays in data and simulation~\cite{el-gamma,muon}.
\paragraph{Lepton Identification} A normalization scale-factor is applied to leptons to account for discrepancies in the identification efficiency between simulation and data.
\paragraph{Lepton Energy and Momentum Resolution} Lepton energies are known within a few percent. To account for the uncertainty, the momentum of leptons is scaled in several ways:
\begin{itemize}
\item The low $\pt$ electron spectrum is shifted up or down to account for mismodeling.
\item The spectrum of higher $\pt$ electrons is shifted up or down to account for mismodeling measured via the $Z \to ee$ calibration.
\item The muon energy is shifted up or down to account for any muon energy mismodeling.
\end{itemize}

\subsubsection{Other Eventwise Uncertainties}
\paragraph{$\bm{\met}$ Resolution} Since $\met$ is computed partially using calibrated jets, leptons, and photons, the majority of $\met$ uncertainty is accounted for through the object uncertainties listed above.
As discussed in \cref{sec:met}, the final component of $\met$ is the $\subvmet{CellOut}$ term associated to the remaining calorimeter clusters.
This component is smeared to produce a corresponding event yield variation~\cite{met-tool}.
The up and down yield variations are defined to be $n_{\text{nom}} \pm (n_{\text{nom}} - n_{\text{smear}})/2$ where $n_{\text{nom}}$ and $n_{\text{smear}}$ are the nominal yield and the yield with $\met$ smearing, respectively.

\paragraph{Pileup} To account for differences between simulation and data in the number of primary vertices per bunch crossing $n_{\textrm{vxp}}$, simulated events are reweighted such the $n_{\textrm{vxp}}$ distributions match.
This reweighting also scales up the overall $n_{\textrm{vxp}}$ in simulation by $1.09 \pm 0.04$. The yields on either end of this range form another systematic variation.

\paragraph{Luminosity} The integrated luminosity using the $\met$ trigger is $\lumiunct$ in 2012, corresponding to a 2.8\% uncertainty, as measured through several techniques~\cite{atlas-lumi}.
This uncertainty is propagated to the fit, but since the major backgrounds are normalized by control regions it technically only influences the normalization of the signal and minor backgrounds.

\paragraph{Trigger} Uncertainties in the lepton triggers are negligible given the offline lepton $\pt$ requirement. To account for uncertainty in the $\met$ trigger, an overall $\pm 2\%$ uncertainty is applied in all $\met$ triggered regions.

\subsubsection{Minor Backgrounds}

\paragraph{Electroweak Processes} A flat 50\% uncertainty is applied to the minor single $t$, $\ttbar V$, and $VV$ backgrounds. The actual uncertainty is expected to be much smaller, but this inflated uncertainty has a negligible effect on the analysis given the minuscule number of events from these processes in the signal region.

\paragraph{QCD Multijet} A conservative 100\% uncertainty is applied to the QCD background, as discussed in \cref{sec:qcd-background}.

\subsection{Theoretical}
\label{sec:sys-theory}
Theoretical uncertainties in generator tunes, cross-sections, and proton PDFs propagate to expected yields.
These uncertainties are generally independent of detector modeling, so in the interest of computational efficiency they are estimated without full event reconstruction.
Instead, the events are simulated up to the formation of stable particles, at which point a proxy signal selection is applied.
Flavor-tagging efficiency is estimated through \emph{tagging rate functions} which are provided by the flavor-tagging group as a function of $\pt$ and $\eta$.

\subsubsection{Signal}
\paragraph{Cross Section} Squark cross-sections are calculated with next-to-leading-order (NLO) strong couplings, with next-to-leading-logarithmic treatment of soft gluons.
The uncertainties are estimated an envelope of PDF sets, and factorization and renormalization scales~\cite{susy-xsec}.
Scalar charm pair production is dominantly through the $s$-channel, and as a result the cross-section is nearly identical to the $\tilde{b}\tilde{b}^*$ and $\tilde{t}\tilde{t}^*$ cross-sections.
Cross-sections and uncertainties are thus taken directly from the $\sbtob$ search~\cite{sbottom}.

\paragraph{Final and Initial State Radiation} Initial and final state radiation is expected to play a relatively minor role in this analysis, since the leading jets are expected to originate from the $\sctoc$ decay.
An overall uncertainty is nonetheless estimated from samples from $\sbtob$ samples with several variations applied.
The factorization and renormalization, parton-jet matching, and  $\alpha_{s}$ scales are varied over a range of 50--200\% of the nominal value. In addition, the \textsc{pythia} final state radiation tune is varied.
A combination of these uncertainties results in an expected yield variation of $\pm 10.7\%$.

\newcommand{\nsig}{n_{\text{sig}}}
\newcommand{\nctl}{n_{\text{ctl}}}
\newcommand{\effsig}{\epsilon_{\text{sig}}}
\newcommand{\effctl}{\epsilon_{\text{ctl}}}
\newcommand{\ndata}{n^{\text{data}}_{\text{ctl}}}
\newcommand{\nother}{n^{\text{other}}_{\text{ctl}}}
\subsubsection{Background}
\label{sec:sys_theorybkg}
All theoretical background uncertainties are propagated to the signal region via \emph{transfer factors}, which relate the number of expected events for a given process in the signal region $\nsig$ to the number in the control region $\nctl$.
With transfer factors, the number of predicted events in the signal region is given by
\begin{equation}
  \nsig = \frac{\effsig}{\effctl} \underbrace{(\ndata - \nother)}_{\nctl}.
  \label{eq:transfer-factor}
\end{equation}
Here $\effsig$ and $\effctl$ are the efficiencies for the process in the signal and control region selection, respectively.
The number of control region events is calculated from the number of data events $\ndata$ and the number of events in other simulated processes $\nother$.
Within \cref{eq:transfer-factor}, the transfer factor is $\effsig / \effctl$.

Theoretical uncertainties are propagated by assuming they apply independently for each sample, and thus that only $\effsig$ and $\effctl$ are affected by the uncertainty.
Upper and lower bounds on $\nsig$ are thus calculated by
\begin{equation}
  \nsig^\pm = \frac{\effsig^\pm}{\effctl^\pm} (\ndata - \nother). % \quad \Rightarrow \quad \frac{\nsig^\pm}{\nctl} = \frac{\effsig^\pm}{\effctl^\pm}.
\end{equation}
The various uncertainties are combined for each process into an overall top, $\wjets$, and $\zjets$ uncertainty.

\paragraph{Top} Both $\ttbar$ and single top $Wt$-channel samples are generated independently with \textsc{powheg} and \textsc{mc@nlo}, then passed to \textsc{herwig} for parton showering and hadronization, since the \textsc{mc@nlo} and \textsc{pythia} versions used are incompatible.
The difference in yields between the two samples is taken as a systematic uncertainty.
To account for differences in parton showering between generators, a sample generated with \textsc{powheg} is showered with both \textsc{pythia} and \textsc{herwig}. The difference forms another uncertainty.

An uncertainty for the single top $s$-channel is calculated by combining the showing and generator systematics together.
A sample generated with \textsc{powheg} and showed with \textsc{pythia} is compared with a sample generated with \textsc{mc@nlo} and showered with \textsc{herwig}.
Renormalization, factorization, ISR, and FSR uncertainties are calculated with a method identical to that used in the signal. In the case of ISR/FSR variations, \textsc{acerMC} is used.

The interference between the single top $Wt$-channel and $\ttbar$ production is non-negligible.
Two schemes to estimate this effect are diagram removal~\cite{single-t-dr} and diagram subtraction~\cite{single-t-ds}.
Diagram removal is used by default.
A systematic error is estimated from the difference in yields between the two methods.

\paragraph{$\wjets$} As with $\ttbar$ the factorization and renormalization scales for $\wjets$ simulation are varied within an envelope.
The baseline $\wjets$ sample only simulates up to 4 additional partons in the event.
To account for this limitation, a sample with exactly 4 partons is compared to a sample with 4 or 5 partons. The relative difference in yields is taken as a relative systematic.

\paragraph{$\zjets$} The factorization and renormalization scale uncertainties, along with uncertainties associated with matrix-element--parton-shower matching are calculated by reweighting samples based on the number of jets with $\pt > 30\,\gev$~\cite{zero-l-int}.
As with $\wjets$, the generator is only able to cope with a limited number of partons.
A systematic to quantify this limitation is again derived from the yield difference between the 4-parton sample and the 4-or-5--parton sample.

\newcommand{\xct}{X_{\textsc{ct10}}}
\newcommand{\xmstw}{X_{\textsc{mstw}}}
\paragraph{PDF Uncertainties} Electroweak samples use the \textsc{ct10} PDF sets. For these samples an intra-PDF uncertainty is computed according to
\begin{equation}
  \Delta \xct = \frac{1}{1.64} \times \frac{1}{2} \sqrt{\sum_i (X^{+}_{i} - X^{-}_{i})^2 }
  \label{eq:intra-pdf}
\end{equation}
where $X_i^{+}$ and $X_i^{-}$ are yields from the $i$th variation in the \textsc{ct10} PDF set. The factor of 1.64 is needed because these variations correspond to a 95\% confidence limit, whereas the desired variation is $1\sigma$.
This uncertainty is combined with the inter-PDF uncertainty by comparing to the \textsc{mstw} pdf set:
\begin{equation}
  \sigma_{\mathrm{PDF}} = \frac{1}{2} \frac{ \max (\xct + \Delta \xct, \xmstw) - \min( \xct - \Delta \xct, \xmstw ) }{\xct}.
\end{equation}
The corresponding uncertainties in $\ttbar$ are computed with the \textsc{mc@nlo} generator, since \textsc{powheg} isn't expected to give accurate results.
