Following the selection discussed in the previous section, simulation still predicts considerable background contamination in the signal region.
The expected backgrounds are roughly 50\% $\zjets$, with the remaining split evenly between $\ttbar$ and $\wjets$.
Unfortunately, the overall magnitudes of these processes aren't particularly constrained in simulation; they are dependent on the proton PDFs and assumptions about jet production, which are still relatively unknown at LHC energy scales.
Instead of estimating these major backgrounds (and extrapolating uncertainties) from previously published results, the backgrounds were estimated from a series of control regions.

In all cases, the $c$-tagging requirements in the control regions is identical to that in the signal regions. Wherever possible, other kinematic requirements are also similar to the signal region requirements. In many cases, however, these requirements must be relaxed to yield sufficient statistics to control the backgrounds.

\subsection{$\zjets$ Background}
The $\zjets$ process accounts for the largest predicted background in the signal regions.
Roughly 70\% of $Z$ decays are hadronic~\cite{pdg2014}, but these decays are unlikely fake signal events given their low $\met$.
The remaining decays are split between leptons (10\%) and neutrinos (20\%).
The leptonic decays are also unlikely to fake a signal event, since the two leptons must evade both the signal region lepton veto and the $\met$ calculation.
The dominant events which contribute to the signal region are thus $\zjets$ events where the $Z$ decays to neutrinos.

Fortunately, designing a control region for $Z \to \nu\nu$ is relatively easy.
Branching ratios for $Z$ bosons are known to exceptional accuracy, so the full suite of $Z$ decays can be extrapolated from any $Z \to \ell \bar{\ell}$ cross-section measurement.
Ideally this cross-section is measured in a region demanding an underlying event similar to that in the signal region---the precision of $Z$ branching ratios can't mitigate large uncertainties in the associated jet production.

With this in mind, we construct a control region populated by events where $Z$ bosons are reconstructed from leptons.
Such events won't fire the $\met$ trigger and thus must be drawn from the lepton trigger streams.
For consistency with the $Z$-decay hypothesis, the events must contain two leptons of the same flavor, with opposite sign, and with an invariant mass of $90 \pm 15\,\gev$.
The leptons are added to the event $\met$ to mimic the contribution from neutrinos.


\subsection{Top Background}
A considerable number of $\ttbar$ events are expected in the signal region.
The normalization of these events is controlled via a dileptonic $\ttbar$ selection; exactly one $e$ and one $\mu$ are required.
The resulting sample is extremely pure in $\ttbar$.


\subsection{QCD Background}
this will be short, paraphrasing the QCD section in the INT
\subsection{$\wjets$ Background}

\begin{table}
  \begin{center}
  

% For inclusion elsewhere: uses the following upstream includes / definitions:
%\RequirePackage{color} % at top of file (\usepackage{color} broke something else)
%\definecolor{orange}{rgb}{1.0, 0.49, 0.0}
%\definecolor{midgreen}{rgb}{0.01, 0.75, 0.24}
%\newcolumntype{C}[1]{>{\centering\let\newline\\\arraybackslash\hspace{0pt}}m{#1}}

\newcommand{\nocut}{---}
\begin{tabular}{|c|l| c | c | c | c | }
\hline
\multirow{2}{*}{Cut} &\multirow{2}{*}{Description}    & Signal regions  &\multicolumn{3}{c|}{Control regions}  \\
\cline{3-6}
 &  & SR & \crz & \crt & \crw \\
\hline
\hline
1 & Trigger & $\met$  & \multicolumn{3}{c|}{single lepton} \\
\hline
2 & Event cleaning &\multicolumn{4}{c|}{ Common to all SR and CR } \\
\hline

\multirow{2}{*}{3} & \multirow{2}{*}{Lepton selection} & \nocut & 2 SF OS & 2 DF OS & 1 \\
\cline{3-6}
                                                    &  & \multicolumn{4}{c|}{No further $e$/$\mu$ after overlap removal with $\pt > 7(6)\,\gev$ for $e$($\mu$)}  \\ \hline

\multirow{2}{*}{4} & $\met$ & $ > 150\,\gev$ & \nocut  & $>50\,\gev$ & $>100\,\gev$ \\ \cline{2-6}
 & $\etmisslep$ & \nocut & $>100\,\gev$ & \multicolumn{2}{c|}{\nocut}   \\ \hline

5 &Leading jet $\pt$ & $> 130\,\gev$ & \multicolumn{2}{c|}{$> 50\,\gev$}  & $> 130\,\gev$  \\ \hline
6 &Second jet $\pt$ & $>100\,\gev$  & \multicolumn{3}{c|}{$> 50\,\gev$}  \\ \hline

7 & $c$-tagging & \multicolumn{4}{c|}{leading 2 jets ($\pt > 50\,\gev$, $|\eta| < 2.5$)}  \\ \hline

8 &$\metdphi{\text{3 jets}}$ & $> 0.4$ & \multicolumn{3}{c|}{\nocut} \\ \hline
9 &$\meteff$ & $> 0.25$ & \multicolumn{3}{c|}{\nocut} \\ \hline

10 & Leading lepton $p_{T}$ & \nocut & $>70\,\gev$ & $ > 25\,\gev$  & $ > 50\,\gev $ \\ \hline
11 & $\mll$ & \nocut & $90 \pm 15\,\gev$ &  $>50\,\gev$ & \nocut \\ \hline
12 & $\mt$ &  \multicolumn{3}{c|}{\nocut} & 40--100$\,\gev$ \\ \hline

13 &$\mcc$ & $>200\,\gev$ & \multicolumn{3}{c|}{\nocut} \\ \hline
14 &$ \mct $ & $>$ 150, 200, 250 \gev & \multicolumn{2}{c|}{\nocut} & $>150\,\gev$ \\ \hline

\end{tabular}

  \caption[Summary of the signal regions]{Summary of the regions in the $\sctoc$ search.}
  \end{center}
\end{table}
