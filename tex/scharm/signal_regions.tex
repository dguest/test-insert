As a 0-lepton search in a hadron collider, the $\sttoc$ search is forced to contend with a large multijet QCD background.
This background can be suppressed with an absolute $\met$ threshold, since most transverse momentum from QCD events should be captured by the ATLAS calorimeter and very little real $\met$ should exist.
To this end the search uses a $\met$ trigger as discussed in \cref{sec:met-trigger}.
After this requirement a small fraction of the original QCD events still remain; if a single energetic jet is badly mismeasured, the overall $\met$ can be pushed above threshold.

Fortunately this mismeasurement is statistically unlikely to effect more than a single jet in the event, and as a result the event $\vmet$ aligns with the jet momentum.
Distributions of $\metdphi{j}$ and $\met$ after the $\met$ trigger are given in \cref{fig:jm-dphi}.
The expected standard model backgrounds are also overlaid. Clearly the vast majority of events are accounted for with $\wjets$, $\zjets$, and $\ttbar$ after the $\met > 150\,\gev$ and $\metdphi{j} > 0.4$ requirements.
As shown in \cref{fig:meteff}, a very small excess of QCD events is still visible at low values of $\meteffshort \equiv \meteff$. In the signal region these events are removed with a $\meteffshort > 0.25$ requirement.

\begin{figure}
  \begin{center}
  \includegraphics[width=0.49\textwidth]{%
    int/figures/stackplots/dans/preselection/jetmet_dphi.pdf}
  \includegraphics[width=0.49\textwidth]{%
    int/figures/stackplots/dans/preselection/met.pdf}
  \caption[Distributions showing only the anti-QCD requirements in the signal region]{Distribution of data and simulated SM backgrounds after the $\met > 150\,\gev$ and $\Delta \phi(j,\vmet)$ requirements. Left: $\metdphi{j} | j \in \{\text{three leading signal jets}\}$. Right: $\met$.
For comparison the expected signal events for two choices of $\scharm$ and $\neut$ masses are also shown---they are overwhelmed by the SM backgrounds.
}
  \label{fig:jm-dphi}
  \end{center}
\end{figure}

Following the anti-QCD cuts, the signal remains small relative to the SM backgrounds. As shown in figure \cref{fig:flavor-comp-presel}, the leading two jets in these two backgrounds are dominantly light flavored.
To reduce these backgrounds the two leading jets are $c$-tagged.
Each tag rejects over $99\%$ of light jets and over $80\%$ of $b$~jets, which reduces electroweak backgrounds by 4 orders of magnitude.
At the same time, applying two $c$~tags reduces the signal by a factor of 20.

\begin{figure}
  \begin{center}
  \includegraphics[width=0.49\textwidth]{%
    int/figures/stackplots/dans/preselection/j0_flavor_truth_label.pdf}
  \includegraphics[width=0.49\textwidth]{%
    int/figures/stackplots/dans/preselection/j1_flavor_truth_label.pdf}
  \caption[Flavor truth label of leading two jets after QCD cuts]{Flavor truth label of leading two jets after QCD cuts have been applied. The truth matching scheme applied is explained in \cref{sec:flavor-truth-matching}.}
  \label{fig:flavor-comp-presel}
  \end{center}
\end{figure}

After the $c$~tagging requirement, the number of $\wjets$, $\zjets$, and $\ttbar$ events are similar within an order of magnitude, although the $\ttbar$ fraction is increased.
The $\ttbar$ background can be further reduced thanks to several variables.
Among the most important is $\mct$, which was introduced in \cref{eq:mctdef}.
The desirable properties of $\mct$ only hold when the two leading particles are identical and decay semi-invisibly.
In the case of $\ttbar$, this means the two leading jets must contain $t \to b \nu$ decays.
In some cases a non-$b$~jet has higher $\pt$ than one of the $b$~jets, which gives rise to two populations of $\ttbar$: $b$-lead $\ttbar$, where the leading two jets are $b$-jets, and non-$b$-lead where one of these jets is a $c$-jet.

\begin{cfig}
  \graphic[0.48]{%
int/figures/mCT/shapeStudies/mct_mbb_bb_ttbar_nominalDphiMin3.pdf}
  \graphic[0.48]{%
    int/figures/mCT/shapeStudies/mct_mbb_bc_cb_ttbar_nominalDphiMin3.pdf}
  \caption[Correlations between $\mct$ and $\mcc$ in $\ttbar$]{Correlations between $\mct$ and $\mcc$ in $\ttbar$ in the signal region.}
  \label{fig:mctmcc}
\end{cfig}

In the case of $b$-lead $\ttbar$, the boost-corrected $\mct$ is bound according to $\mct < 135\,\gev$~\cite{mctboost}.
In non-$b$-lead cases, where the two leading jets don't correspond to the $b$-jets in $t \to b \nu$, this bound no longer holds.
These cases give rise to the long tail of $\ttbar$ events shown in
\cref{fig:massctblind}.
Fortunately, the invariant mass of the leading two jets $\mcc$ gives another handle on these events.
Two dimensional $\ttbar$ distributions in the $\mcc$--$\mct$ plane are given in \cref{fig:mctmcc}.
In the case where the $b$-lead hypothesis holds, most events are concentrated with low $\mct$. When this assumption is violated, a second population appears at low $\mcc$ and slightly higher $\mct$.
This observation motivates a pair of anti-$\ttbar$ requirements; events are removed if $\mct < 150\,\gev$ or $\mcc < 200\,\gev$.

Additional $\pt$ requirements are imposed on the leading jets to reduce both $W$/$Z$ + jets and generic backgrounds.
A leading jet is required $\pt > 130\,\gev$, while the subleading jet must satisfy $\pt > 100\,\gev$.
Stack plots showing standard model background distributions, with overlaid signal points, are given in \cref{fig:discrim-sr}.
The final set of three signal regions are differentiated by their $\mct$ requirement;
these regions correspond to $\mct < 150$, $200$, and $250\,\gev$.

\newcommand{\srplot}[1]{int/figures/stackplots/dans/signal_mct150/#1}
\begin{cfig}
  \subfigure[$\mcc$]{\graphic[0.48]{\srplot{mass_cc_blinded.pdf}}}
  \subfigure[$\mct$]{\graphic[0.48]{\srplot{mass_ct_blinded.pdf}}
    \label{fig:massctblind}} \\
  \subfigure[Leading jet $\pt$]{\graphic[0.48]{\srplot{j0_pt_blinded.pdf}}}
  \subfigure[Subleading jet $\pt$]{\graphic[0.48]{\srplot{j1_pt_blinded.pdf}}}
  \caption[Discriminants in the signal region]{Discriminants used in the signal region.}
  \label{fig:discrim-sr}
\end{cfig}

\begin{figure}
  \begin{center}
  \includegraphics[width=0.50\textwidth]{%
    int/figures/stackplots/dans/preselection/met_eff.pdf}
  \caption[Effective missing transverse energy $\meteff$ after anti-QCD requirements in the signal region]{Distribution of $\meteff$ in data and simulated SM backgrounds after the $\met > 150\,\gev$ and $\Delta \phi(j,\vmet)$ requirements.
}
  \label{fig:meteff}
  \end{center}
\end{figure}

