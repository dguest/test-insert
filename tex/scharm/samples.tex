Almost all backgrounds and all signals models rely to some extent on simulated samples.
As mentioned in \cref{sec:regions-and-fitting}, the overall normalization of dominant backgrounds is extracted from data.
On the other hand, all other properties of the backgrounds are dictated by parameters of the simulation.

Within the colliding protons, the distribution of parton momenta is  characterized by \emph{parton distribution functions} (PDFs) which are measured from designated experiments and maintained by a number of international collaborations.
When two partons interact in the hard-scatter, the fundamental interactions between partons are simulated by a wide array of programs, each of which has various strengths and weaknesses.
%% The hard-scatter also includes tunable parameters.
Following the hard-scatter, several more methods can simulate jet fragmentation. Both the hard scatter and fragmentation process rely on a number of parameters, collectively referred to as the \emph{tune}.
Overall, there are a number of choices at each stage of the generation process.
The combination chosen depends on the process simulated, and is informed by a huge number of studies which this thesis can't realistically accommodate.

Particle interactions with the detector and the detector response takes place in a \textsc{geant4}~\cite{geant} framework~\cite{atlassimulation}.
While the precise detector alignment and material properties are periodically updated, they are generally standard across analyses. In some cases, the calorimeter simulation is accelerated using the \textsc{AtlFast-II}~\cite{atlfast} package.

\subsection{Signal Samples}
The analysis relies on several assumptions about the mass hierarchy of the signal. We assume that only two particles are easily accessible at LHC energies: the $\scharm$ and the $\neut$, with the $\neut$ the lightest.
This disallows intermediate decays such as $\scharm \to c + \supp{\chi}^{0}_2$ and $\scharm \to s + \cha$. All other sparticles are set to a mass of $4.5\,\tev$, and we assume a 100\% branching ration for $\sctoc$.
The $\sttoc$ analysis relied on \textsc{MadGraph}~\cite{madgraph5} with the \textsc{auet2b}~\cite{auet2b} tune for signal matrix-element calculations, since it accurately models initial state radiation. The generator was set to simulate the $c \bar{c}$ production plus one additional parton at next-to-leading order.
Fragmentation was passed off to \textsc{pythia6}~\cite{pythia2}, and calorimeter response is modeled with \textsc{AtlFast-II}.
For consistency, the $\sctoc$ search uses these packages as well.

\begin{figure}
  \includegraphics[width=\textwidth]{int/figures/acceff/gridpoints.pdf}
  \caption[Signal grid points]{Points in the signal grid for the $\sctoc$ analysis. Point color corresponds to the number of events generated for each point, in thousands. Note that only the points near the lower diagonal line of the $\sttoc$ region are actually considered in the non-boosted search discussed here---this search has no sensitivity to points closer to $m_{\scharm} = m_{\neut}$.}
  \label{fig:siggrid}
\end{figure}

The signal points are arranged in several overlapping signal grids. The roughest consists of signal points in a rectangular arrangement at $100\,\gev$ intervals in both $m_{\scharm}$ and $m_{\neut}$. This grid covers $m_{\scharm} \leq 800\,\gev$ and $m_{\neut} \leq 500\,\gev$, omitting points where $\Delta m < m_W$.
To increase resolution at the edge of the search's sensitivity, a second grid is overlaid with the same spacing but offset by $50\,\gev$ along the $m_{\scharm}$ and $m_{\neut}$ axes.
The second grid covers $m_{\scharm} \leq 550\,\gev$ and $m_{\neut} \leq 250\,\gev$, again omitting points where $\Delta m < m_W$.
A third set of points is borrowed from the $\sttoc$ grid to smoothly interpolate into the $\sttoc$ exclusion.
This consists of points where $m_{W} - 10\,\gev < \Delta m < m_{W}$. The full set of points used in the final search, including those where the final $\cls$ value was borrowed directly from the $\sttoc$ analysis, is shown in \cref{fig:siggrid}.

\subsection{Background Samples}

Background samples are simulated with a variety of generators, chosen according to the simulated process.
\begin{description}
\item[$\bm{t \bar{t}}$ production] The $t\bar{t}$ pairs are generated with \textsc{powheg}~\cite{powheg} and the \textsc{cteq6l1}~\cite{cteq} PDF set. They are passed to \textsc{pythia}, using the \textsc{perugia 2011C}~\cite{perugia} tune.
\item[Single $\bm{t}$ production] Dominant production channels---$Wt$ and $s$---are simulated with \textsc{powheg + pythia}, again with \textsc{cteq6l1} and \textsc{perguia 2011C}.
The $t$-channel is generated like the $Wt$ and $s$ channels, but using \textsc{AcerMC}~\cite{acermc} instead of \textsc{powheg}.
\item[$\bm{t \bar{t}} + \bm{W}$/$\bm{Z}$] Top pairs produced in association with $W$ and $Z$ bosons are generated with \textsc{MadGraph}, with the \textsc{auet2}~\cite{auet2} tune. Again \textsc{pythia} with \textsc{cteq6l1} is used jet fragmentation.
\item[$\bm{W}$/$\bm{Z} + $ jets] Bosons produced in association with jets are simulated using \textsc{sherpa}~\cite{sherpa}, which includes its own jet fragmentation model and a default tune. In this case, the \textsc{ct10}~\cite{CT10} PDF set is used.
\item[Diboson] Diboson events---$WW$, $WZ$, and $ZZ$---are generated in the same manner as $W$/$Z + $ jets events.
\end{description}

