While the LHC and ATLAS detector operated continuously through 2012 with very little down time, some data is inevitably corrupted by detector issues.
Electronic or data acquisition issues that arise on timescales of minutes--hours are filtered with the help of a \emph{good runs list} (GRL).
Good runs lists are tailored to the demands of the analysis, since not all analyses require the same functional subsystems.
All runs recorded by ATLAS are broken into 1 minute \emph{lumi-blocks} which are reviewed by ATLAS personnel.
Lumi-blocks with defects in critical subsystems removed.

On a shorter timescale, some recorded events may be corrupted by physical processes or detector issues. In either case, events must be cleaned for analysis. To remove poorly reconstructed events, the primary vertex must be assigned at least 4 tracks.
Events with badly reconstructed muons or jets are also removed, as are events where the timing of the leading two jets suggests they may have been produced outside the bunch-crossing.
These requirements, and several others more specific to hardware issues, are listed in \cref{tab:event-clean}.

\begin{table}
  \begin{center}
  \begin{tabular}{|c|l|l|}
\hline
Cut & Description & Implementation \\
\hline
\hline
\multirow{2}{*}{1}  & \multirow{2}{*}{Data quality GRL} & Lumi blocks must be in Good Runs List  \\
   & & GRL: \texttt{data12\_8TeV DetStatus-v61-pro14-02} \\ \hline
%% 2  & Trigger & \texttt{EF\_xe80\_tclcw\_tight} \\\hline
2  & Good primary vertex & Require $\geq$ 4 tracks pointing to primary vertex\\ \hline
\multirow{2}{*}{3}  & \multirow{2}{*}{Bad muon cleaning} & Veto event if preselected muons (before overlap removal) \\
& &  with $\sigma({q}/{p})/{|{q}/{p}|} > 0.2$ are found \\ \hline

\multirow{2}{*}{4}  & \multirow{2}{*}{Jet cleaning} & Veto events with jets failing \veryloose{} and $\pt > 20\,\gev$ \\
   &  & (after jet-electron overlap removal) \\ \hline
5  & Bad Tile Cells & Reject events affected by nonoperational tile drawers \\ \hline
6  & LAr \& Tile quality & $\mathtt{larError} \neq 2$ \& $\mathtt{tileError} \neq 2$\\ \hline
7  & \textsc{TileTripReader} & Veto if event coincides with trip in tile calorimeter \\ \hline
8  & TTC reset & \texttt{(coreFlags \& 0x40000) \!= 0}\\ \hline
%% 10 & Jet centrality & Jets with $\left|\eta\right| > 2.5$ are removed from the list of jets\\ \hline
%% 11 & Pile-up jets & Jets with $\pt \leq 50\,\gev$, $\left|\eta\right| \leq 2.4$ and $|\mathrm{JVF}| \leq 0.5$ are removed \\ \hline
9 & Timing & Average timing of two leading jets within 5 ns of bunch crossing\\ \hline
\multirow{3}{*}{10} & \multirow{3}{*}{Jet charged fraction} & Veto event if for either of the leading two jets: \\ & & $\pt > 100\,\gev$, $|\eta| < 2$, and \\
& & $\chf < 0.02$ \logicor ($\chf < 0.05$ and $\emf > 0.9$)\\ \hline
\hline
\end{tabular}

  \caption[Event cleaning requirements]{Event cleaning requirements. }
  \label{tab:event-clean}
  \end{center}
\end{table}
