The reconstruction of jets and leptons is imperfect in any collider: a pion from a jet can resemble an electron; a $\tau$-lepton can resemble a jet; a $c$-jet can easily resemble a jet from QCD interactions in the underlying event.
Object definitions are thus a compromise between efficiency and fake rates.
Since events contain multiple objects, the optimal efficiencies and fake rates are very analysis dependent.
This section gives a detailed description of the object definitions in the $\sctoc$ search.

In general, objects are selected in three steps: first a \emph{preselected} collection is formed for jets, electrons, and muons; next an overlap removal stage resolves ambiguity between objects, and finally \emph{signal} objects are defined with more stringent quality requirements.

\subsection{Jets}
\label{sec:jets}
%% Hadronic QCD jets are orders of magnitude more common than leptons at the LHC.
%% They are also difficult to measure properly, since they consist of
Jets are formed from calorimeter clusters, which are clustered with the anti-$k_t$ algorithm~\cite{antikt} with a radius parameter of 0.4.
All jets are calibrated with a local cluster weighting~\cite{LCJets}.
In the case of simulated jets the same flavor labeling scheme as described in \cref{tag:sec:data-and-simulation} is applied. The preselected jets consist of all jets with with $\pt > 20\,\gev$. To remove electrons which have been reconstructed as jets, any jet with a preselected electron within $\Delta R < 0.2$ is discarded.

\begin{table}
  \begin{center}
    \begin{tabular}{|l|c|}
\hline
Background & Requirement \\ \hline
\hline
\multicolumn{2}{|c|}{All Jets} \\ \hline
\multirow{2}{*}{Hadronic Endcap Noise} & $\nege \leq 60\,\gev$ \\ \cline{2-2}
 & $\hecf \leq 0.5$ \logicor $|\hecq| \leq 0.5$ \logicor $\qmean < 0.8$ \\ \hline
\hline
\multicolumn{2}{|c|}{Forward Jets ($|\eta| \geq 2$)} \\ \hline
Non-collision & $\emf \geq 0.05$ \\ \hline
\hline
\multicolumn{2}{|c|}{Central Jets ($|\eta| < 2.8$)} \\ \hline
EM Calo Noise & $\emf \leq 0.95$ \logicor $|\larq| \leq 0.8$ \logicor $\qmean < 0.8$ \\ \hline
\hline
\multicolumn{2}{|c|}{More Central Jets ($|\eta| < 2$)} \\ \hline
\multirow{2}{*}{Non-collision} & $\emf \geq 0.05 $ \logicor $ \chf \geq 0.05$ \\
\cline{2-2}
 & $\fmax \leq 0.99$ \\ \hline
\end{tabular}

    \caption[Jet \veryloose{} requirements]{Jet requirements to pass the \veryloose{} selection.}
    \label{tab:veryloosejet}
  \end{center}
\end{table}

A number of quality requirements are applied to the preselected jets, as tabulated in \cref{tab:veryloosejet}. These requirements rely on the following parameters~\cite{beam-backgrounds}:
\begin{description}
\item[$\emf$] The fraction of the jet energy deposited in the electromagnetic calorimeter.
\item[$\chf$] Charged jet fraction is defined as
  \begin{equation}
    \chf = \sum_{i \in \{\text{tracks}\}} \pt^{i} / \pt^{\text{jet}},
  \end{equation}
  where the sum is over all tracks associated to the jet.
\item[$\fmax$] Maximum fraction of calorimeter energy contained in one layer.
\item[$\nege$] Sum of negative energy in cells used by the jet.
\item[$\hecf$] Fraction of jet energy in the hadronic end cap.
\item[$\hecq$] Fraction of jet energy in the hadronic end cap which corresponds to cells with a $Q$-factor above 4000. The $Q$-factor is a measurement of pulse shape quality for a cell. It is defined as
\begin{equation}
  Q = \sum_{i \in \{\text{samples}\}} (m_i - s_i)^2
\end{equation}
where $m_i$ is the measured amplitude in sample $i$, and $s_i$ is the simulated amplitude for the same sample.
\item[$\larq$] Same as $\hecq$, but for the liquid argon calorimeter.
\item[$\qmean$] Average $Q$-factor for cells in the jet, weighted by energy squared.
\item[$\jvf$] Jet vertex fraction is the sum of the $\pt$ of tracks matched to the primary vertex divided by the $\pt$ sum of all tracks in a jet.
\end{description}
If a preselected jet fails any of these requirements, the entire event is discarded.

Signal jets are subjected to two further requirements.
The first requirement is due to $c$-tagging; jets must be within the central region of the detector where tracking is possible. To this end jets with $|\eta| < 2.5$ are skipped when selecting signal jets.
The second requirement mitigates pileup; low $\pt$ central jets are required to be associated with the primary vertex. For jets with $\pt \leq 50\,\gev$ and $|\eta| \leq 2.4$, more than half the sum of track $\pt$ must be carried by tracks associated to the primary vertex.

\begin{table}
  \begin{center}
    \begin{tabular}{|l|c|}
\hline
Requirement & Value \\ \hline
\hline
\multicolumn{2}{|c|}{Preselected Jet} \\ \hline
$\pt$ & $\pt > 20\,\gev$ \\ \hline
\hline
\multicolumn{2}{|c|}{Overlap Cleaned Jet} \\ \hline
Superset & Preselected Jets \\ \hline
Overlap Removal & No preselected electrons with $\Delta R(j,e) < 0.2$ \\ \hline
\hline
\multicolumn{2}{|c|}{Event Veto Jets} \\ \hline
Superset & Overlap Cleaned Jets \\ \hline
Quality & fail \veryloose{} \\ \hline
\hline
\multicolumn{2}{|c|}{Signal Jets} \\ \hline
Superset & Overlap Cleaned Jets \\ \hline
$\eta$ & $|\eta| < 2.5$ \\ \hline
Jet Vertex Fraction & $\jvf > 0.5$ \logicor $\pt > 50\,\gev$ \logicor $|\eta| > 2.4$ \\ \hline
\end{tabular}

    \caption[Jet definitions for the $\sctoc$ search]{Jet definitions in the $\sttoc$ search.}
  \end{center}
\end{table}
\subsubsection{$c$-tagging Operating Points}
The $\sctoc$ signal regions assume the most energetic jets in the event will be $c$-jets from the $\scharm$ decay.
Based on this assumption, the leading two jets must pass the ``medium'' $c$-tagging requirement as discussed in \cref{tag:sec:op}. The location of this operating point in the light vs $b$ rejection plane is given in \cref{fig:c-tag-medium}.
Uncertainties associated with $c$-tagging are potentially very large, and
these uncertainties propagate to an uncertainty in the expected yield in each region.
To diminish the total yield uncertainty in the signal region, all signal and control regions include the same $c$-tagging requirement.
Ideally this allows the fit to absorb $c$-tagging systematics into the normalization of the backgrounds.
In practice, due to differing flavor composition between regions, this absorption is imperfect and flavor-tagging systematics remain large.

\begin{figure}
  \begin{center}
  \includegraphics[width=0.7\textwidth]{figures/external/rejrej-simple.pdf}
  \caption[JetFitterCharm medium operating point on the light vs $b$ rejection plane]{Light vs $b$ rejection for various $c$ tagging efficiencies, showing the medium operating point.}
  \label{fig:c-tag-medium}
  \end{center}
\end{figure}

\subsection{Leptons}
Leptons are excluded from the signal region, but are nonetheless important to identify events in the control region.
As with jets, preselected leptons are selected with relatively loose criteria.
These leptons veto events from the signal region.
The more stringent signal leptons are required in events populating the control regions.
Control regions are populated with events which fire single lepton triggers. To simplify the interpenetration of trigger efficiencies, only signal leptons which fire the trigger are considered.

\subsubsection{Electrons}
All tracks in ATLAS are initially fit under the pion hypothesis.
A subset of these tracks are identified as electrons with several algorithms.
The standard \textsc{electron} algorithm seeds electrons with calorimeter clusters.
A second \textsc{softe} algorithm seeds electrons from tracks, and a third \textsc{frwd} reconstructs electrons in the forward calorimeter.
In all cases the initial track is refit as an electron which allows the fitter to account for bremsstrahlung.

This analysis only considers electrons seeded with the \textsc{electron} algorithm. Electrons that are seeded with both \textsc{electron} and \textsc{softe} are also allowed.\footnote{Within ATLAS the \textsc{electron} authorship requirement is usually summarized as ``require author equal to 1 or 3''; the electron authorship is stored in a bit field, where the first bit corresponds to \textsc{electron} and the second to \textsc{softe}.}
Preselected electrons must also have $\pt > 7\,\gev$ and $|\eta| < 2.47$, where both are based on the calorimeter cluster information, to ensure that tracks are within the tracking system.
Electrons reconstructed with problematic hardware are marked with designated \texttt{el\_OQ} flags and removed.
Fake electrons are reduced by the \textsc{Medium++} requirement, which combines electron shower shapes, track--cluster matching, tracking hit requirements, and TRT high-threshold hits.

Real electrons can be produced within jets, which causes potential ambiguity with object definitions.
As stated above, when $\Delta R(e,j) < 0.2$, the object is defined as an electron and the jet is removed.
In cases where $0.2< \Delta R(e,j) < 0.4$, the jet is kept and the electron is removed.
With larger separation both objects are kept.

Several further requirements are imposed on signal electrons.
The quality requirement is raised to \textsc{Tight++}.\footnote{The \textsc{Tight++} electron requirement uses roughly the same variables as \textsc{Medium++}, but tightens the thresholds.}
In addition the electron must be isolated in both the tracker and the calorimeter.
%% In the tracker, the combined $\pt$ of tracks surrounding the electron in a cone defined by $\Delta R < 0.3$ must sum to less than 15\% of the electron $\pt$.
These isolation requirements are described in \cref{tab:electrons}.
Finally, we require a low impact parameter: the significance of the transverse impact parameter must be below 5, and a shift of less than $0.4\,\text{mm}$ perpendicular to the track direction must bring the longitudinal displacement into agreement with the primary vertex.

\begin{table}
  \begin{center}
  \begin{tabular}{|l|c|}
\hline
Requirement            & Value \\
\hline
\hline
\multicolumn{2}{|c|}{Preselected Electron}\\
\hline
AuthorElectron      &  1 or 3 \\
\hline
Acceptance     & $E_\T > 7~\gev, |\eta^\mathrm{clust}| < 2.47$         \\
\hline
Quality & MediumPP \\
\hline
Cleaning & cut on quality flag \texttt{el\_OQ} \\
\hline
Overlap      & $\Delta{}R(e,jet)<0.2$ or $>0.4$ \\
\hline
\hline
\multicolumn{2}{|c|}{Signal Electron}\\
\hline
Quality & Tight++ \\
%\hline
%Acceptance     & $|\eta^\mathrm{clust}| < 2.47$, ($E_T > 20~\gev$)        \\%the \pt requirement is driven by the trigger thresholds. For leading lepton in any lepton-triggered region it is 25 GeV, for sub-leading leptons use the preselected lepton acceptance cuts.
\hline
%\multicolumn{2}{|c|}{Old IsSignalElecton}  \\
%\hline
%Isolation   & ptCone20/\pt\ $<$ 0.1\\
%\hline
%\hline
%\multicolumn{2}{|c|}{New IsSignalElectronExp} \\
% _etcut(25000.), _id_isocut(0.16), _calo_isocut(0.18), _d0sigcut(5.), _z0cut(0.4), _pt_isoMax(0.) 
Track Isolation   & $\mathtt{ptCone30}/\pt < 0.15$\\
\hline
PU corr Calo Isolation & $\mathtt{topoEtcone30\_corrected} (+ \text{PU corr})/\pt < 0.18$\\
\hline
Longitudinal IP & $z_0 <$ 0.4mm\\
\hline
Transverse IP signifiance & $d_0/\sigma_{d_0} < 5$\\
\hline
\end{tabular}

  \caption[List of electron selection criteria]{Electron definitions. The \textsc{Medium++} and \textsc{Tight++} requirements are standard ATLAS-wide definitions to reject fake electrons. Object quality flags (\texttt{el\_OQ}) indicate that hardware issues may compromise the electron; requiring \texttt{(el\_OQ \& 1446) == 0} means that none of the bits in the \texttt{BADCLUSTERELECTRON} bitmask are set. The pileup correction $k$ in the calorimeter isolation depends on the number of vertices in the event with 5 or more tracks ($\nvxp$), and is different in data ($20.15\,\mev/\text{vertex}$) and simulation ($17.94\,\mev/\text{vertex}$).}
  \label{tab:electrons}
  \end{center}
\end{table}

\subsubsection{Muons}
Muons in this analysis are seeded from inner detector tracks and either combined with a full track in the muon spectrometer or with track segments, using the \textsc{staco} algorithm.
Preselected muon tracks must have $\pt > 6\,\gev$ and $|\eta| < 2.4$.
A number of quality further quality requirements are listed in \cref{tab:muons}.
Muons created within hadronic jets are removed by discarding any within $\Delta R < 0.4$ of a preselected jet.

Signal muon selection adds requirements. High energy cosmic muons from the upper atmosphere occasionally penetrate the ATLAS cavern and leave tracks in the detector. To guard against these, only muons which pass near the primary vertex are accepted.
As with electrons, signal muons must be isolated in the tracker and calorimeter.

\begin{table}
  \begin{center}
  \begin{tabular}{|l|c|}
\hline
Cut            & Value \\
\hline
\hline
\multicolumn{2}{|c|}{Preselected muon}\\
\hline
Algorithm      & STACO, combined or segment-tagged muon \\
\hline
Acceptance     & $\pt > 6~\gev, |\eta| < 2.4$          \\
\hline
Quality        & Loose    \\
\hline
ID track quality & $\geqslant 1$ b-layer hit when it can be expected \\
                 & $\geqslant 1$ pixel hit or crossed dead pixel sensor \\
                 & $\geqslant 5$ SCT hits or crossed dead SCT sensor\\
                 & pixel holes + SCT holes $< 3$\\
                 & If $0.1 < |\eta| < 1.9$: $n_{\mathrm{TRT}} \geqslant 6$ or $n_{\mathrm{TRT}}^{\mathrm{outliers}} < 0.9 n_{\mathrm{TRT}}$ \\
               & If $|\eta| \geqslant 1.9$ and $n_{\mathrm{TRT}} \geqslant 6$: $n_{\mathrm{TRT}}^{\mathrm{outliers}} < 0.9 n_{\mathrm{TRT}}$ \\
\hline
Overlap     & $\Delta{}R(\mu,\text{jet}) > 0.4$\\
\hline
\hline
\multicolumn{2}{|c|}{Signal muon}\\
\hline
%Acceptance     & $|\eta| < 2.4$ ($\pt > 20~\gev$)          \\%the \pt requirement is driven by the trigger thresholds. For leading lepton in any lepton-triggered region it is 25 GeV, for sub-leading leptons use the preselected lepton acceptance cuts.
%\hline
Quality        & Loose    \\
\hline
Cosmics        & $|z_{\mu} - z_{\mathrm{PV}}| < 1\,\mathrm{mm}$, $d_0 < 0.2$ mm          \\
\hline
%\hline
%\multicolumn{2}{|c|}{Old IsSignalMuon} \\
%\hline
%Isolation & ptCone20 $<$ 1.8 GeV \\
%\hline
%\hline
%\multicolumn{2}{|c|}{New IsSignalMuonExp} \\
%_ptcut(25000.), _id_isocut(0.12), _calo_isocut(0.12), _d0sigcut(3.), _z0cut(0.4), _pt_isoMax(0.) {}
%\hline
Track Isolation & $\mathtt{ptcone30\_trkelstyle}/\pt < 0.12$ \\
\hline
PU corr Calo Iso & $\mathtt{etcone30} (+ \text{PU corr})/\pt < 0.12$ \\
\hline
Longitudinal IP & $z_0 <$ 0.4mm\\
\hline
Transverse IP signifiance & $d_0/\sigma_{d_0} < 3$\\
\hline
\end{tabular}

  \caption[List of muon selection criteria]{Muon definitions. The \textsc{Loose} quality requirement is an ATLAS-wide standard.
The pileup corrections $k_1$ and $k_2$ in the calorimeter isolation depends on the number of vertices in the event with 5 or more tracks ($\nvxp$). In data these corrections are $k_1 = 64.8\,\mev/\text{vertex}$, $k_2 = 0.98\,\mev/\text{vertex}^2$. In simulation they are $k_1 = 69.2\,\mev/\text{vertex}$, $k_2 = 0.76\,\mev/\text{vertex}^2$.
}
  \label{tab:muons}
  \end{center}
\end{table}
%% \subsection{Overlap Removal}
\subsection{Missing Energy}
\label{sec:met}
The objects described above are combined to produce a number of high level variables, which will be described where appropriate.
One of the most fundamental is $\met$, which in this search is defined by~\cite{metpaper}
\begin{equation}
E^{\text{miss}}_{\T\,\mathrm{RefFinal}} =  \left| \subvmet{RefJet} + \subvmet{RefGamma} + \subvmet{Muon} +  \subvmet{RefEle} + \subvmet{CellOut} \right|.
\end{equation}
The terms $\subvmet{RefJet}$, $\subvmet{RefGamma}$, $\subvmet{Muon}$, and $\subvmet{RefEle}$ are defined as the negative vector sum of the calibrated jet, photon, muon, and electron objects, respectively.
The $\subvmet{CellOut}$ term accounts for clusters not assigned to any calibrated objects.
Some acceptance limitations apply to these objects: jets below $\pt = 20\,\gev$ aren't considered, nor are muons with $\pt < 7\,\gev$ or electrons with $\pt < 10\,\gev$~\cite{koomet}.

%% \begin{equation}
%% \meff = \sum_{i\leq n} (\pt^{\rm jet})_i + \met
%% \end{equation}
%% Most of this will be a table
%% \subsubsection{$\met$}

