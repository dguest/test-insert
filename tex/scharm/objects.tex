The reconstruction of jets and leptons is imperfect in any collider: a pion from a jet can resemble an electron; a $\tau$-lepton can resemble a jet; a $c$-jet can easily resemble a jet from QCD interactions in the underlying event.
Object definitions are thus a compromise between efficiency and fake rates.
Since events contain multiple objects, the optimal efficiencies and fake rates are very analysis dependent.
This section gives a detailed description of the object definitions in the $\sctoc$ search.

In general, objects are selected in three steps: first a ``preselected'' collection is formed for jets, electrons, and muons; next an overlap removal stage resolves ambiguity between objects, and finally ``signal'' objects are defined with more stringent quality requirements.

\subsection{Jets}
%% Hadronic QCD jets are orders of magnitude more common than leptons at the LHC.
%% They are also difficult to measure properly, since they consist of
Jets are formed from calorimeter clusters, which are clustered with the anti-$k_t$ algorithm~\cite{antikt} with a radius parameter of 0.4.
All jets are calibrated with a local cluster weighting~\cite{LCJets}.
In the case of simulated jets the same flavor labeling scheme as described in \cref{tag:sec:data-and-simulation} is applied. The preselected jets consist of all jets with with $\pt > 20\,\gev$. To remove electrons which have been reconstructed as jets, any jet with a preselected electron within $\Delta R < 0.2$ is discarded.

A number of quality requirements are applied to the preselected jets, as tabulated in \cref{tab:veryloosejet}. If a preselected jet fails any of these requirements, the entire event is discarded. These requirements rely on the following parameters~\cite{beam-backgrounds}:
\begin{description}
\item[$\emf$] The fraction of the jet energy deposited in the electromagnetic calorimeter.
\item[$\chf$] Charged jet fraction is defined as
  \begin{equation}
    \chf = \sum_{i \in \{\text{tracks}\}} \pt^{i} / \pt^{\text{jet}},
  \end{equation}
  where the sum is over all tracks associated to the jet.
\item[$\fmax$] Maximum fraction of calorimeter energy contained in one layer.
\item[$\nege$] Sum of negative energy in cells used by the jet.
\item[$\hecf$] Fraction of jet energy in the hadronic end cap.
\item[$\hecq$] Fraction of jet energy in the hadronic end cap which corresponds to cells with a $Q$-factor above 4000. The $Q$-factor is a measurement of pulse shape quality for a cell. It is defined as
\begin{equation}
  Q = \sum_{i \in \{\text{samples}\}} (m_i - s_i)^2
\end{equation}
where $m_i$ is the measured amplitude in sample $i$, and $s_i$ is the simulated amplitude for the same sample.
\item[$\larq$] Same as $\hecq$, but for the liquid argon calorimeter.
\item[$\qmean$] Average $Q$-factor for cells in the jet, weighted by energy squared.
\item[$\jvf$] Jet vertex fraction is the sum of the $\pt$ of tracks matched to the primary vertex divided by the $\pt$ sum of all tracks in a jet.
\end{description}

\begin{table}
  \begin{center}
    \begin{tabular}{|l|c|}
\hline
Background & Requirement \\ \hline
\hline
\multicolumn{2}{|c|}{All Jets} \\ \hline
\multirow{2}{*}{Hadronic Endcap Spikes} & $\nege \leq 60\,\gev$ \\ \cline{2-2}
 & $\hecf \leq 0.5$ \logicor $|\hecq| \leq 0.5$ \logicor $\qmean < 0.8$ \\ \hline
\hline
\multicolumn{2}{|c|}{Forward Jets ($|\eta| \geq 2$)} \\ \hline
Non-collision & $\emf \geq 0.05$ \\ \hline
\hline
\multicolumn{2}{|c|}{Central Jets ($|\eta| < 2.8$)} \\ \hline
EM Calo Noise & $\emf \leq 0.95$ \logicor $|\larq| \leq 0.8$ \logicor $\qmean < 0.8$ \\ \hline
\hline
\multicolumn{2}{|c|}{More Central Jets ($|\eta| < 2$)} \\ \hline
\multirow{2}{*}{Non-collision} & $\emf \geq 0.05 $ \logicor $ \chf \geq 0.05$ \\
\cline{2-2}
 & $\fmax \leq 0.99$ \\ \hline
\end{tabular}

    \caption[Jet \veryloose{} requirements]{Jet requirements to pass the \veryloose{} selection.}
    \label{tab:veryloosejet}
  \end{center}
\end{table}
\begin{table}
  \begin{center}
    \begin{tabular}{|l|c|}
\hline
Requirement & Value \\ \hline
\hline
\multicolumn{2}{|c|}{Preselected Jet} \\ \hline
$\pt$ & $\pt > 20\,\gev$ \\ \hline
\hline
\multicolumn{2}{|c|}{Overlap Cleaned Jet} \\ \hline
Superset & Preselected Jets \\ \hline
Overlap Removal & No preselected electrons with $\Delta R(j,e) < 0.2$ \\ \hline
\hline
\multicolumn{2}{|c|}{Event Veto Jets} \\ \hline
Superset & Overlap Cleaned Jets \\ \hline
Quality & fail \veryloose{} \\ \hline
\hline
\multicolumn{2}{|c|}{Signal Jets} \\ \hline
Superset & Overlap Cleaned Jets \\ \hline
$\eta$ & $|\eta| < 2.5$ \\ \hline
Jet Vertex Fraction & $\jvf > 0.5$ \logicor $\pt > 50\,\gev$ \logicor $|\eta| > 2.4$ \\ \hline
\end{tabular}

    \caption[Jet definitions]{Jet definitions.}
  \end{center}
\end{table}
\subsubsection{$c$-tagging Operating Points}
%% Charm tagging takes place before the final stage of calibration.
%% As such, the 
\subsection{Leptons}
\subsubsection{Electrons}
\begin{table}
  \begin{center}
  \begin{tabular}{|l|c|}
\hline
Requirement            & Value \\
\hline
\hline
\multicolumn{2}{|c|}{Preselected Electron}\\
\hline
Author      &  1 or 3 \\
\hline
Acceptance     & $E_\T > 7~\gev, |\eta^\mathrm{clust}| < 2.47$         \\
\hline
Quality & \textsc{Medium++} \\
\hline
Cleaning & $(\mathtt{el\_OQ \& 1446}) = 0$  \\
\hline
Overlap      & $\Delta{}R(e,j)<0.2$ OR $>0.4$, $j \in$ preselected jets \\
\hline
\hline
\multicolumn{2}{|c|}{Signal Electron}\\
\hline
Quality & \textsc{Tight++} \\
%\hline
%Acceptance     & $|\eta^\mathrm{clust}| < 2.47$, ($E_T > 20~\gev$)        \\%the \pt requirement is driven by the trigger thresholds. For leading lepton in any lepton-triggered region it is 25 GeV, for sub-leading leptons use the preselected lepton acceptance cuts.
\hline
%\multicolumn{2}{|c|}{Old IsSignalElecton}  \\
%\hline
%Isolation   & ptCone20/\pt\ $<$ 0.1\\
%\hline
%\hline
%\multicolumn{2}{|c|}{New IsSignalElectronExp} \\
% _etcut(25000.), _id_isocut(0.16), _calo_isocut(0.18), _d0sigcut(5.), _z0cut(0.4), _pt_isoMax(0.) 
Track Isolation   & $\mathtt{ptCone30} < 0.16 \pt$\\
\hline
Calorimeter Isolation & $\mathtt{topoEtcone30\_corrected} - k \nvxp < 0.18 \pt$\\
\hline
Longitudinal IP & $z_0 \sin(\theta) < 0.4\,\text{mm}$\\
\hline
Transverse IP signifiance & $d_0/\sigma_{d_0} < 5$\\
\hline
\end{tabular}

  \caption[List of electron selection criteria]{Electron definitions.}
  \end{center}
\end{table}
\subsubsection{Muons}
\begin{table}
  \begin{center}
  \begin{tabular}{|l|c|}
\hline
Requirement            & Value \\
\hline
\hline
\multicolumn{2}{|c|}{Preselected muon}\\
\hline
Algorithm      & \textsc{staco}, combined or segment-tagged muon \\
\hline
Acceptance     & $\pt > 6~\gev, |\eta| < 2.4$          \\
\hline
Quality        & \textsc{Loose}    \\
\hline
\multirow{6}{*}{ID track quality} & $\geq 1$ b-layer hit when it can be expected \\
                 & $\geq 1$ pixel hit or crossed dead pixel sensor \\
                 & $\geq 5$ SCT hits or crossed dead SCT sensor\\
                 & pixel holes + SCT holes $< 3$\\
                 & If $0.1 < |\eta| < 1.9$: $n_{\mathrm{TRT}} \geq 6$ or $n_{\mathrm{TRT}}^{\mathrm{outliers}} < 0.9 n_{\mathrm{TRT}}$ \\
               & If $|\eta| \geq 1.9$ and $n_{\mathrm{TRT}} \geq 6$: $n_{\mathrm{TRT}}^{\mathrm{outliers}} < 0.9 n_{\mathrm{TRT}}$ \\
\hline
Overlap     & $\Delta{}R(\mu,j) > 0.4$, $j \in $ preselected jets\\
\hline
\hline
\multicolumn{2}{|c|}{Signal muon}\\
\hline
%Acceptance     & $|\eta| < 2.4$ ($\pt > 20~\gev$)          \\%the \pt requirement is driven by the trigger thresholds. For leading lepton in any lepton-triggered region it is 25 GeV, for sub-leading leptons use the preselected lepton acceptance cuts.
%\hline
%% Quality        & Loose    \\
%% \hline
Cosmics Veto   & %% $|z_{\mu} - z_{\mathrm{PV}}| < 1\,\mathrm{mm}$
$d_0 < 0.2\,\text{mm}$          \\
\hline
%\hline
%\multicolumn{2}{|c|}{Old IsSignalMuon} \\
%\hline
%Isolation & ptCone20 $<$ 1.8 GeV \\
%\hline
%\hline
%\multicolumn{2}{|c|}{New IsSignalMuonExp} \\
%_ptcut(25000.), _id_isocut(0.12), _calo_isocut(0.12), _d0sigcut(3.), _z0cut(0.4), _pt_isoMax(0.) {}
%\hline
Track Isolation & $\mathtt{ptcone30\_trkelstyle} < 0.12 \pt$ \\
\hline
Calorimeter Isolation & $\mathtt{etcone30} - k_1 \nvxp - k_2 (\nvxp)^2 < 0.12 \pt$ \\
\hline
Longitudinal IP & $z_0 < 0.4\,\text{mm}$ \\
\hline
Transverse IP signifiance & $d_0/\sigma_{d_0} < 3$\\
\hline
\end{tabular}

  \caption[List of muon selection criteria]{Muon definitions.}
  \end{center}
\end{table}
\subsection{Overlap Removal}
\subsection{Higher Vairables}
\begin{equation}
\meff = \sum_{i\leq n} (\pt^{\rm jet})_i + \met
\end{equation}
Most of this will be a table
\subsubsection{$\met$}

