The background model and systematic uncertainties described in \cref{sec:backgrounds,sec:systematics}
were chosen primarily by experience; the $\sctoc$ is the second
$c$-tagged search within \atlas, and was preceded by dozens of $\met + \text{jets}$ based searches in the $\cmenergy$ dataset alone.
Nevertheless, even the most carefully scrutinized background model can
be flawed. As a precaution against unblinding the signal regions with an
incorrect background model, I first checked several ``validation'' regions which select primarily background events similar to, but mutually exclusive of, the signal events.

Two validation regions were defined. Like the signal regions, the validation regions require a $\met$
trigger, no leptons, and two $c$-tagged jets. To avoid signal contamination
and ensure separation between signal and validation events, the two regions
invert either the $\mct$ or $\mcc$ requirement. Thus \vrmct\ includes events where $\mct$ is below $150\,\text{GeV}$ (but are otherwise identical to signal events), while \vrmcc\ includes events where $\mcc < 200\,\text{GeV}$.

The distributions of important variables in these regions are shown in \cref{fig:dist-vrmct,fig:dist-vrmcc}, where the backgrounds have been normalized in two ways; the ``pre-fit'' normalization assumes the nominal simulated cross-section as defined by the SUSY group, whereas the ``post-fit'' normalization multiplies the pre-fit values by normalization
parameters obtained in the control regions. In \cref{tab:vrs} these two
predictions are compared. The fit results in slightly larger-than-nominal predictions in both validation regions.

\Cref{tab:vrs} also compares the fitted prediction to the number of observed events. In both cases the observed data shows a deficit of the order of $\sim 1.3 \sigma$ when compared to simulation. Obviously any new particles in these regions should result in a surplus in data than a deficit, which leaves several explanations:
\begin{enumerate}
\item Statistical fluctuation, \label{enum:statfluc}
\item Mismodeling of backgrounds, \label{enum:mismodel}
\item New non--standard-model particles in the control regions, \label{enum:new}
%% \item Some combination of \cref{enum:statfluc,enum:mismodel,enum:new}
%% \label{enum:comb}
\end{enumerate}
Of these possibilities \cref{enum:new} is very unlikely given that no non--standard-model particles decaying to leptons have been discovered in 8~TeV data, despite dozens of published searches.
\Cref{enum:statfluc} is plausible given the low statistical significance of
the deficit, but less-so given that the deficit exists in both (statistically independent) regions. Item \cref{enum:mismodel}, by contrast,
could easily explain a systematic deficit, since any mismodeling would likely be correlated between the two similar regions. In any case, the
discrepancy is, roughly speaking, a 1 in 10 anomaly: while it could suggest a flawed model, altering the model to ``fit'' the data is arguably an equally flawed methodology. In the interest of scientific integrity, we
moved on to unblinding the search without redefining the control or signal
regions.

\begin{figure}[!p]
\centering
%Add nbtags, mbb and MET
\includegraphics[width=0.45\columnwidth]{int/figures/stackplots/dans/fr_mct/mass_ct.pdf}
\includegraphics[width=0.45\columnwidth]{int/figures/stackplots/dans/fr_mct/mass_ct_afterFit.pdf}\\
\includegraphics[width=0.45\columnwidth]{int/figures/stackplots/dans/fr_mct/met.pdf}
\includegraphics[width=0.45\columnwidth]{int/figures/stackplots/dans/fr_mct/met_afterFit.pdf}
\caption[Distributions of $\mct$ and $\met$ in \vrmct\ before and after the fit]{Distributions of $\mct$ and $\met$ in \vrmct\ before (left) and after (right) performing the fit.}
\label{fig:dist-vrmct}
\end{figure}

\begin{figure}[!p]
\centering
%Add nbtags, mbb and MET
\includegraphics[width=0.45\columnwidth]{int/figures/stackplots/dans/fr_mcc/mass_cc.pdf}
\includegraphics[width=0.45\columnwidth]{int/figures/stackplots/dans/fr_mcc/mass_cc_afterFit.pdf}\\
\includegraphics[width=0.45\columnwidth]{int/figures/stackplots/dans/fr_mcc/met.pdf}
\includegraphics[width=0.45\columnwidth]{int/figures/stackplots/dans/fr_mcc/met_afterFit.pdf}
\caption[Distributions of $\mct$ and $\mcc$ in the \vrmcc\ region before and after the fit.]{Distributions of $\mct$ and $\mcc$ in \vrmcc\ before (left) and after (right) performing the fit.}
\label{fig:dist-vrmcc}
\end{figure}


\begin{table}
\begin{center}
\input{int/tables/generated/vrsr/vrcrsr/vr_fit/yieldtable}
\caption[Results of the background-only fit for the validation regions]{Background-only fit results extrapolated to validation region yields. The total background uncertainty is systematic only. The systematic variations are given in more detail in \cref{tab:vrs-systable}.}
\label{tab:vrs}
\end{center}
\end{table}

\begin{table}
\begin{center}
\input{int/tables/generated/vrsr/vrcrsr/vr_fit/systable}
\caption[Background-only fit systematics extrapolated to the validation regions]{Background-only fit results extrapolated to validation regions. }
\label{tab:vrs-systable}
\end{center}
\end{table}
