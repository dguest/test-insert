
Physics analysis within ATLAS is organized according to a branching hierarchy. At the top level, analyses are divided according to the major components of the LHC physics program. These major working groups operate with relative autonomy and set their own standards for publications. Analyses generated by the ``Standard Model'' group focus on measurements of known quantities, and thus strive for transparency and to minimize elaborate fits and model-dependent conclusions. On the other end of spectrum, ``Exotics'' analyses use a variety of innovative techniques to control for Standard Model backgrounds, and test hypothetical models with both Bayesian and Frequentest statistics. The ``Higgs'' group follows a variety of analysis strategies, but in the interpenetration of results the frequentest dogma reigns supreme. The SUSY group opts for a strategy similar to that in the Higgs group---statistical interpretations are strictly regulated, and the use of a single set of statistical tools is obligatory. While this approach stifles innovation, and shoehorns many searches into confusing and opaque statistical methodology, it does have one advantage: all SUSY results present final results in roughly the same format.

\subsection{Strategy for all ATLAS SUSY searches}

The lowest-level unit of a SUSY analysis is the \emph{region}, which is defined by a series of selection criteria or \emph{cuts} on the full 2012 dataset. These cuts may simply reject questionable or ill-understood data, for example in cases where parts of the detector are offline or otherwise misbehaving, or they may be designed to select a subset of events especially pure in one physical process. The regions are chosen to be statistically disjoint. Several sub-categories of regions exist:
\begin{description}
\item[signal regions] (SRs) are regions where the hypothetical SUSY process is enriched: in a search for particles decaying to dark matter, for example, $\met$ may become large as the event recoils against the invisible particles, and thus the signal region may require large $\met$. In almost all cases, even the harshest cuts are unable to remove all Standard Model backgrounds from the control region. Care must therefore be taken to estimate these backgrounds precisely.
\item[control regions] (CRs) are regions where cuts enrich background processes. The role of control regions is to measure backgrounds as precisely as possible in a region near the signal region.
\item[validation regions] (VRs) are regions near the signal region with similar background composition, but lacking enriched signal. These regions are used to cross-check the background constraints derived from the control regions.
\end{description}

\begin{figure}[h]
  \begin{center}
    \includegraphics[width=0.7\textwidth]{dot/objects.pdf}
    \caption[Analysis Strategy]{Everyone loves a diagram, right?}
  \end{center}
\end{figure}
