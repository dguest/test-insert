\newcommand{\limitbandexplanation}{Observed limits are given by the solid red line. The dotted red lines indicate the maximum range of the observed exclusion when the signal cross section is varied by $\pm 1 \sigma$. Expected limits are given by the dotted blue line. The yellow band surrounding this line indicates the $1 \sigma$ confidence interval when including all sources of uncertainty listed in \cref{sec:systematics} \emph{except} the signal cross section uncertainty.}

%% The fit is implemented by a collection of stacked software packages,
%% standard to the ATLAS SUSY group.
%% At the top level HistFitter~\cite{histfitter} is used.
%% This tool wrap the HistFactory~\cite{histfactory} package, which creates RooFit ``workspaces''.
%% These workspaces are fit, and $\cls$ is extracted, with HistFitter functions (which in turn call RooFit / RooStats).
Systematic variations are categorized as ``experimental'' and ``theoretical''.
The experimental systematics considered are described in \cref{sec:sys_experimental}.

\subsection{Theory Systematic Combination}

Theoretical systematics for the backgrounds are derived as described in \cref{sec:sys_theorybkg}. When running the fit, theory systematics are not directly applied as yield variations in each region. Instead, the systematics are used to derive ``transfer factors'' from the region that best constrains each background to the other regions used in the fit, and the uncertainty for each region is calculated {relative to the constraining region} using these transfer factors. Uncertainties calculated in this way are tabulated in \cref{tab:ttbar_theory_syst,tab:Wjets_theory_syst,tab:Zjets_theory_syst}. These uncertainties are further combined into overall theory systematics (those of the form ``overall down stat'' and ``overall up stat''). There are two important notes about the above procedure:
\begin{enumerate}
\item The theory systematics are calculated relative to the constraining region, meaning that the systematic variation {\em in} the constraining region is, by definition, zero.
\item Hypothetically, any region could be chosen as the constraining region for any background (e.g. we could quote uncertainties in every background relative to \crw). We choose to constrain the top backgrounds ($t + t \bar{t}$) using \crt, \wjets\ using \crw, and \zjets\ using \crz.
\end{enumerate}

Signal ISR and FSR uncertainties are described in \cref{sec:signal-isr-fsr}, and are added to the signal yields in all fits.
In accordance with SUSY group recommendations\footnote{See \url{https://twiki.cern.ch/twiki/bin/viewauth/AtlasProtected/SUSYLimitPlotting}},
signal cross-section uncertainties are shown only in the ``observed''
limits. An uncertainty band on the observed limits is calculated by
running three separate fits for each signal point: a nominal fit with
no cross section systematics applied, and a fit with ``up'' and ``down''
variations added to the expected signal yield.

\section{Exclusions in the $\scharm$--$\neut$ Plane}

Lacking any data excess in the signal regions, we instead set 95\% $\cls$ exclusion limits in the scharm--neutralino mass plane, as shown in \cref{fig:plane_pretty_regions}. Upper limits for cross sections in the $\scharm$--$\neut$ plane are given in \cref{fig:exclusion_plane_ul}.

\begin{figure}
\begin{center}
\includegraphics[width=1.0\textwidth]{int/figures/limit_tree/full_exclusion/exclusion_best.pdf}
\caption[Best signal regions in the $m_{\scharm}$--$m_{\neut}$ plane]{%
Expected and observed exclusion in the $m_{\scharm}$--$m_{\neut}$ plane.
Points with $\cls < 0.05$ are considered excluded.
The three signal regions, with $\mct = \{150, 200, 250\}\text{ GeV}$ are combined with the two $\tilde{t} \to c \neut$ regions (C1 and C2), by taking the region which produces the minimum $\cls$ for each point.}
\label{fig:plane_best_regions}
\end{center}
\end{figure}

\begin{figure}
\begin{center}
\includegraphics[width=1.0\textwidth]{int/figures/limit_tree/full_exclusion/exclusion_overlay.pdf}
\caption[Exclusion in the $m_{\scharm}$--$m_{\neut}$ plane by signal region]{%
Overlay of the expected exclusion limits in the $m_{\scharm}$--$m_{\neut}$ plane, using the individual regions.
}
\label{fig:plane_regions_expected}
\end{center}
\end{figure}

\begin{figure}
\includegraphics[width=\textwidth]{int/figures/limit_tree/full_exclusion/exclusion_pretty.pdf}
\caption[Exclusion in the $m_{\scharm}$--$m_{\neut}$ plane]{
Exclusion in the $m_{\scharm}$--$m_{\neut}$ plane, using the three $\scharm$ fit configurations and the $\tilde{t} \to c \neut$ C1 and C2 configurations. The monojet exclusion is shown for comparison. \limitbandexplanation}
\label{fig:plane_pretty_regions}
\end{figure}

The expected exclusion (based on the fit setup described in
appendix~\ref{app:fitResults}) is shown in
figure~\ref{fig:plane_best_regions}. The three ``$\tilde{c}$'' fit
configurations---those described in this thesis---are considered, each
consisting of the three control regions and one of the three signal
regions defined by the $\mct$ requirement, for a total of four regions
per fit.
In all cases the uncertainties described in \cref{sec:systematics} are accounted for with additional fit parameters.
In addition, we considered two ``$\tilde{t}$'' signal regions (C1 and C2) from the $\tilde{t} \to c\neut$~\cite{stopCharmATLAS} analysis.
While the $\tilde{c}$ signal regions were optimized for high $\Delta m \equiv m_{\scharm} - m_{\neut}$ signal points with two resolvable $c$-jets, these regions loose sensitivity for low $\Delta m$ values, as shown in
\cref{fig:plane_regions_expected}.
The $\tilde{t}$ regions, by contrast, were optimized for ISR boosted $c$-jets, and are sensitive to signal points with low $\Delta m$.

The five fit configurations are merged by taking the ``best'' configuration---that which is expected to exclude the signal with the highest confidence---for each signal point. Contours for the expected and observed exclusion are drawn by interpolating between signal points. Simple linear interpolation of $\cls$, however,  is known to introduce a bias because $\cls$ is relatively non-linear as a function of both $\scharm$ and $\neut$ mass. Following the precedent from previous SUSY searches~\cite{susy-limit-setting}, the $\cls$ values at each signal point are transformed to Gaussian significance $Z$ using the Gaussian inverse cumulative distribution function. Linear interpolation is then used to define $Z(m_{\scharm}, m_{\neut})$ for all points in the plane and draw contours where $Z(m_{\scharm}, m_{\neut}) = Z_{95 \% \text{ CL}} \approx 1.645$.

Exclusions from individual fit configurations are given in \cref{fig:exclusion_plane_ul}, along with 95\% $\cls$ $\scharm$ cross-section upper limits. %% The interpolation in the \scharm--\neut plane is discussed further in \cref{sec:cls-interpolation}, while
Cross-checks of indivial signal points are discussed in \cref{sec:fit-signal-points}.

\newcommand{\planeulcaption}[1]{
\caption[Cross section upper limits in the $m_{\scharm}$--$m_{\neut}$ plane using the $\mct > #1\,\gev$ signal region]{
Cross section upper limits (in fb) in the $m_{\scharm}$--$m_{\neut}$ plane using the $\mct > #1\,\gev$ signal region.
Gray numbers indicate the cross-section upper-limit for each signal point.
}}

\begin{figure}
\includegraphics[width=1.0\textwidth]{int/figures/limit_tree/full_exclusion/upper_limits/mct150.pdf}
\planeulcaption{150}
\label{fig:exclusion_mct150}
\end{figure}

\begin{figure}
\includegraphics[width=1.0\textwidth]{int/figures/limit_tree/full_exclusion/upper_limits/mct200.pdf}
\planeulcaption{200}
\label{fig:exclusion_mct200}
\end{figure}

\begin{figure}
\includegraphics[width=1.0\textwidth]{int/figures/limit_tree/full_exclusion/upper_limits/mct250.pdf}
\planeulcaption{250}
\label{fig:exclusion_mct250}
\end{figure}

\begin{figure}
\includegraphics[width=1.0\textwidth]{int/figures/limit_tree/full_exclusion/upper_limits/scharm_combined.pdf}
\label{fig:exclusion_scharm_combined}
\caption[Cross section upper limits in the $m_{\scharm}$--$m_{\neut}$ combined plane]{
Cross section upper limits (in fb) in the $m_{\scharm}$--$m_{\neut}$ plane, taking the most sensitive region for each point.
Gray numbers indicate the upper limit for each signal point. Note that \cref{fig:plane_pretty_regions} differs from this figure because the former also includes the C1 and C2 regions from the $\sttoc$ search; at the time of writing cross-section upper-limits weren't available for the $\sttoc$ regions. }
\label{fig:exclusion_plane_ul}
\end{figure}

\clearpage

\subsection{Model Independent Upper Limits}
\label{sec:model-independent-ul}

Model independent upper limits on non--Standard Model physics are
derived using a ``discovery'' fit, in which we assume the hypothetical
process only contributes to the signal region yield and leaves the
control regions unaffected. One discovery fit is run for each signal
region. This fit is considerably simpler than the exclusion fit;
whereas the exclusion fit accounts for all systematic variations and
fits the control and signal regions simultaneously, each discovery fit
considers a single signal region with a single combined background
estimate. Expected standard model background yields, observed yields,
and a single overall background uncertainty are taken directly from
\cref{tab:SRyields}. Fit results are converted to upper limits by
solving for $\cls(\mu) = 0.05$, where $\mu$ is the signal
strength. The $\cls(\mu)$ values themselves are calculated with both
asymptotic formulae~\cite{asymptotics} and by running with 3000
pseudo-experiments to derive test statistic distributions.  The result
is a 95\% $\cls$ upper limit on the visible cross section, as summarized
in \cref{tab:model-ind-lim}.

\begin{table}
\centering
\subfigure[asymptotic limits]{
\input{int/tables/model_independent_upper_limits/UpperLimitTable_asimov.tex}
\label{tab:miul-asymtotic}
}
\subfigure[limit with 3000 toys]{
\input{int/tables/model_independent_upper_limits/UpperLimitTable_nToys3000.tex}
\label{tab:miul-toys}
}
\caption[Breakdown of model independent upper limits]{
Model independent upper limits calculated using asymtotic approximations~\subref{tab:miul-asymtotic} and with 3000 pseudo-experiments~\subref{tab:miul-toys}. Left to right: 95\% CL upper limits on the visible cross section
($\langle\epsilon\sigma\rangle_{\rm obs}^{95}$) and on the number of
signal events ($S_{\rm obs}^{95}$ ).  The third column
($S_{\rm exp}^{95}$) shows the 95\% CL upper limit on the number of
signal events, given the expected number (and $\pm 1\sigma$
excursions on the expectation) of background events.
The last two columns
indicate the $CL_B$ value, i.e. the confidence level observed for
the background-only hypothesis, and the discovery $p$-value ($p(s = 0)$).}
\label{tab:model-ind-lim}
\end{table}



%% \cite{histfitter}.
