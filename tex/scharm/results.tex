As discussed in \cref{sec:analysis-strategy}, backgrounds in the
signal regions are estimated with a profile-likelihood fit
\cite{histfactory}, which attempts to normalize the number of signal and background events such that:
\begin{enumerate}
\item the number of simulated events in the control and signal regions
  matches with data, and
\item the parameters normalizing each background and signal sample are
  consistent across all regions.
\end{enumerate}
Several types of floating parameters are used in the fit:
\begin{description}
\item[free parameters:] these constrain the
  overall normalization of the simulated $\wjets$,
  $\zjets$, $\ttbar$ + single $t$, and
  signal. Such parameters are initalized with a nominal value of 1,
  but the fit pays no penalty for values that deviate from the initial
  value.
\item[constrained parameters:] these
  parameters describe systematic variations. A gaussian restraint
  penalizes the fit when these values shift from the nominal
  value. The width of this restraint is derived from dedicated
  studies as discussed below.
\end{description}

The fit is implemented by a collection of stacked software packages,
standard to the ATLAS SUSY group. At the top level
HistFitter~\cite{histfitter} is used. This tool wrap the
HistFactory~\cite{histfactory} package, which creates RooFit
``workspaces''. These workspaces are fit, and $\cls$ is extracted, with
HistFitter functions (which in turn call RooFit /
RooStats). Systematic variations are categorized as ``experimental''
and ``theoretical''. The experimental systematics considered are described in \cref{sec:sys_experimental}.


%% \cite{histfitter}.
