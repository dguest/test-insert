The first dedicated search for scalar charm quarks decaying to charm jets and neutrilinos at the LHC is presented in this thesis.
This search
%% , which considers all data from the 2012 $\cmenergy$ dataset,
yields no evidence for SUSY.
The lack of signal is interpenetrated in the $\scharm$--$\neut$ plane in terms of a $95\%\,\cls$ exclusion and extends the previous limits on second-generation squark masses---held by a generic search for zero leptons and 2--6 jets~\cite{atlas-inclusive}---by roughly $100\,\gev$.

These limits are likely to increase again in the near future. At the initial run 2 energy of $13\,\tev$, the cross-section for pair-production of $700\,\gev$ scalar charm quarks is expected to increase from $0.0081$ to $0.067\,\mathrm{pb}$~\cite{susy-expected-run2}.
Coupled with roughly 10 times the instantaneous luminosity, this higher cross-section will increase the expected scalar charm production rate by roughly two orders of magnitude. The dominant $\wjets$ and $\zjets$ backgrounds will also increase, of course, but to a lesser extent.


Beyond providing the most stringent limits to date on second-generation squark production, this search also demonstrates the power of charm-tagging.
The limits described in \cref{sec:exclusions} rely on boosted and nonboosted signal regions, both of which are designed around a $c$-tagged event selection.
Despite relative immaturity and large systematic uncertainties stemming from poor modeling, %and calibrations appropriated from $b$-tagging
$c$-tagging is able to extend the sensitivity in both signal regions considerably.

Still, charm tagging remains a work in progress, with a long list of potential improvements to implement in the near future.
%% In the near future $c$-taggers can be improved in a number of ways.
The rather crude JetFitter retuning that resulted in JetFitterCharm could and should be revisited; tweaking the vertex-merging probabilities and track selection could potentially improve both $b$ and $c$~tagging efficiency.
On the modeling front, an ongoing effort to integrate \textsc{EvtGen}~\cite{evtgen} with \atlas\ should mitigate some systematic uncertainties while improve the simulation-based training of future $c$-taggers.
Meanwhile the $D^{*}$ $c$-jet calibration could be cross-checked against both the $W + c$~\cite{wcharm-cal} and soft-$\mu$ based calibrations.

Even if SUSY is never found, charm tagging will remain useful to measure Higgs couplings
Generalizing beyond $c$-tagging, the algorithms presented in \cref{sec:tag-future} have clear applications in $b$-tagging, especially in $b$-vs-$c$ discrimination.
In the short term improved $b$-tagging will be important for $H \to b \bar{b}$ and $ttH$ searches.
%% Improved $b$-tagging

%% The first generation of $c$-taggers were calibrated with the baseline $b$-tagging algorithms, so future taggers may also benefit from additional $c$-jet specific , additional calibration, and improved tagging algorithms (see ) may improve $c$-tagging further. 

\begin{cfig}
  \graphic[0.8]{int/figures/limit_tree/full_exclusion/exclusion_inclusive.pdf}
  \caption[Charm-tagged limits in the $\scharm$--$\neut$ plane, compared to untagged analysis]{Limits from the $\sctoc$ search in the $\scharm$--$\neut$ plane, compared to an ``inclusive'' search for second generation squarks.}
\end{cfig}

As the final analyses of run 1 are finalized and published, this picture is altogether too common; of the handful of light superpartners predicted by natural SUSY, none have been observed.

cross-section for $\tilde{b}$ goes from 0.0081 to 0.067 pb with 8 to 13 $\tev$~\cite{susy-expected-run2}

say something about~\cite{flavored-naturalness}
