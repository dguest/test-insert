The first dedicated search for scalar charm quarks decaying to charm jets and neutrilinos at the LHC is presented in this thesis.
This search
%% , which considers all data from the 2012 $\cmenergy$ dataset,
yields no evidence for SUSY or physics beyond the standard model.
The lack of signal is interpenetrated in the $\scharm$--$\neut$ plane in terms of a $95\%\,\cls$ exclusion and extends the previous limits on second-generation squark masses---held by a generic search for zero leptons and 2--6 jets~\cite{atlas-inclusive}---by roughly $100\,\gev$.

These limits are likely to increase again in the near future. At the initial run 2 energy of $13\,\tev$, the cross-section for pair-production of $700\,\gev$ scalar charm quarks is expected to increase from $0.0081$ to $0.067\,\mathrm{pb}$~\cite{susy-expected-run2}.
Coupled with roughly 10 times the instantaneous luminosity, this higher cross-section will increase the expected scalar charm production rate by roughly two orders of magnitude. While the dominant $\wjets$, $\zjets$ and $\ttbar$ backgrounds will also increase to a lesser extent, the $\sctoc$ limits are significantly limited by statistics---even if signal and background cross-sections increased equally, this search would gain from additional data.

Beyond providing the most stringent limits to date on second-generation squark production, this search also demonstrates the power of charm-tagging.
The limits described in \cref{sec:exclusions} rely on boosted and nonboosted signal regions, both of which are designed around a $c$-tagged event selection.
Despite relative immaturity and large systematic uncertainties stemming from poor modeling and limited calibration statistics, %and calibrations appropriated from $b$-tagging
$c$-tagging is able to extend the sensitivity in both signal regions considerably.

The rough-cut nature of run 1 $c$-tagging makes it especially interesting for run 2, where many potential improvements await.
%% In the near future $c$-taggers can be improved in a number of ways.
Improvements could begin with the upstream algorithms; the rather crude JetFitter retuning that resulted in JetFitterCharm could and should be revisited. Tweaking the vertex-merging probabilities and track selection could potentially improve both $b$ and $c$~tagging efficiency.
On the modeling front, an ongoing effort to integrate \textsc{EvtGen}~\cite{evtgen} with \atlas\ simulation should mitigate some systematic uncertainties while improving the simulation-based training of future $c$-taggers.
A potentially major hardware upgrade, a new insertable barrel pixel layer~\cite{IBLTDR} may improve vertex resolution early in run 2.
Meanwhile the $D^{*}$ $c$-jet calibration could be cross-checked against both the $W + c$~\cite{wcharm-cal} and soft-$\mu$ based calibrations.

The success of $c$-tagging in run 1 also provides a proof-of-principle on which other measurements are justified.
Within SUSY, the possibility of $\tilde{t}$--$\tilde{c}$ predicts decay signatures of the form $t \tilde{c} (\tilde{t} c) + \met $~\cite{flavored-naturalness,squark-mixing}, which could benefit from $c$-tagging.
In the Higgs sector, the $H \to c \bar{c}$ coupling could provide a general probe into new physics~\cite{charminghiggs} or at the very least an additional confirmation of the SM Higgs mechanism.
Beyond $c$-tagging, the algorithms presented in \cref{sec:tag-future} have clear applications in $b$-tagging, especially in $b$-vs-$c$ discrimination.
%% In the short term improved $b$-tagging will be important for $H \to b \bar{b}$ and $ttH$ searches.

In contrast to the bright future of $c$-tagging, the prospects for SUSY and especially the MSSM are rather glum.
%% As the final analyses of run 1 are finalized and published, this picture is altogether too common;
Of the handful of light superpartners predicted by natural SUSY, none have been observed.
Most damningly, many natural predictions place several new particles dead-center in LHC exclusions.
%% At the onset of run 2, many critics argue that SUSY has failed.
Damning evidence aside, critics and apologists agree on one point: SUSY never really dies, it just gets heavier.
Even if the theory has lost its crowning jewel---the naturalness it supposedly brings to the standard model---SUSY or something like it could still appear in run 2. The LHC will collect data for the next decade, with or without SUSY; we have no excuse to do anything but go on searching.
%% It's a well-known fact that SUSY can never die

%% Improved $b$-tagging

%% The first generation of $c$-taggers were calibrated with the baseline $b$-tagging algorithms, so future taggers may also benefit from additional $c$-jet specific , additional calibration, and improved tagging algorithms (see ) may improve $c$-tagging further. 

\begin{cfig}
  \graphic[0.8]{int/figures/limit_tree/full_exclusion/exclusion_inclusive.pdf}
  \caption[Charm-tagged limits in the $\scharm$--$\neut$ plane, compared to untagged analysis]{Limits from the $\sctoc$ search in the $\scharm$--$\neut$ plane, compared to an ``inclusive'' search for second generation squarks.}
\end{cfig}

