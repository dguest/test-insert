\chapter{The \atlas\ Experiment}
\section{The LHC}
\chapquote{\ldots and boom she goes!}{Day}
\label{sec:lhc}

The Large Hadron Collider (LHC) is a 26.7 km long circular particle accelerator which sits roughly 100 m below ground outside Geneva Switzerland.
Throughout the ring, 1,232 dipole magnets, each $16.5\,\text{m}$ long, bend the two counter-rotating proton beams into a circular path, while numerous quadrupole, sextupole, and octupoles focus the beams. Radio frequency cavities accelerate the protons from an initial injection energy of $450\,\gev$ to a collision energy of up to $14\,\tev$.~\cite{lhc-machine}
%% The beams collide at four interaction points within underground detector halls.



%% At design energy, two counter-rotating beams of protons collide at 4 defined interaction points with a proton-proton center of mass energy of $\sqrt{s} = 14\,\text{TeV}$.

\subsection{The Injection Chain}
Protons in the LHC begin as hydorgen gas which is ionized by a duoplasmatron proton ion source and injected into Linac 2. These protons leave Linac 2 with an energy of $50\,\mev$ and enter the Proton Synchrotron Booster (PSB).
Every 1.2 seconds, the PSB accelerates the protons to $1.4\,\gev$ and injects them into the Proton Synchrotron (PS).~\cite{lhc-machine} Following two fills from the PSB, the PS accelerates the protons to $25\,\gev$ in roughly 1.2 seconds, and injects them into the Super Proton Synchrotron (SPS). The entire PS cycle takes roughly 3.5 seconds~\cite{ps-thesis}.

The SPS is the final injection stage before the LHC. After four fills from the PS, the SPS accelerates protons to 450 $\gev$ over 4.3 seconds.~\cite{ramp-time}
The LHC is filled in 12 batches from the SPS, each between 5.8 and 7.9 $\mu$s long, with 11 0.94 $\mu$s gaps to allow for the injection kicker magnets to engage. A final 3.0 $\mu$s ``abort'' gap completes the ring, long enough to allow the beam dump kickers to activate.


\subsubsection{Power Use}
The LHC uses about 4\% of the energy of Geneva~\cite{lhc-energy}. Half is on cryogenics.
\begin{itemize}
\item LHC cryogenics 27.5 MW
\item LHC experiments 22 MW
\item CERN 180 MW
\item Geneva 1.3 GW
\end{itemize}
\begin{figure}
  \subfigure[Total Luminosity Delivered to ATLAS]{
    \graphic{0.5}{misc/cc/atlas-lumi.pdf} \label{fig:atlas-lumi}}
  \subfigure[Number of Interactions per Bunch Crossing]{
    \graphic{0.5}{misc/cc/int-per-bx.pdf} \label{fig:int-per-bx}}
  \caption[LHC Luminosity Delivered to ATLAS]{%
    LHC luminosity delivered to ATLAS.}
\end{figure}
%% While currently the world's most powerful and expensive particle collider, the LHC was designed to use existing infrastructure where possible to mitigate costs.
%% The vast majority of the machine was designed to fit in the tunnel bored for the Large Electron-Positron collider in the 1980s.


\section{The \atlas\ Detector}
\label{sec:atlas}
\begin{itemize}
\item Maybe put \cref{sec:pheno} here?
\item Mention how systematics work here (mention how they can be correlated between regions)
\end{itemize}
\subsection{Trigger}
\label{sec:trigger}
Brief summary of the trigger.
\begin{itemize}
\item Must introduce ``offline'' and ``online'' (referenced later)
\item Must discuss prescaling
\end{itemize}
\subsection{Data Acquisition and Processing}
\begin{itemize}
\item Concept of a run (referenced later)
\end{itemize}
\section{\atlas\ Software and Simulation}
\label{sec:software}
\begin{itemize}
\item \atlas\ geometry model
\item mention how simulation is sometimes weighted
\item concept of a signal grid (and signal ``point'')
\end{itemize}
