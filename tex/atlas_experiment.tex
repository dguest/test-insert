\section{The \atlas\ Detector}

\atlas\ is one of two ``general purpose'' detectors at the LHC focusing on proton collisions.
The detector is composed of several nested sub-detectors which provide complementary information when reconstructing an event.
An inner-detector includes both silicon and drift-tube based trackers which measure the paths of charged particles as they move outward from the interaction point.
Surrounding the tracker is a superconducting solenoid which creates a $2\,\text{T}$ axial magnetic field.
By measuring the curvature of charged particles within this field, ATLAS is able to establish the particle momentum and charge.
Outside the solenoid is the calorimeter, which measures the energy of both charged and strongly interacting neutral particles.
%% The calorimeter is designed to measure all particles out to $|\eta| < 4.9$.
Only two known particles normally penetrate the calorimeter: muons and neutrinos.
Of these, muon momentum is measured by the muon spectrometer, which surrounds the calorimeter.
Neutrinos are inferred by an overall momentum imbalance transverse to the beam axis after summing the momenta of all other particles.

\subsection{Inner Detector}

\begin{cfig}
  \graphic{1.0}{misc/cc/atlas-id-acceptance.pdf}
  \caption[ATLAS inner detector geometry]{Geometry of the ATLAS inner detector.}
  \label{fig:atlas-id-acceptance}
\end{cfig}

%% While currently the world's most powerful and expensive particle collider, the LHC was designed to use existing infrastructure where possible to mitigate costs.
%% The vast majority of the machine was designed to fit in the tunnel bored for the Large Electron-Positron collider in the 1980s.

\label{sec:atlas}
\begin{itemize}
\item Maybe put \cref{sec:pheno} here?
\item Mention how systematics work here (mention how they can be correlated between regions)
\end{itemize}
\subsection{Trigger}
\label{sec:trigger}
Brief summary of the trigger.
\begin{itemize}
\item Must introduce ``offline'' and ``online'' (referenced later)
\item Cite Will Butt~\cite{atlas-lvl1}.
\item Must discuss prescaling
\end{itemize}
\subsection{Data Acquisition and Processing}
\begin{itemize}
\item Concept of a run (referenced later)
\end{itemize}
\subsection{\atlas\ Software and Simulation}
\label{sec:software}
\begin{itemize}
\item \atlas\ geometry model
\item mention how simulation is sometimes weighted
\item concept of a signal grid (and signal ``point'')
\end{itemize}
