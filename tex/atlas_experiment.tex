\chapter{The Experiment}
\chapquote{\ldots and boom she goes!}{Day}
\label{sec:lhc}

The Large Hadron Collider (LHC) is a 26.7 km long circular synchrotron which sits roughly 100 m below ground outside Geneva Switzerland.
As the world's most powerful particle accelerator, it marks the epicenter of several worldwide collaborations involving thousands of physicists on the bleeding-edge of particle physics.
The accelerator itself contains two proton beams, which cross at four interaction points where the protons collide head-on.
The products of these collisions are measured by four experiments: ALICE is optimized for collisions between lead nuclei; LHCb is specifically designed to measure $b$-hadron decays; ATLAS and CMS are designed to detect and measure a wide range of particles in and beyond the standard model.
%% The beams collide at four interaction points within underground detector halls.

\section{CERN}
In the years immediately following World War 2, European leaders recognized the potential benefits of a common nuclear research organization.
Beyond the obvious contribution to rebuilding the continent's devastated scientific infrastructure, such an international collaboration would shunt the scientific and engineering fervor of the previous decades---the fervor which had helped destroy Europe so thoroughly---into more peaceful and diplomatic pursuits.
The first international resolution toward this goal was adopted in December 1951. Within two months 11 countries had established the Conseil Européen pour la Recherche Nucléaire (CERN). By September of 1954, construction had begun in Meyrin Switzerland, and CERN was officially renamed the European Organization for Nuclear Research.
While no longer an acronym, the name CERN  still officially applies~\cite{cern-timeline}.

Over the next half-century increasingly powerful accelerators were constructed at CERN. By the early 1980's the Super Proton Synchrotron (SPS) was colliding protons with anti-protons, with a center of mass energy of $540\,\gev$, which lead to the discovery of the $W$ and $Z$ bosons~\cite{ua1w,ua2w,ua1z}.
In the late 80's a huge 27 kilometer circular tunnel was bored through the sedimentary rock beneath the Franco-Swiss countryside to house the Large Electron-Positron Collider (LEP), which remains the most powerful lepton collider of all time. In 2000, after 10 years of measuring the standard model with unprecedented (and still unsurpassed, in many cases) precision, LEP was decommissioned to make room for the LHC in the same tunnel~\cite{lep-summary}.
\begin{cfig}
\end{cfig}

%% \begin{cfig}
%%   \graphic{0.49}{misc/cc/cern-map.jpg}
%%   \graphic{0.49}{misc/cc/cern-from-air.jpg}
%%   \caption[CERN From Above]{(left) Map of CERN, the LHC, and surrounding political boundaries. Taken from Ref~\cite{cern-map}. (right) View of CERN from a northwestern vantage point. Point 1 is visible on the upper right, while point 4 is on the lower left. Taken from Ref~\cite{cern-from-air}.}
%%   \label{fig:cern-above}
%% \end{cfig}
\begin{cfig}
  \graphic{0.8}{misc/cc/cern-from-air.jpg}
  \caption[LHC From The Air]{View of the LHC from a northwestern vantage point. Point 1 is visible on the upper right, while point 4 is on the lower left. Taken from Ref~\cite{cern-from-air}.}
  \label{fig:cern-from-air}
\end{cfig}

Construction of the LHC continued throughout the 2000's. While the LHC was undergoing commissioning in 2008, an electrical fault caused an explosion which delayed collisions by roughly one year~\cite{lhc-incident}. Repairs were swift, however, and the first successful proton-proton collisions took place in November of 2009.

%% At design energy, two counter-rotating beams of protons collide at 4 defined interaction points with a proton-proton center of mass energy of $\sqrt{s} = 14\,\text{TeV}$.

\section{LHC Injection Chain}
Protons in the LHC begin as dilute hydrogen gas. In the first stage of acceleration, the gas is fed into duoplasmatron proton ion source and injected into Linac 2. These protons leave Linac 2 with an energy of $50\,\mev$ and enter 4 parallel rings in the Proton Synchrotron Booster (PSB), the first of many injection synchrotrons leading to the LHC.
The PSB clumps protons into the \emph{bunches} needed for coherent acceleration, and accelerates them to $1.4\,\gev$.
Every 1.2 seconds, the PSB injects these bunches into the Proton Synchrotron (PS)~\cite{lhc-machine}. Following two fills from the PSB, the PS temporarily ceases to accept protons and instead accelerates its current fill to $25\,\gev$ before injecting them into the Super Proton Synchrotron (SPS). The entire PS cycle takes roughly 3.6 seconds~\cite{ps-thesis}.

\begin{cfig}
  \graphic{1.0}{misc/cc/accelerator-complex.jpg}
  \caption[The LHC Accelerator Complex]{The Large Hadron Collider accelerator complex, including the injection chain. Taken from Ref~\cite{accelerator-complex}.}
  \label{fig:lhc-accelerator-complex}
\end{cfig}

The SPS is the final injection stage before the LHC. After four fills from the PS, the SPS accelerates protons to 450 $\gev$ in 4.3 seconds~\cite{ramp-time}. Including the time spent waiting for fills from the PS, each SPS cycle takes 21.6 seconds.
Both counter-rotating LHC beams consist of 12 fills from the SPS, resulting in an LHC fill time of roughly 4 minutes per beam.
Allowing for various calibrations (including the circulation of ``pilot bunches'') the minimum time to fill the LHC is 16 minutes.
Bunch trains within the LHC are carefully spaces to allow injection kicker magnets time to engage.
Each of the 12 SPS fills passes within 5.8 to 7.9 $\mu$s, separated by 11 0.94 $\mu$s gaps. A final 3.0 $\mu$s abort gap completes the circular bunch train, long enough to allow the beam dump kickers to activate.

Once filled, the LHC slowly ramps up to a maximum energy of $7\,\tev$ per beam in 20 minutes and commences collisions. These collisions are the dominant cause of beam loss during normal operation; neglecting all other loss channels the beam lifetime is approximately 29 hours, when other loss channels are accounted for this lifetime drops to roughly 15 hours~\cite{lhc-machine}.
Although the length of a given run varies, the beams are generally dumped and refreshed within 12 hours. In 2012 the mean length of a run was 6.1 hours. Of roughly 300 runs, approximately 16 continued for more than 15 hours, with one run dumping after 23 hours~\cite{lhc-run1}.

In the long periods between LHC fills, the injection chain can be put to other uses.
A number of fixed-target experiments occupy the ``north area'' in Pr\'evessin or detector halls within the Meyrin site, as showin in \cref{fig:lhc-accelerator-complex}.
In general, when the more powerful of two sequential accelerators in the injection chain is ramping up in energy, the less powerful accelerator is left with spare cycles which can be diverted to lower-energy experiments.

\section{The Large Hadron Collider}
Throughout the ring of the LHC, 1,232 dipole magnets, each $16.5\,\text{m}$ long, bend the two counter-rotating proton beams into a circular path, while radio frequency cavities accelerate the protons from an initial injection energy of $450\,\gev$ to a collision energy of up to $14\,\tev$.
The dipole magnets rely on the exceptionally intense magnetic fields, above $8\,\mathrm{T}$, to contain the beams.
Generating these fields in an electromagnet requires extremely high current, which is only possible with superconducting NbTi coils.
Since NbTi is only superconducting below $\sim 10\,\text{K}$, the entire 26.7 km of the LHC is bathed in $2\,\text{K}$ liquid helium.
~\cite{lhc-machine}

\begin{cfig}
  \graphic{0.8}{misc/cc/cern-map.jpg}
  \caption[CERN Map]{Map of CERN, the LHC, and surrounding political boundaries. Taken from Ref~\cite{cern-map}.}
  \label{fig:cern-map}
\end{cfig}


\subsection{Power Use}
The LHC uses about 4\% of the energy of Geneva~\cite{lhc-energy}. Half is on cryogenics.
\begin{itemize}
\item LHC cryogenics 27.5 MW
\item LHC experiments 22 MW
\item CERN 180 MW
\item Geneva 1.3 GW
\end{itemize}
\begin{figure}
  \subfigure[Total Luminosity Delivered to ATLAS]{
    \graphic{0.5}{misc/cc/atlas-lumi.pdf} \label{fig:atlas-lumi}}
  \subfigure[Number of Interactions per Bunch Crossing]{
    \graphic{0.5}{misc/cc/int-per-bx.pdf} \label{fig:int-per-bx}}
  \caption[LHC Luminosity Delivered to ATLAS]{%
    LHC luminosity delivered to ATLAS.}
\end{figure}
\begin{itemize}
  \item Quite beam current (582 mA, p47 of~\cite{lhc-machine})
\end{itemize}
%% While currently the world's most powerful and expensive particle collider, the LHC was designed to use existing infrastructure where possible to mitigate costs.
%% The vast majority of the machine was designed to fit in the tunnel bored for the Large Electron-Positron collider in the 1980s.


\section{The \atlas\ Detector}
\label{sec:atlas}
\begin{itemize}
\item Maybe put \cref{sec:pheno} here?
\item Mention how systematics work here (mention how they can be correlated between regions)
\end{itemize}
\subsection{Trigger}
\label{sec:trigger}
Brief summary of the trigger.
\begin{itemize}
\item Must introduce ``offline'' and ``online'' (referenced later)
\item Must discuss prescaling
\end{itemize}
\subsection{Data Acquisition and Processing}
\begin{itemize}
\item Concept of a run (referenced later)
\end{itemize}
\subsection{\atlas\ Software and Simulation}
\label{sec:software}
\begin{itemize}
\item \atlas\ geometry model
\item mention how simulation is sometimes weighted
\item concept of a signal grid (and signal ``point'')
\end{itemize}
