\section{The \atlas\ Detector}
\label{sec:atlas}

\atlas\ is one of two ``general purpose'' detectors at the LHC focusing on proton collisions.
The detector is composed of several nested sub-detectors which provide complementary information when reconstructing an event.
In general, \atlas\ is approximately symmetric about the beam axis, and also across the $z = 0$ plane.
An inner-detector includes both silicon and drift-tube based trackers which measure the paths of charged particles as they move outward from the interaction point.
Surrounding the tracker is a superconducting solenoid which creates a $2\,\text{T}$ axial magnetic field.
By measuring the curvature of charged particles within this field, ATLAS is able to establish the particle momentum and charge.
Outside the solenoid is the calorimeter, which measures the energy of both charged and strongly interacting neutral particles.
%% The calorimeter is designed to measure all particles out to $|\eta| < 4.9$.
Only two known particles normally penetrate the calorimeter: muons and neutrinos.
Of these, muon momentum is measured by the muon spectrometer, which surrounds the calorimeter.
Neutrinos are inferred by an overall momentum imbalance transverse to the beam axis after summing the momenta of all other particles.

\subsection{Tracker}

The \atlas\ tracker~\cite{atlas-id1,atlas-id2} consists of several inner pixel layers, surrounded by a silicon strip detector, surrounded by a drift-tube tracker which doubles as a transition radiation detector.
Particles within the tracker volume are bent into helical paths by the axial magnetic field and reconstructed as \emph{tracks}.

Like all \atlas\ subdetectors the tracker covers as much solid angle as possible surrounding the interaction point: the full range of $\phi$ and out to $|\eta| < 2.5$.
One quadrant of the inner detector is shown in \cref{fig:atlas-id-acceptance}.
Sensors are arranged in layers and angled to maximize coverage in $\eta$; \emph{barrel} layers at low $\eta$ are aligned with the beam axis, while \emph{wheel} layers at higher $\eta$ extend radially out from the beam.
Out to the maximum coverage in $\eta$ each particle is expected to cross at least 7 silicon sensors and 35--40 drift-tubes~\cite{DetPap,atlas-trt2}.
Energy deposits within these layers serve both as seeds and 3-dimensional constraints for reconstructed tracks.

\begin{cfig}
  \graphic{1.0}{misc/cc/atlas-id-acceptance.pdf}
  \caption[ATLAS inner detector geometry]{Quarter section view of the ATLAS inner detector.}
  \label{fig:atlas-id-acceptance}
\end{cfig}


\paragraph{The Pixel Detector} The \atlas\ pixel detector consists of three barrel layers and three wheels on each side of the $z = 0$ plane.
The barrel layers are positioned 50, 89, and 123$\,\mathrm{mm}$ from the beam and extend out to $|z| < 400\,\mathrm{mm}$.
Beyond the extent of the barrel layers, the six wheels are positioned at $|z| = 495$, $580$, and $650\,\mathrm{mm}$, and extend from $R = 89$ to $150\,\mathrm{mm}$.
These layers are composed of a total of 1744 $19 \times 63\,\mathrm{mm}$ identical modules.
Each module contains 46,080 active pixels, 90\% of which measure $50 \times 50\,\mathrm{\mu m}$, and the rest of which measure $50 \times 600\,\mathrm{\mu m}$.
The modules are oriented such that the smaller pixel dimension points along $\hat{\phi}$~\cite{DetPap}.

\paragraph{The Semiconductor Tracker}
The Semiconductor Tracker (SCT) occupies the space outside the pixel detector, where charged tracks from collisions are slightly less numerous and occupancies (the fraction of events where a given element registers a hit) are lower.
It consists of 4 barrel layers which span $299 < R < 514\,\mathrm{mm}$, $|z| < 749\,\mathrm{mm}$, and 9 wheels which range in $|z|$ over 854--2720$\,\mathrm{mm}$ and span $275 < R < 560\,\mathrm{mm}$.
The SCT consists of 2112 barrel modules and $2 \times 1976$ modules in the two wheel sections, for a total of 4088 modules.
Each module consists of several strip sensors with 768 active strips each.
The width of these strips varies between modules: barrel strips are each $80\,\mathrm{\mu m}$ wide; wheel strips are trapezoidal such that strip edges fan out radially from the beam line, but are roughly 57--90$\,\mathrm{\mu m}$ in width.
Each sensor contains 768 active strips.
Pairs of overlapping strip sensors are mounted back-to-back on modules with an offset of $\pm 20\,\mathrm{mrad}$ to give a stereo measurement of $z$ (in the case of the barrel) or $R$~\cite{DetPap}.

\paragraph{The Transition Radiation Tracker}
The Transition Radiation Tracker~\cite{atlas-trt,atlas-trt2} (TRT) is a combined drift-tube tracker and transition radiation detector.
The basic elements of the trackers are $4\,\mathrm{mm}$ diameter polyimide (Kapton) straws, coated with aluminum to increase conductivity and reinforced with carbon fiber.
Through the center of each straw runs a $31\,\mathrm{\mu m}$ diameter gold-plated tungsten wire.
This wire acts as an anode and is kept grounded, while the straw cathode is held at $-1530\,\mathrm{V}$.
The gas within the straws is constantly circulated and consists of 70\% Xenon, 27\% CO$_2$, and 3\% O$_2$, held at 5--10$\,\mathrm{mbar}$ over atmospheric pressure.

Charged particles crossing the straws liberate electrons which drift toward the anode at roughly $0.05\,\mathrm{mm/ns}$, creating an avalanche of charge along the way.
The typical gain of the charge avalanche is $2.5 \times 10^4$.
When this avalanche reaches the anode the charges are transmitted down the wires and digitized as \emph{low threshold} or \emph{high threshold} hits.
The leading edge of a low threshold hit defines a \emph{drift circle}, which constrains the distance between the straw wire and track within roughly $130\,\mathrm{\mu m}$.

\begin{cfig}
  \graphic{0.6}{misc/cc/trt-high-threshold.pdf}
  \caption[TRT high threshold probability for $e$, $\mu$, and pions]{
    Probability of a high-threshold hit in the TRT barrel section as a function of the $\gamma$ for $e$ (open squares), $\mu$ (full triangles) and pions (open circles) in the energy range 2--350$\,\gev$, as measured in the combined test-beam.}
  \label{fig:trt-high-threshold}
\end{cfig}

In addition to low threshold hits, charged particles in the TRT produce \emph{transition radiation} in proportion to their Lorentz $\gamma$ factor.
The gas mixture within the TRT is designed to absorb transition radiation photons, which create much larger electron cascades and result in high threshold hits.
As shown in \cref{fig:trt-high-threshold}, these high threshold hits offer an additional discriminant between electrons and pions, since pions are much more massive and thus have a lower $\gamma$ for the same momentum.

\paragraph{Front-end Readouts} Each of the three inner detector components includes a memory buffer and receives a clock signal synchronized with the LHC clock.
Charged particles leave hits, which are recorded in the buffer along with a time-stamp indicating the associated bunch crossing.
The buffers store hits for approximately $3.2\,\mathrm{\mu s}$, long enough for the level 1 trigger decision (see \cref{sec:trigger}).
%% Hits are only read off the front-end boards only when a level 1 trigger fires.


%% While currently the world's most powerful and expensive particle collider, the LHC was designed to use existing infrastructure where possible to mitigate costs.
%% The vast majority of the machine was designed to fit in the tunnel bored for the Large Electron-Positron collider in the 1980s.

\subsection{Calorimeters}
Outside the \atlas\ tracker two concentric calorimeters measure particle energy out to $|\eta| < 4.9$.
The inner (electromagnetic or EM) calorimeter is optimized to measure the energy of electrically charged particles and photons, while the outer (hadronic) calorimeter is optimized for neutral strongly-interacting particles.
A secondary purpose of the calorimeters is to absorb particles as they move outward from the interaction point and prevent them from passing into the next layer of \atlas.
The EM calorimeter uses liquid argon as an active medium and relies on metal absorber plates to create particle showers from incoming particles.
The barrel hadronic calorimeter uses steel plates as absorbers and clear plastic scintillating tiles as an active medium.
Other sections of the hadronic calorimeter (endcap and forward) use liquid argon and various metal absorbers.

\paragraph{The Electromagnetic Calorimeter}



\begin{cfig}
  \subfigure[Barrel and endcap calorimeter]{
    \graphic{0.48}{misc/cc/atlas-calo-quarter.pdf}
    \label{fig:atlas-calo-quarter}}
  \subfigure[Forward calorimeter]{
    \graphic{0.48}{misc/cc/atlas-fcal-quarter.pdf}
    \label{fig:atlas-fcal-quarter}}
  \caption[Atlas calorimeter quarter-view]{Schematic quarter-view of the \atlas\ calorimeters. The barrel and endcap calorimeters, shown in~\subref{fig:atlas-calo-quarter}, use liquid argon and plastic scintillating tiles, respectively. The forward calorimeter~\subref{fig:atlas-fcal-quarter} uses liquid argon exclusively.}
  \label{fig:atlas-calos-quarter}
\end{cfig}

\begin{itemize}
\item Maybe put \cref{sec:pheno} here?
\item Mention how systematics work here (mention how they can be correlated between regions)
\end{itemize}
\subsection{Trigger}
\label{sec:trigger}
Brief summary of the trigger.
\begin{itemize}
\item Must introduce ``offline'' and ``online'' (referenced later)
\item Cite Will Butt~\cite{atlas-lvl1}.
\item Must discuss prescaling
\end{itemize}
\subsection{Data Acquisition and Processing}
%% \begin{itemize}
%% %% \item Concept of a run (referenced later)
%% \end{itemize}
\subsection{\atlas\ Software and Simulation}
\label{sec:software}
\begin{itemize}
\item \atlas\ geometry model
\item mention how simulation is sometimes weighted
\item concept of a signal grid (and signal ``point'')
\end{itemize}
