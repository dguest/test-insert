\section{Collider Physics}
\label{sec:pheno}
\begin{itemize}
\item maybe say something about how SUSY feynman diagrams work.
\end{itemize}
\subsection{Proton Collisions}
For the purposes of this thesis, the details of the LHC \cref{sec:lhc} 
\begin{figure}
  \includegraphics[width=\textwidth]{misc/cc/atlas-sm.pdf}
  \caption[Summary of measured SM cross-sections in ATLAS]{%
Summary of measured SM cross-sections in ATLAS. Taken from ref~\cite{atlas-sm}.
}
\begin{itemize}
\item explain ISR
\item explain cross-sections
\end{itemize}
\subsection{Hadrons, Electrons, and Muons}
\begin{itemize}
\item must mention isolation (referenced later)
\item Lifetimes and time dilation
\item electromagnetic and strong interactions in detectors
\item Magnitude of cross-sections: EW vs strong
\end{itemize}
\subsection{Neutrinos (and Neutrilinos)}
the known unknowns
\begin{itemize}
\item introduce $\met$, $\vmet$
\item reference detector chapter
\end{itemize}
\subsection{Taus, and Heavy Quarks}
\begin{itemize}
\item more on lifetimes
\item tracking
\item vertex finding
\end{itemize}
\subsection{W, Z, Higgs, and Everything Else}
\begin{itemize}
\item $Z$ peak reconstruction
\item $\mt$, $\mct$, etc
\item Feynman diagrams for $\wjets$, $\zjets$, $\ttbar$
\item Feynman diagrams for SUSY pair-produced quarks
\end{itemize}
\subsection{Implications for Experiments}
\label{sec:det-design}
\end{figure}
\begin{itemize}
\item define ``the hard scatter'', explain why it's appropriate to use a definite article.
\item explain implications for triggering
\end{itemize}
