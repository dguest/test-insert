\section{Collider Physics}
\label{sec:pheno}
%% \begin{itemize}
%% \item maybe say something about how SUSY feynman diagrams work.
%% \end{itemize}
\subsection{Proton Collisions}
The LHC (discussed in more detail in \cref{sec:lhc}) accelerates beams of protons and collides them in the center of the LHC's four main detectors.
Of the many parameters that describe colliders, two of the most important are center-of-mass energy ($\sqrt{s}$) and instantaneous luminosity ($\lumi$). These are measures of how ``hard'' protons are colliding and how frequently collisions occur, respectively.

Within a proton collider, $\sqrt{s}$ is only indirectly related to the energy of fundamental interactions.
This is because a proton is composed of many strongly-bound partons---interactions between protons are in fact collisions between the constituent particles.
As a result, the energy of the fundamental interaction is described by a convolution between the beam energy and the \emph{parton distribution function}.
In 2012, individual protons within the counter-circulating proton beams collided with a center-of-mass energy of $8\,\tev$ ($\cmenergy$).

\begin{figure}
  \includegraphics[height=0.7\textheight]{misc/cc/crosssections2012HE_v4.pdf}
  \caption[Standard model cross-sections as a function of collider energy]{%
Standard model cross-sections as a function of collider energy. Taken from ref~\cite{sm-crosssec}.}
\end{figure}

\begin{figure}
  \includegraphics[width=\textwidth]{misc/cc/atlas-sm.pdf}
  \caption[Summary of measured SM cross-sections in ATLAS]{%
    Summary of measured SM cross-sections in ATLAS. Taken from ref~\cite{atlas-sm}.}
  \label{fig:sm-crosssection}
\end{figure}

\begin{figure}
  \includegraphics[width=\textwidth]{misc/cc/mstw-proton.pdf}
  \caption[Proton parton distribution function]{%
    Proton parton distribution function. Taken from ref~\cite{mstw-proton}.}
\end{figure}


\begin{itemize}
\item explain ISR
\item explain cross-sections
\end{itemize}
\subsection{Hadrons, Electrons, and Muons}
\begin{itemize}
\item must mention isolation (referenced later)
\item Lifetimes and time dilation
\item electromagnetic and strong interactions in detectors
\item Magnitude of cross-sections: EW vs strong
\end{itemize}
\subsection{Neutrinos (and Neutrilinos)}
the known unknowns
\begin{itemize}
\item introduce $\met$, $\vmet$
\item reference detector chapter
\end{itemize}
\subsection{Taus, and Heavy Quarks}
\begin{itemize}
\item more on lifetimes
\item tracking
\item vertex finding
\end{itemize}
\subsection{W, Z, Higgs, and Everything Else}
\begin{itemize}
\item $Z$ peak reconstruction
\item $\mt$, $\mct$, etc
\item Feynman diagrams for $\wjets$, $\zjets$, $\ttbar$
\item Feynman diagrams for SUSY pair-produced quarks
\end{itemize}
\subsection{Implications for Experiments}
\label{sec:det-design}
\begin{itemize}
\item define ``the hard scatter'', explain why it's appropriate to use a definite article.
\item explain implications for triggering
\end{itemize}
