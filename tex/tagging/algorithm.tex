The JetFitterCharm algorithm consists of two components: a series of ``base taggers'', which convert tracking information into physical parameters relevant to flavor{}-tagging; and a neural network to combine these parameters into a final discriminant.

%% The layout of the ATLAS flavor-tagging framework is shown in figure
%% \ref{fig:tagging-arch}. 

\subsection{Base Taggers}

Tracks are selected within a
$\Delta R$ cone surrounding the jet center in the calorimeter, which varies as a function of the transverse momentum of the jet. These tracks are passed to three
top level algorithms, which distill the detailed track information
to several dozen variables per jet. The base taggers are summarized only briefly below, since detailed descriptions exist elsewhere~\cite{AdvancedTaggers}.%% While each of the
%% three algorithms can be used as a flavor tagger in its own right, they
%% measure slightly different properties and thus provide complementary
%% information~\cite{Burmeister:1542409}:
\begin{description}
  \item[IP3D:] The IP3D tagger takes as input the transverse and longitudinal
    signed impact parameter significance of tracks with respect to the primary vertex,
    $S_{xy}$ and $S_{z}$.
    Based on these a two-dimensional likelihood function is computed from simulation, separately for $b$, light, and $c$-jet flavors.
    The IP3D outputs,
    $P_{b}^{\rm IP}$, $P_{c}^{\rm IP}$, and $P_{\rm light}^{\rm IP}$ are then calculated according to
     $P_f^{\rm IP} = \prod_{k=0}^{N_{\rm{tracks}}} L_f^{\rm trk} \left(S_{xy, k},S_{z, k}\right)$
    where the $f$ subscripts represent the three flavors of jet and $L^{\rm trk}$ is the
    likelihood function derived from simulation.
  \item[SV1:] SV1 fits the tracks in the jet into the most probable secondary
    vertex position. After fitting, SV1 returns several summary parameters, as detailed in table \ref{tag:tab:parameters}.
  \item[JetFitter:] JetFitter reconstructs the decay chain under the hypothesis
    that one or more displaced vertices should fall on a line beginning at the
    primary vertex and extending in the direction of the jet's associated
    calorimeter deposits. One major strength of JetFitter is its
    ability to reconstruct both a secondary and a tertiary
    vertex. JetFitterCharm uses a retuned variant of JetFitter which
    is more likely to cluster the secondary and tertiary vertices
    separately.
\end{description}


\begin{table}
  \centering
  \begin{tabular}{c | c | c }
    Algorithm             & Variable Name & Description \\
    \hline
    \multirow{4}{*}{Kinematic}
                          & \multirow{2}{*}{$p_{T}^{\rm cat}$} & $\pt$ category of jet, divisions [GeV]: \\
                          &                & \catpt \\
                          & \multirow{2}{*}{$\eta^{\rm cat}$}  & $|\eta|$ category of jet, divisions: \\
                          &                &  \cateta \\
    \hline
    \multirow{1}{*}{IP3D} & $\log (P_{b}^{\rm IP}/ P_{\rm light}^{\rm IP})$ & log ratio between $b$-jet and light-jet likelihood value \\
    \hline
    SV1               & $n_{\rm trk}^{{\rm SV1}}$     & Number of tracks matched to the vertex \\
 
    \hline
    \multirow{1}{*}{}

                          & $n_{\rm 2t}$      & Number of two-track vertices found in the jet \\
    SV1, JetFitter& $m_{\rm vx}$      & Secondary vertex mass \\
                         
  (similar variables)& \multirow{2}{*}{$E_{\rm vx}/E_{\rm jet}$}  & Ratio of the vertex track energy sum \\
                         &                  & to the jet track energy sum \\
                          & $d / \sigma_{d}$ & Secondary vertex flightlength significance \\
    \hline
    \multirow{8}{*}{JetFitter}
                          & $n_{\rm vx}$      & Number of reconstructed vertices with  $\ge 2$ tracks \\
                          & $n_{\rm trk}^{\rm JV}$     & Number of tracks matched to vertices with $\ge 2$ tracks \\
                          & $n_{\rm 1t}$      & Number of single-track vertices\\
                          & $L_{xy}^{1}$ & Transverse displacement of the secondary vertex \\
                          & $L_{xy}^{2}$ & Transverse displacement of the tertiary vertex \\
                          & $\min \varphi_{\rm trk}$ & Minimum track rapidity along jet axis \\
                          & $\langle \varphi_{\rm trk} \rangle$ & Mean track rapidity along jet axis \\
                          & $\max \varphi_{\rm trk} $ & Maximum track rapidity along jet axis \\
                          %% & $\min \pt^{\rm rel}$ & Minimum track momentum transverse to jet axis \\
                          %% & $\langle \pt^{\rm rel} \rangle$ & Mean track momentum transverse to jet axis \\
                          %% & $\max \pt^{\rm rel}$ & Maximum track momentum transverse to jet axis \\

  \end{tabular}
  \caption[Variable summary table]{Summary of the variables used by the JetFitterCharm neural net. JetFitterCharm uses a ``charm tuned'' variant of the standard JetFitter used by other ATLAS taggers. The charm tuned JetFitter also adds $L_{xy}^1$, $L_{xy}^2$, and $\varphi_{\rm trk}$. Note that $\varphi_{\rm trk}$ is the track rapidity computed with respect to the jet axis.}
  \label{tag:tab:parameters}
\end{table}

\subsection{Neural Network}

Variables summarized in table~\ref{tag:tab:parameters} are passed into a neural
network, which calculates the posterior probability that a jet is bottom, charm, or ``light'' flavored. These probabilities are referred to as $P_b$, $P_c$, and $P_{\rm light}$ respectively. The neural network was trained on simulated $\ttbar$ events with the JETNET package~\cite{Peterson:1993nk}, using the labeling described in section~\ref{tag:sec:data-and-simulation} as target posteriors. Tau jets are ignored in the training. As a precaution against training the neural net with a kinematic flavor{} bias, the training sample is reweighed in two-dimensional categories of $\pt^{\rm cat}$ and $\eta^{\rm cat}$, such that the relative  fractions of $b$, $c$ and light jets is constant in all categories.

%GP I would not add this part, with a comparison of the information flow of the other taggers
%\begin{figure}
%  \begin{center}
%    \includegraphics[width=0.6\textwidth]{figures/dot/tagging-arch.pdf}\\
%    \includegraphics[width=0.6\textwidth]{figures/dot/tagging-leg.pdf}
%  \caption[flavor tagging architecture]{The ATLAS flavor tagging
%    information flow. JetFitterCharm is similar to JetFitterCOMBNN, but uses
%    looser vertexing cuts in JetFitter, and adds the SV1 secondary-vertex finder outputs as inputs
%    to the neural network.}
%  \label{fig:tagging-arch}
%  \end{center}
%\end{figure}

%The similarity between JetFitterCharm and the standard ATLAS flavor-tagging tools is illustrated in figure~\ref{fig:tagging-arch}. Both JetFitterCharm and MV1 require the IP3D posteriors as inputs, and both use outputs from SV1 and some variant of JetFitter.  MV1 does not take JetFitter and SV1 outputs directly: instead it relies on posteriors from a lookup table (in the case of SV1) or from a neural net (in the case of JetFitter). JetFitterCharm, by contrast, uses the SV1 and JetFitter outputs directly.

%% JetFitterCharm uses a $\charm$-tuned variant of JetFitter which is able to
%% resolve slightly more $D$ decay vertices than the original. As in the
%% standard $\bottom$ tagging framework, the top level outputs are
%% are fed into a neural net, which maps the input variables to three posterior
%% probabilities correspoiding to $\bottom$-, $\charm$-, and $\light$-jets.
%% This neural net is trained on $t \bar{t}$ using the JETNET
%% package.
