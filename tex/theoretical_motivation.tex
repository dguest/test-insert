\chapter{Theoretical Motivation}

The Standard Model (SM) consists of 17 fundamental particles which together constitute every observed particle and most of the known forces.
It provides a remarkably precise fit to the majority of particle physics data, predicted the $W$, $Z$, and Higgs bosons~\cite{wzmass1,higgs1,higgs2} decades before their discovery~\cite{ua1w,ua2w,ua1z,ua2z,atlashiggs,cmshiggs}.
For all its success, however, the Standard Model is an incomplete model of physics.

This is by design. The Standard Model is a verifiable and self-consistent description of the strong, weak, and electromagnetic interactions, but by definition it omits many phenomena.
Historically, the first such omission was gravitation; even today any attempts to incorporate gravity into particle theory are extremely speculative.
Recent observations have pointed to other glaring omissions:
neutrinos, massless in the Standard Model, clearly have mass;
the cosmos is filled with roughly 5 times more `dark matter' than the SM matter we've observed;
and the Standard Model lacks any means to produce the matter-antimatter asymmetry we see in the universe.
%% Furthermore, the Higgs mass itself

The Standard Model isn't \emph{wrong}, it's simply reached the limit of it's predictive power.

\section{The Standard Model}
\label{sec:standard-model}
The Standard Model of Particle Physics (SM) unifies the string, weak, and electromagnetic forces under a single theoretical framework.
\begin{figure}
  \includegraphics[width=\textwidth]{misc/cc/standard-model.pdf}
  \caption[The Standard Model of particle physics]{The Standard Model of particle physics. Taken from Ref.~\cite{smwiki}.}
\end{figure}

Should cite~\cite{ewuv,ewgaugeinvariance,weakinthev}.

Discuss how the standard model makes accurate predictions of phenomena: it's not supposed to explain everything.
\subsection{Electromagnetism}

This should still make sense for grandma.

Discuss visible matter: chemistry comes from quarks and electrons. Introduce Electromagnetic force.

It stops making sense for grandma here. Introduce QED and feynman diagrams informally.

\subsection{$W$ and $Z$ Bosons}
Start with electromagnetic force, move into weak interactions. Maybe mention UV divergence of Fermi interaction as justification for W and Z bosons.
\subsection{The Higgs Field}
Simplest possible explanation as to why we need a Higgs.

Talk about mass corrections from the top quark.


\section{Supersymmetry}
\label{sec:susy}
\begin{itemize}
\item Historical origin
\item Cite~\cite{susywes}
\item reference string theory
\item Spectrum (before and after breaking)
\item Motivate with ``accidental'' problems it solves
\begin{itemize}
  \item dark matter
  \item maybe something more about WIMPs
\end{itemize}
\end{itemize}

\subsection{Naturalness}
light stop cancels top
\subsection{LHC exclusions}
leaves the situation rather un-natural
\subsection{Supersymmetric Charm Physics}
\begin{itemize}
\item limitations based on FCNC
\item loophole for scharm
\item include Feynman diagram of signal
\end{itemize}

\begin{figure}
  \begin{center}
    \includegraphics[width=0.5\textwidth]{%
      paper/figures/feynman_diagram/scsc-ccN1N1.pdf}
    \caption{Feynman diagram of the $\sctoc$ process.}
    \label{fig:sctocfeyn}
  \end{center}
\end{figure}

\section{Collider Physics}
\label{sec:pheno}
\subsection{Hadrons, Electrons, and Muons}
\begin{itemize}
\item Lifetimes and time dilation
\item electromagnetic and strong interactions in detectors
\item Jets
\item Magnitude of cross-sections: EW vs strong
\end{itemize}
\subsection{Neutrinos (and Neutrilinos)}
the known unknowns
\begin{itemize}
\item introduce $\met$, $\vmet$
\item reference detector chapter
\end{itemize}
\subsection{Taus, and Heavy Quarks}
\begin{itemize}
\item more on lifetimes
\item tracking
\item vertex finding
\end{itemize}
\subsection{W, Z, Higgs, and Everything Else}
\begin{itemize}
\item Charge conservation
\item Invariant mass
\item $\mt$, $\mct$, etc
\item Feynman diagrams for $\wjets$, $\zjets$, $\ttbar$
\item Feynman diagrams for SUSY pair-produced quarks
\end{itemize}

