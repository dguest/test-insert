\chapter{Theoretical Motivation}

The standard model (SM) includes all 17 fundamental particles observed to date and explains their interactions via three of the four fundamental forces.
%% The standard model (SM) consists of 17 fundamental particles which together constitute every observed particle and most of the known forces.
It provides a remarkably precise fit to the majority of particle physics data, and predicted the $W$, $Z$, and Higgs bosons~\cite{wzmass1,higgs1,higgs2} decades before their discovery~\cite{ua1w,ua2w,ua1z,ua2z,atlashiggs,cmshiggs}.
For all its success, however, the standard model is an incomplete model of particle physics.

This is by design. The standard model is a verifiable and self-consistent description of the strong, weak, and electromagnetic interactions, but by definition it omits many phenomena.
Historically, the first such omission was gravitation; even today any attempts to incorporate gravity into particle theory are extremely speculative.
Recent observations have pointed to other glaring omissions:
%% neutrinos, massless in the standard model, clearly have mass;
and the universe is filled with roughly 5 times more mysterious `dark matter' than visible matter.
%% and the visible matter shows a matter-antimatter asymmetry which can't be explained by SM processes.
Furthermore, the standard model appears `fine tuned'---very small shift in the constants of the model will completely change its behavior.

With the discovery of the Higgs boson, every major prediction of the standard model has been verified, leaving few hints as to the physical mechanism behind the omitted pieces.
The physics community now finds itself groping for a theory beyond the standard model (BSM) which can continue where the standard model left off. One promising candidate is Supersymmetry~\cite{susyprimer,srednicki,pdg2014}.
By postulating a new symmetry between bosons and ferminons, Supersymmetry (SUSY for short) predicts a \emph{superpartner} to each SM particle. Standard model couplings to SUSY particles could produce dark matter,
%% explain the observed matter-antimatter asymmetry in the universe,
and hint at the structure of a framework to unify all of the forces.
They could also mitigate the fine-tuning issue within the standard model, provided that some SUSY particles are of comparable mass to their SM counterparts. This last feature is especially important from an experimental standpoint; superpartners with similar masses to the SM particles could be produced in accelerators.

The myriad of new particles predicted by SUSY are obvious targets for searches at particle accelerators. Currently, the most sensitive searches are based around Large Hadron Collider (LHC) at CERN, where two counter-circulating particle beams collide within the ATLAS and CMS detectors. Over the last several years, both ATLAS and CMS have conducted many dozens of searches, none of which have discovered SUSY.
Since the exact properties of these hypothetical particles are unknown, the searches require a variety of approaches, and some versions of SUSY remain much more difficult to test.
This thesis documents a search for one particularly challenging scenario, where supersymmetric charm quarks ($\scharm$ quarks) decay to standard model charm quarks ($c$ quarks).
%% \footnote{The statement that searches have covered ``a large fraction of SUSY parameter space'' is heavily dependent on one's prior assumptions about SUSY. If one disregards the fine-tuning argument, there is less reason to assume SUSY should appear at energy scales accessible to the LHC.}
%% This thesis documents the search in one such patch of SUSY parameter space, specifically one in which supersymmetric particles decay to charm quarks.

This chapter is structured as follows:
\cref{sec:standard-model,sec:susy} give an overview of the standard model and Supersymmetry, in enough detail to motivate the search;
%% \cref{sec:susylimits} reviews the current state of SUSY searches;
and \cref{sec:pheno} describes the consequences of the standard model from the collider physics point of view. This chapter is not intended to be a comprehensive theoretical discussion of any of these subjects, as many excellent references already exist~\cite{pdg2014,peskin,srednicki,susyprimer}.

\section{The Standard Model}
\label{sec:standard-model}
The Standard Model of Particle Physics (SM) unifies the string, weak, and electromagnetic forces under a single theoretical framework.
\begin{figure}
  \includegraphics[width=\textwidth]{misc/cc/standard-model.pdf}
  \caption[The Standard Model of particle physics]{The Standard Model of particle physics. Taken from Ref.~\cite{smwiki}.}
\end{figure}

Should cite~\cite{ewuv,ewgaugeinvariance,weakinthev}.

Discuss how the standard model makes accurate predictions of phenomena: it's not supposed to explain everything.
\subsection{Electromagnetism}

This should still make sense for grandma.

Discuss visible matter: chemistry comes from quarks and electrons. Introduce Electromagnetic force.

It stops making sense for grandma here. Introduce QED and feynman diagrams informally.

\subsection{$W$ and $Z$ Bosons}
Start with electromagnetic force, move into weak interactions. Maybe mention UV divergence of Fermi interaction as justification for W and Z bosons.
\subsection{The Higgs Field}
Simplest possible explanation as to why we need a Higgs.

Talk about mass corrections from the top quark.


\section{Supersymmetry}
\label{sec:susy}
\begin{itemize}
\item Historical origin
\item Cite~\cite{susywes}
\item reference string theory
\item Spectrum (before and after breaking)
\item Motivate with ``accidental'' problems it solves
\begin{itemize}
  \item dark matter
  \item maybe something more about WIMPs
\end{itemize}
\end{itemize}

\subsection{Naturalness}
light stop cancels top
\subsection{LHC exclusions}
leaves the situation rather un-natural
\subsection{Supersymmetric Charm Physics}
\begin{itemize}
\item limitations based on FCNC
\item loophole for scharm
\item include Feynman diagram of signal
\end{itemize}

\begin{figure}
  \begin{center}
    \includegraphics[width=0.5\textwidth]{%
      paper/figures/feynman_diagram/scsc-ccN1N1.pdf}
    \caption{Feynman diagram of the $\sctoc$ process.}
    \label{fig:sctocfeyn}
  \end{center}
\end{figure}

%% \section{Status of SUSY Searches}
\label{sec:susylimits}

%% \section{Collider Physics}
\label{sec:pheno}
\subsection{Hadrons, Electrons, and Muons}
\begin{itemize}
\item Lifetimes and time dilation
\item electromagnetic and strong interactions in detectors
\item Jets
\item Magnitude of cross-sections: EW vs strong
\end{itemize}
\subsection{Neutrinos (and Neutrilinos)}
the known unknowns
\begin{itemize}
\item introduce $\met$, $\vmet$
\item reference detector chapter
\end{itemize}
\subsection{Taus, and Heavy Quarks}
\begin{itemize}
\item more on lifetimes
\item tracking
\item vertex finding
\end{itemize}
\subsection{W, Z, Higgs, and Everything Else}
\begin{itemize}
\item Charge conservation
\item Invariant mass
\item $\mt$, $\mct$, etc
\item Feynman diagrams for $\wjets$, $\zjets$, $\ttbar$
\item Feynman diagrams for SUSY pair-produced quarks
\end{itemize}

