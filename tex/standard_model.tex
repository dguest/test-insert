\section{The Standard Model}
\label{sec:standard-model}
The Standard Model of Particle Physics (SM) unifies the string, weak, and electromagnetic forces under a single theoretical framework.
The known fundamental particles are shown in \cref{fig:sm}.
Interactions among these particles are introduced through internal gauge symmetries of the form
% Poincar\'e group?
\begin{equation}
  \su{3} \otimes \su{2} \otimes \mathrm{U}(1),
  \label{eq:smgroups}
\end{equation}
where the $\su{3}$ group is responsible for the strong force and the $\su{2}$ and $\mathrm{U}(1)$ together form the electroweak force.
\begin{figure}
  \includegraphics[width=\textwidth]{misc/cc/standard-model.pdf}
  \caption[The Standard Model of particle physics]{The Standard Model of particle physics. Taken from Ref.~\cite{smwiki}.}
  \label{fig:sm}
\end{figure}

\subsection{The Strong Force}
The $\su{3}$ symmetry is the basis of the strong force, also known as quantum chromodynamics (QCD), which affects all particles with color charge (quarks and gluons).
Under QCD, each quark carries one of three colors, referred to as \emph{red}, \emph{green}, and \emph{blue} while the anti-quarks carry the corresponding anti-colors.
The color component of the quark and anti-quark field can be written as
\begin{align}
  q_{\mathrm{c}} &= \begin{pmatrix} r \\ g \\ b \end{pmatrix}
  &
  \bar{q}_{\mathrm{c}} &= \begin{pmatrix} \bar{r} \\ \bar{g} \\ \bar{b} \end{pmatrix}
\end{align}
where $r$, $g$, and $b$ represent the color or anti-color.
The 8 gluons form the adjoint representation of $\su{3}$ and mediate the strong force through their interactions with quarks. Collectively, quarks and gluons are referred to as \emph{partons}.

The strong force has several interesting features. First, no free color-charged objects exist; all quarks and gluons are bound in color-neutral composite particles. When colored objects are forced apart, the connecting color fields narrow into ``flux-tubes'' of constant width and therefore don't dissipate. At some distance\footnote{The distance at which a flux-tube breaks is on the order of femtometers, so this phenomena isn't directly observable outside lattice simulations.} the field energy greatly exceeds the rest energy of a quark-antiquark ($q\bar{q}$) pair and the tube breaks, with one quark from the a new $q\bar{q}$ pair terminating each of the new tubes.
As result net colored objects are never found in isolation.
A second interesting feature of the strong force is that the strong coupling weakens at higher energy scales. This ``running'' of the coupling constant is illustrated in \cref{fig:alpha-strong}.
\begin{figure}
  \begin{center}
    \includegraphics[width=0.6\textwidth]{misc/cc/alpha-strong.pdf}
    \caption[The strong coupling constant $\alphas$]{The strong coupling constant $\alphas(\mathrm{Q})$, as a function of the energy scale of the interaction. Taken from Ref.~\cite{atlasalphastrong}.}
    \label{fig:alpha-strong}
  \end{center}
\end{figure}

This behavior has important implications in hadron colliders.
When quarks within protons ``hard scatter'' (i.e. transfer a large amount of momentum between colliding partons), some energy is transferred into color fields, which subsequently fragment into a \emph{jet} containing many color-neutral hadrons. During the hard scatter and immediately afterword $\alphas$ is relatively small and can be treated perturbatively, but the coupling increases and becomes non-perturbative as the colored fields coalesce into color-neutral composite particles. As a result jets are notoriously difficult to model.
%% , especially at lower energies where $\alphas$ grows so large that perturbative calculations are impossible. Fortunately,
 %% at LHC energies $\alphas$ is small enough to be modeled perturbatively following the hard scatter. Even so, modeling jets accurately is an ongoing challenge;
Several algorithms~\cite{pythiatheory,herwigpretheory} have been developed for this purpose, but generally give slightly different predictions and deviate from observation.
Experimentally, effects from this deviation are some of the largest sources of systematic uncertainty.

\subsection{The Electroweak Force}

The electroweak force couples to all Standard Model fermions. There are 12 fermions; 6 quarks and 6 leptons, arranged in three successively heavier \emph{generations}. Each generation contains 4 particles: a charged lepton (e.g. an electron), a neutrino, an up-type quark, and a down-type quark. Weak interactions couple left-handed charged leptons to neutrinos and left-handed up-type quarks to down-type quarks. The left-handed leptons are thus arranged in doublets, while the right-handed fermions are singlets:\footnote{Some form of right-handed neutrino must exist to give the neutrinos their small mass, but is omitted from the standard model because the coupling mechanism is not precisely known.}
\begin{align}
  \text{leptons:}\quad &
  \left(\begin{matrix} e_{\mathrm{L}} \\ \nu_{\mathrm{L}} \end{matrix} \right),
  e_{\mathrm{R}}
  &
  \text{quarks:}\quad &
  \left(\begin{matrix} u_{\mathrm{L}} \\ d_{\mathrm{L}} \end{matrix} \right),
  u_{\mathrm{R}}, d_{\mathrm{R}}.
\end{align}
In what follows $\psi_{\mathrm{L}}$ and $\psi_{\mathrm{R}}$ represent generic left-handed doublets and right-handed singlets, respectively.

%% With all the fermions in place, we could now derive a full theory by requiring local $\su{2} \otimes \mathrm{U}(1)$ gauge symmetry, which would naturally give rise to gauge bosons and something resembling  could
Fermions interact via the exchange of gauge bosons, which arise by requiring  a local $\su{2} \otimes \mathrm{U}(1)$ gauge symmetry.
A Lagrangian for these interactions can be written as
\begin{equation}
  \lagr_{\psi} = -\frac{1}{4} F^{a \mu \nu}F^a_{\mu \nu} -
  \frac{1}{4}B^{\mu \nu}B_{\mu \nu} +
  \sum_{\mathrm{L}} \bar{\psi}_{\mathrm{L}} i \slashed{D}_{\mathrm{L}} \psi_{\mathrm{L}} +
  \sum_{\mathrm{R}} \bar{\psi}_{\mathrm{R}} i \slashed{D}_{\mathrm{R}} \psi_{\mathrm{R}}
  \label{eq:shittysm}
\end{equation}
where the sums are over all left or right handed fermions. The two gauge field strength tensors, $F^{a}_{\mu\nu}$ and $B_{\mu \nu}$,  correspond to the $\su{2}$ and $\mathrm{U}(1)$ symmetries, and are defined by
\begin{align}
 F^{a}_{\mu\nu} &= \partial_\mu A^{a}_\nu - \partial_\nu A^{a}_\mu +
 g f^{abc} A^b_\mu A^c_\nu &
 B_{\mu\nu} &= \partial_\mu B_\nu + \partial_\nu B_\mu,
\end{align}
where $g$ is the $\su{2}$ gauge coupling constant, and the $f^{abc}$ are the (totally antisymmetric) structure constants for $\su{2}$.
Right- and left-handed lepton couplings in equation \cref{eq:shittysm} are separated to emphasize that they are fundamentally different particles. This is especially clear from the covariant derivatives $\slashed{D}_{\mathrm{L}}$ and $\slashed{D}_{\mathrm{R}}$:
\begin{align}
  \slashed{D}_{\mathrm{L}} &= \gamma^{\mu} \left(\partial_\mu -
  i g \frac{\sigma^a}{2} A^a_\mu  - i g' Y B_\mu \right) &
  \slashed{D}_{\mathrm{R}} &= \gamma^{\mu} \left( \partial_\mu -
  i g' Y B_\mu \right)
  \label{eq:su2cov}
\end{align}
where $Y$ is the \emph{hypercharge}, and is a constant which depends on the lepton and handedness. In the right-handed case the fermions couple only to the $B_\mu$ field with a strength proportional to $g'$. In the left-handed case, the fermion doublet gains an additional coupling of the form $\bar{\psi}_{\mathrm{L}}\sigma^a A^a_{\mu}\psi_{\mathrm{L}}$, where the $\sigma^a$ are the Pauli matrices. Thus the two components of $\psi_{\mathrm{L}}$ are mixed by the $A^a_{\mu}$ fields.

The lagrangian $\lagr_\psi$ has some nice properties: it can model the weak interaction by mixing up-type with down-type quarks, and charged leptons with neutrinos; and it is (almost) a superset of quantum electrodynamics (QED) if $g$ is set to zero. On the other hand, mass terms are conspicuously missing. Terms of the form $m \bar{\psi} \psi$, $m^2 A^{a\mu} A^a_\mu$, or $m^2 B^\mu B_\mu$ will break $\su{2}$ gauge invariance, so some other solution is needed.

\subsection{The Higgs Field}

Adding mass interactions to $\lagr_\psi$ in any sensible way requires the Higgs field, which takes the form of a complex doublet. This gives the field four degrees of freedom at every point in space:
\begin{equation}
  \phi = \frac{1}{\sqrt{2}} \begin{pmatrix} \phi_1 + i \phi_2 \\ \phi_3 + i \phi_4 \end{pmatrix}.
\end{equation}
With the Higgs field, mass-like terms can be added to the Lagrangian without breaking gauge-invariance.
The full SM Lagrangian, complete with the Higgs field, can be written as
\begin{equation}
  \lagr_{\mathrm{SM}} = \lagr_{\psi} + \lagr_{\phi} + \lagr_{\mathrm{Yuk}}
\end{equation}
where $\lagr_{\psi}$ is given in \cref{eq:shittysm}. The first new term, $\lagr_{\phi}$, describes the Higgs field and its interactions with the gauge bosons. The last term, $\lagr_{\mathrm{Yuk}}$, gives the \emph{Yukawa} couplings between the Higgs and the fermions and is responsible for fermion masses.

The Higgs field Lagrangian $\lagr_{\phi}$ can be written as two terms, a kinetic term and a potential $V(\phi)$:
\begin{equation}
  \lagr_{\phi} = \underbrace{|D_{\mu} \phi|^2}_{\text{kinetic}} -V(\phi).
  \label{eq:lhiggs}
\end{equation}
In the kinetic term, the covariant derivative is similar to that in \cref{eq:su2cov}:
\begin{equation}
  D_{\mu} = \partial_\mu - i g \frac{\sigma^a}{2} A^a_\mu  - i g' \frac{1}{2} B_\mu,
\end{equation}
where $Y$ has been set to $1/2$, since only one SM Higgs field exists in the SM.\footnote{Here the hypercharge convention where $Q = \sigma^3/2 + Y$, where $Q$ is the overall charge, is used.}
The potential term $V(\phi)$ is given by
\begin{equation}
  V(\phi) = - \mu^2 \phi^\dagger \phi + \lambda (\phi^\dagger \phi)^2,
  \label{eq:vhiggs}
\end{equation}
where both $\mu$ and $\lambda$ are strictly positive.
As a result of the negative quadratic term in $V(\phi)$, the Higgs field is unstable at $|\phi| = 0$. This leads to so-called ``spontaneous symmetry breaking''; the Higgs field will settle to the classical ground state $\langle \phi \rangle^2 = \mu^2 / \lambda$.
It is important to note that no symmetries are actually harmed in this process---the Lagrangian is still perfectly invariant under local $\su{2} \otimes \mathrm{U}(1)$ gauge transformations---the spontaneously broken field has merely a acquired a vacuum expectation value (VEV).

The connection between this VEV and the SM particle masses becomes more explicit when the gauge is fixed. For the sake of first-order (tree-level) expansions, a convenient choice is the \emph{unitary} gauge. For all values of $\phi$, there exists some transformation in $\su{2} \otimes \mathrm{U}(1)$ such that
\begin{equation}
  \phi = \frac{1}{\sqrt{2}}\begin{pmatrix} \phi_1 + i \phi_2 \\ \phi_3 + i \phi_4 \end{pmatrix}
  \quad \Longrightarrow \quad
  \phi' = \frac{1}{\sqrt{2}} \begin{pmatrix} 0 \\ v + H \end{pmatrix}
\end{equation}
where $v = \sqrt{\mu^2 / \lambda}$ is the VEV, and $H \in \mathbb{R}$
is the final remaining degree of freedom. With $\phi$ rewritten in this way nonzero values of $H$ correspond to excitations in the Higgs field, i.e. the Higgs boson.

In the unitary gauge with $H = 0$, the individual $A^a_\mu$ and $B_\mu$ fields are no longer mass eigenstates of \cref{eq:lhiggs}. The new mass eigenstates are
\begin{align}
  W^\pm_\mu &= \frac{1}{\sqrt{2}} \left(A_{\mu}^1 \mp i A_\mu^2\right) \\
  Z_\mu &= \frac{1}{\sqrt{g^2 + {g'}^2}} \left( g A_\mu^3 - g' B_\mu \right) \\
  A_\mu &= \frac{1}{\sqrt{g^2 + {g'}^2}} \left( g' A_\mu^3 + g B_\mu \right).
\end{align}
These first two fields are the familiar $W$ and $Z$ bosons, while the last is the photon. Substituting these fields into \cref{eq:lhiggs}, the Higgs field lagrangian is
\begin{equation}
  \mathscr{L}_{\phi} =
  - \frac{1}{2} \partial^\mu H \partial_\mu H
  + \left(\frac{v^2g^2}{4} W^{+\mu}W^-_\mu +
  v^2 \frac{g^2 + {g'}^2}{8} Z^\mu Z_\mu \right) \left(1 + \frac{H}{v} \right)^2
  - V(\phi)
  \label{eq:lew}
\end{equation}
where, in this gauge
\begin{equation}
  V(\phi) = \lambda \left(\frac{1}{4} H^4 + v H^3 + v^2 H^2 - \frac{1}{4} v^4 \right).
  \label{eq:higgs}
\end{equation}
The last term does not couple to any fields and is thus a non-physical artifact. From \cref{eq:lew}, the masses of the $W$ and $Z$ bosons are given by
\begin{align}
  m_{W} &= \frac{v g}{2}  &
  m_{Z} &= \frac{v\sqrt{g^2 + {g'}^2}}{2} = \frac{m_{W}}{\cos \theta_W}
\end{align}
where $\theta_W$ is the \emph{weak mixing angle} defined as $\theta_W = \tan ^{-1} (g'/g)$.
This simple set of parameters, $g$, $g'$, and $v$, which govern both the gauge boson masses and their coupling to fermions is one of the most remarkable successes of the standard model. In every observable parameter theory and experiment are in perfect agreement.

Contrasting from the relatively simple structure of the gauge bosons, the Higgs boson mass and couplings rely on a number of \emph{ad hoc} parameters.
The Yukawa interactions couple the Higgs field to fermions, as described in the final component of the SM lagrangian
\begin{equation}
  \lagr_{\mathrm{Yuk}} =
  - \Gamma^u_{mn} \bar{q}_{\mathrm{L} m} \tilde{\phi} u_{\mathrm{R}n}
  - \Gamma^d_{mn} \bar{q}_{\mathrm{L} m} \phi d_{\mathrm{R}n}
  - \Gamma^e_{mn} \bar{\ell}_{\mathrm{L} m} \phi e_{\mathrm{R}n}
\end{equation}
%% where $q_{\mathrm{L}}$ is the left-handed quark doublet, $u_{\mathrm{R}}$ and $d_{\mathrm{L}}$ are the up and down right-handed quark singlets, $\ell_{\mathrm{L}}$ is the left-handed lepton doublet, and $e_{\mathrm{R}}$ is the right-handed lepton doublet.
where $\tilde{\phi}_i \equiv \epsilon_{ij} \phi_j^*$ ($\epsilon_{ij}$ is the fully antisymmetric tensor) and the $\Gamma_{mn}$ matrices denote the couplings among the three generations. These matrices are singlehandedly responsible for breaking the otherwise perfect generational symmetry. As such, the fermion fields can be redefined to diagonalize the $\Gamma$ matrices in terms of mass matrices:
\begin{equation}
  M^f = \frac{v}{\sqrt{2}} V^f_{\mathrm{L}} \Gamma^f V^f_{\mathrm{R}}
\end{equation}



Far more suspicion surrounds $\lambda$ in \cref{eq:higgs}, which is proportional to the Higgs mass through $m_{H} = \lambda v^2$. Couplings between the Higgs and the fermions lead to an enormous gulf between the bare and dressed masses.
Being a free parameter, $\lambda$ can be adjusted to correct the dressed Higgs mass, but this adjustment is roughly 30 orders of magnitude larger than the dressed Higgs mass. This correction is generally seen as ``unnatural'': evidence that unseen particles are at play.
The $H^3$ and $H^4$ couplings also remain untested; the shape of the Higgs potential is unconfirmed beyond the first order.
%% Many more \emph{ad hoc} parameters enter into the SM when mass terms are added to the fermions in the form of Yukawa couplings.


%% With only three parameters---$g$, $g'$, and $v$---the standard model is able to predict the $W$ and $Z$ boson masses, in addition to  \emph{and} the photon with the fermions, all with only three parameters.


%% To illustrate this, it 
 %% by defining the covariant derivative
%% \begin{equation}
%%   D_\mu = \partial_\mu - i g A^a_{\mu} \frac{\sigma^a}{2} - i g' Y B_{\mu}
%% \end{equation}

\begin{itemize}
\item Should cite~\cite{ewuv,ewgaugeinvariance,weakinthev}.
\item Give the Yukawa couplings in the unitary gauge (up couple to up, down to down)
\item possibly explain how Yukawa coupling gives rise to CKM matrix
\end{itemize}

%% Discuss how the standard model makes accurate predictions of phenomena: it's not supposed to explain everything.
%% \subsection{Electromagnetism}

%% This should still make sense for grandma.

%% Discuss visible matter: chemistry comes from quarks and electrons. Introduce Electromagnetic force.

%% It stops making sense for grandma here. Introduce QED and feynman diagrams informally.

%% \subsection{$W$ and $Z$ Bosons}
%% Start with electromagnetic force, move into weak interactions. Maybe mention UV divergence of Fermi interaction as justification for W and Z bosons.
%% \subsection{The Higgs Field}
%% Simplest possible explanation as to why we need a Higgs.

%% Talk about mass corrections from the top quark.

