\section{The Standard Model}
\label{sec:standard-model}
The Standard Model of Particle Physics (SM) unifies the string, weak, and electromagnetic forces under a single theoretical framework.
The known fundamental particles are shown in \cref{fig:sm}.
Interactions among these particles are introduced through internal gauge symmetries of the form
% Poincar\'e group?
\begin{equation}
  \su{3} \otimes \su{2} \otimes \mathrm{U}(1),
  \label{eq:smgroups}
\end{equation}
where the $\su{3}$ group is responsible for the strong force and the $\su{2}$ and $\mathrm{U}(1)$ together form the electroweak force.
\begin{figure}
  \includegraphics[width=\textwidth]{misc/cc/standard-model.pdf}
  \caption[The Standard Model of particle physics]{The Standard Model of particle physics. Taken from Ref.~\cite{smwiki}.}
  \label{fig:sm}
\end{figure}

\subsection{The Strong Force}
The $\su{3}$ symmetry is the basis of the strong force, also known as quantum chromodynamics (QCD), which affects all particles with color charge (quarks and gluons).
Under QCD, each quark carries one of three colors, referred to as \emph{red}, \emph{green}, and \emph{blue} while the anti-quarks carry the corresponding anti-colors.
The color component of the quark and anti-quark field can be written as
\begin{align}
  q_{\mathrm{c}} &= \begin{pmatrix} r \\ g \\ b \end{pmatrix}
  &
  \bar{q}_{\mathrm{c}} &= \begin{pmatrix} \bar{r} \\ \bar{g} \\ \bar{b} \end{pmatrix}
\end{align}
where $r$, $g$, and $b$ represent the color or anti-color.
The 8 gluons form the adjoint representation of $\su{3}$ and mediate the strong force through their interactions with quarks. Collectively, quarks and gluons are referred to as \emph{partons}.

The strong force has several interesting features. First, no free color-charged objects exist; all quarks and gluons are bound in color-neutral composite particles. When colored objects are forced apart, the connecting color fields narrow into ``flux-tubes'' of constant width and therefore don't dissipate. At some distance\footnote{The distance at which a flux-tube breaks is on the order of femtometers, so this phenomena isn't directly observable outside lattice simulations.} the field energy greatly exceeds the rest energy of a quark-antiquark ($q\bar{q}$) pair and the tube breaks, with one quark from the a new $q\bar{q}$ pair terminating each of the new tubes.
As result net colored objects are never found in isolation.
A second interesting feature of the strong force is that the strong coupling weakens at higher energy scales. This ``running'' of the coupling constant is illustrated in \cref{fig:alpha-strong}.
\begin{figure}
  \begin{center}
    \includegraphics[width=0.6\textwidth]{misc/cc/alpha-strong.pdf}
    \caption[The strong coupling constant $\alphas$]{The strong coupling constant $\alphas(\mathrm{Q})$, as a function of the energy scale of the interaction. Taken from Ref.~\cite{atlasalphastrong}.}
    \label{fig:alpha-strong}
  \end{center}
\end{figure}

This behavior has important implications in hadron colliders.
When quarks within protons ``hard scatter'' (i.e. transfer a large amount of momentum between colliding partons), some energy is transferred into color fields, which subsequently fragment into a \emph{jet} containing many color-neutral hadrons. During the hard scatter and immediately afterword $\alphas$ is relatively small and can be treated perturbatively, but the coupling increases and becomes non-perturbative as the colored fields coalesce into color-neutral composite particles. As a result jets are notoriously difficult to model.
%% , especially at lower energies where $\alphas$ grows so large that perturbative calculations are impossible. Fortunately,
 %% at LHC energies $\alphas$ is small enough to be modeled perturbatively following the hard scatter. Even so, modeling jets accurately is an ongoing challenge;
Several algorithms~\cite{pythiatheory,herwigpretheory} have been developed for this purpose, but generally give slightly different predictions and deviate from observation.
Experimentally, effects from this deviation are some of the largest sources of systematic uncertainty.

\subsection{The Electroweak Force}

The electroweak force couples to all Standard Model fermions. There are 12 fermions; 6 quarks and 6 leptons, arranged in three successively heavier \emph{generations}. Each generation contains 4 particles: a charged lepton (e.g. an electron), a neutrino, an up-type quark, and a down-type quark. Weak interactions couple left-handed charged leptons to neutrinos and left-handed up-type quarks to down-type quarks. The left-handed leptons are thus arranged in doublets, while the right-handed fermions are singlets:\footnote{Some form of right-handed neutrino must exist to give the neutrinos their small mass, but is omitted from the standard model because the coupling mechanism is not precisely known.}
\begin{align}
  \text{leptons:}\quad &
  \left(\begin{matrix} e_{\mathrm{L}} \\ \nu_{\mathrm{L}} \end{matrix} \right),
  e_{\mathrm{R}}
  &
  \text{quarks:}\quad &
  \left(\begin{matrix} u_{\mathrm{L}} \\ d_{\mathrm{L}} \end{matrix} \right),
  u_{\mathrm{R}}, d_{\mathrm{R}}.
\end{align}
In what follows $\psi_{\mathrm{L}}$ and $\psi_{\mathrm{R}}$ represent generic left-handed doublets and right-handed singlets, respectively.

%% With all the fermions in place, we could now derive a full theory by requiring local $\su{2} \otimes \mathrm{U}(1)$ gauge symmetry, which would naturally give rise to gauge bosons and something resembling  could
Fermions interact via the exchange of gauge bosons, which arise by requiring  a local $\su{2} \otimes \mathrm{U}(1)$ gauge symmetry.
A Lagrangian for these interactions can be written as
\begin{equation}
  \mathcal{L}_{\psi} = \frac{1}{4} F^{a \mu \nu}F^a_{\mu \nu} +
  \frac{1}{4}B^{\mu \nu}B_{\mu \nu} +
  \sum_{\mathrm{L}} \bar{\psi}_{\mathrm{L}} i \slashed{D}_{\mathrm{L}} \psi_{\mathrm{L}} +
  \sum_{\mathrm{R}} \bar{\psi}_{\mathrm{R}} i \slashed{D}_{\mathrm{R}} \psi_{\mathrm{R}}
  \label{eq:shittysm}
\end{equation}
where the sums are over all left or right handed fermions. The two gauge field strength tensors, $F^{a}_{\mu\nu}$ and $B_{\mu \nu}$,  correspond to the $\su{2}$ and $\mathrm{U}(1)$ symmetries, and are defined by
\begin{align}
 F^{a}_{\mu\nu} &= \partial_\mu A^{a}_\nu - \partial_\nu A^{a}_\mu +
 g f^{abc} A^b_\mu A^c_\nu &
 B_{\mu\nu} &= \partial_\mu B_\nu + \partial_\nu B_\mu,
\end{align}
where $g$ is the $\su{2}$ gauge coupling constant, and the $f^{abc}$ are the (totally antisymmetric) structure constants for $\su{2}$.
Right- and left-handed lepton couplings in equation \cref{eq:shittysm} are separated to emphasize that they are fundamentally different particles. This is especially clear from the covariant derivatives $\slashed{D}_{\mathrm{L}}$ and $\slashed{D}_{\mathrm{R}}$:
\begin{align}
  \slashed{D}_{\mathrm{L}} &= \gamma^{\mu} \left(\partial_\mu -
  i g \frac{\sigma^a}{2} A^a_\mu  - i g' Y B_\mu \right) &
  \slashed{D}_{\mathrm{R}} &= \gamma^{\mu} \left( \partial_\mu -
  i g' Y B_\mu \right)
  \label{eq:su2cov}
\end{align}
where $Y$ is the \emph{hypercharge}, and is a constant which depends on the lepton and handedness. In the right-handed case the fermions couple only to the $B_\mu$ field with a strength proportional to $g'$. In the left-handed case, the fermion doublet gains an additional coupling of the form $\bar{\psi}_{\mathrm{L}}\sigma^a A^a_{\mu}\psi_{\mathrm{L}}$, where the $\sigma^a$ are the Pauli matrices. Thus the two components of $\psi_{\mathrm{L}}$ are mixed by the $A^a_{\mu}$ fields.

The lagrangian $\mathcal{L}_\psi$ has some nice properties: it can model the weak interaction by mixing up-type with down-type quarks, and charged leptons with neutrinos; and it is (almost) a superset of quantum electrodynamics (QED) if $g$ is set to zero. On the other hand, mass terms are conspicuously missing. Terms of the form $m \bar{\psi} \psi$, $m^2 A^{a\mu} A^a_\mu$, or $m^2 B^\mu B_\mu$ will break $\su{2}$ gauge invariance, so some other solution is needed.

\subsection{The Higgs Field}

Adding mass interactions to $\mathcal{L}_\psi$ in any sensible way requires the Higgs field, which takes the form of a complex doublet. This gives the field four degrees of freedom at every point in space:
\begin{equation}
  \phi = \frac{1}{\sqrt{2}} \begin{pmatrix} \phi_1 + i \phi_2 \\ \phi_3 + i \phi_4 \end{pmatrix}.
\end{equation}
With the Higgs field, mass-like terms can be added to the Lagrangian without breaking gauge-invariance.
The full SM Lagrangian, complete with the Higgs field, can be written as
\begin{equation}
  \mathcal{L}_{\mathrm{SM}} = \mathcal{L}_{\psi} + \mathcal{L}_{\phi} + \mathcal{L}_{\mathrm{Yuc}}
\end{equation}
where $\mathcal{L}_{\psi}$ is the piece given in \cref{eq:shittysm}. The first new piece, $\mathcal{L}_{\phi}$, describes the Higgs field and its interactions with the gauge bosons. The final piece is $\mathcal{L}_{\mathrm{Yuk}}$, which gives the \emph{Yukawa} couplings between the Higgs and the fermions and is responsible for fermion masses.

Starting with 
This field can be coupled to the $\su{2} \otimes \mathrm{U}(1)$ gauge fields through the covariant derivative as in \cref{eq:su2cov}.
 %% by defining the covariant derivative
%% \begin{equation}
%%   D_\mu = \partial_\mu - i g A^a_{\mu} \frac{\sigma^a}{2} - i g' Y B_{\mu}
%% \end{equation}

\begin{itemize}
\item show the higgs potential
\item explain symmetry breaking
\item Should cite~\cite{ewuv,ewgaugeinvariance,weakinthev}.
\item List mass parts of the lagrangian after symmetry breaking
\item possibly explain how Yukawa coupling gives rise to CKM matrix
\end{itemize}

%% Discuss how the standard model makes accurate predictions of phenomena: it's not supposed to explain everything.
%% \subsection{Electromagnetism}

%% This should still make sense for grandma.

%% Discuss visible matter: chemistry comes from quarks and electrons. Introduce Electromagnetic force.

%% It stops making sense for grandma here. Introduce QED and feynman diagrams informally.

%% \subsection{$W$ and $Z$ Bosons}
%% Start with electromagnetic force, move into weak interactions. Maybe mention UV divergence of Fermi interaction as justification for W and Z bosons.
%% \subsection{The Higgs Field}
%% Simplest possible explanation as to why we need a Higgs.

%% Talk about mass corrections from the top quark.

