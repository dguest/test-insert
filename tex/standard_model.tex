\section{The Standard Model}
\label{sec:standard-model}
The Standard Model of Particle Physics (SM) unifies the string, weak, and electromagnetic forces under a single theoretical framework.
The known fundamental particles are shown in \cref{fig:sm}.
Geometrically the Standard Model is a product of the Poincar\'e group and the internal gauge symmetries
\begin{equation}
  \su{3} \otimes \su{2} \otimes \mathrm{U}(1),
  \label{eq:smgroups}
\end{equation}
where the $\su{3}$ group is responsible for the strong force and the $\su{2}$ and $\mathrm{U}(1)$ together form the electroweak force.
\begin{figure}
  \includegraphics[width=\textwidth]{misc/cc/standard-model.pdf}
  \caption[The Standard Model of particle physics]{The Standard Model of particle physics. Taken from Ref.~\cite{smwiki}.}
  \label{fig:sm}
\end{figure}

\subsection{The Strong Force}
The $\su{3}$ group is the basis of the strong force, also known as quantum chromodynamics (QCD), which affects all particles with color charge (quarks and gluons).
Under QCD, each quark carries one of six colors, referred to as \emph{red}, \emph{green}, and \emph{blue} and the corresponding anti-colors.
The color component of the quark field can be written as
\begin{equation}
  \psi_{\mathrm{color}} = \left( \begin{matrix}
    \psi_{\rm r} \\ \psi_{\rm g} \\ \psi_{\rm b}
  \end{matrix} \right)
\end{equation}
where the subscripts represent the color.
The 8 gluons form the adjoint representation of $\su{3}$ and mediate the strong force through their interactions with quarks.

The strong force has several interesting features. First, no free color-charged objects exist; all quarks and gluons are bound in composite particles. When colored objects are forced apart, the connecting color fields narrow into ``flux-tubes'' of constant width and therefore don't dissipate. At some distance\footnote{The distance at which a flux-tube breaks is on the order of femtometers.} the field energy greatly exceeds the rest energy of a quark-antiquark ($q\bar{q}$) pair and the tube breaks, with one quark from the a new $q\bar{q}$ pair terminating each of the new tubes.
A second interesting feature of the strong force is that the strong coupling weakens at higher energy scales. This ``running'' of the coupling constant is illustrated in \cref{fig:alpha-strong}.
\begin{figure}
  \begin{center}
    \includegraphics[width=0.6\textwidth]{misc/cc/alpha-strong.pdf}
    \caption[The strong coupling constant $\alphas$]{The strong coupling constant $\alphas(\mathrm{Q})$, as a function of the energy scale of the interaction. Taken from Ref.~\cite{atlasalphastrong}.}
    \label{fig:alpha-strong}
  \end{center}
\end{figure}

This behavior has important implications in hadron colliders.
When quarks within protons ``hard scatter'' (i.e. transfer a large amount of momentum via the electroweak force), some energy is transferred into color fields, which subsequently fragment into a \emph{jet} of color-neutral hadrons. QCD is a notoriously difficult theory to work with, especially at lower energies where $\alphas$ grows so large that perturbative calculations are impossible. Fortunately, at LHC energies $\alphas$ is small enough to be modeled perturbatively. Even so, modeling jets accurately is an ongoing challenge; several algorithms~\cite{pythiatheory,herwigpretheory} have been developed for this purpose and generally give slightly different predictions.

\subsection{The Electroweak Force}

The electroweak force is modeled by the $\su{2} \otimes \mathrm{U}(1)$ component of \cref{eq:smgroups}.

Should cite~\cite{ewuv,ewgaugeinvariance,weakinthev}.

Discuss how the standard model makes accurate predictions of phenomena: it's not supposed to explain everything.
\subsection{Electromagnetism}

This should still make sense for grandma.

Discuss visible matter: chemistry comes from quarks and electrons. Introduce Electromagnetic force.

It stops making sense for grandma here. Introduce QED and feynman diagrams informally.

\subsection{$W$ and $Z$ Bosons}
Start with electromagnetic force, move into weak interactions. Maybe mention UV divergence of Fermi interaction as justification for W and Z bosons.
\subsection{The Higgs Field}
Simplest possible explanation as to why we need a Higgs.

Talk about mass corrections from the top quark.

