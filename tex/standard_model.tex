\section{The Standard Model}
\label{sec:standard-model}
The Standard Model of Particle Physics (SM) unifies the string, weak, and electromagnetic forces under a single theoretical framework.
The relationship between these forces and the known fundamental particles is depicted in \cref{fig:sm}.
Geometrically, it is a product of the Poincar\'e group and the internal symmetries
\begin{equation}
  \mathrm{SU}(3) \otimes \mathrm{SU}(2) \otimes \mathrm{U}(1).
\end{equation}
The $\mathrm{SU}(3)$ group is the basis of the strong force, also known as quantum chromodynamics (QCD), which affects all particles with color charge, specifically the quarks and gluons.
This 

\begin{figure}
  \includegraphics[width=\textwidth]{misc/cc/standard-model.pdf}
  \caption[The Standard Model of particle physics]{The Standard Model of particle physics. Taken from Ref.~\cite{smwiki}.}
  \label{fig:sm}
\end{figure}

Should cite~\cite{ewuv,ewgaugeinvariance,weakinthev}.

Discuss how the standard model makes accurate predictions of phenomena: it's not supposed to explain everything.
\subsection{Electromagnetism}

This should still make sense for grandma.

Discuss visible matter: chemistry comes from quarks and electrons. Introduce Electromagnetic force.

It stops making sense for grandma here. Introduce QED and feynman diagrams informally.

\subsection{$W$ and $Z$ Bosons}
Start with electromagnetic force, move into weak interactions. Maybe mention UV divergence of Fermi interaction as justification for W and Z bosons.
\subsection{The Higgs Field}
Simplest possible explanation as to why we need a Higgs.

Talk about mass corrections from the top quark.

