
\section{CERN}
\label{sec:cern}

In the years immediately following WWII, European leaders recognized the potential benefits of a common nuclear research organization.
Beyond the obvious need to rebuild the continent's devastated scientific infrastructure was a desire to shunt the recent technological fervor into more peaceful and diplomatic pursuits.
The first international resolution toward this goal was adopted in December 1951. Within two months 11 countries had established the Conseil Européen pour la Recherche Nucléaire (CERN). In September of 1954, not long after construction had begun in Meyrin Switzerland, CERN was officially renamed the European Organization for Nuclear Research~\cite{cern-timeline}.
The name CERN, while only a true acronym in those first three years, remains official to this day.

Over the next half-century increasingly powerful accelerators were constructed at CERN. By the early 1980's the Super Proton Synchrotron (SPS) was colliding protons with anti-protons, with a center of mass energy of $540\,\gev$, leading to the discovery of the $W$ and $Z$ bosons~\cite{ua1w,ua2w,ua1z}.
In the late 80's a 27 kilometer circular tunnel was bored through the sedimentary rock beneath the Franco-Swiss countryside to house the Large Electron-Positron Collider (LEP), which remains the most powerful lepton collider of all time. In 2000, after 10 years of measuring the standard model with unprecedented (and still unsurpassed, in many cases) precision, LEP was decommissioned to make room for the LHC in the same tunnel~\cite{lep-summary}.

%% \begin{cfig}
%%   \graphic[0.49]{misc/cc/cern-map.jpg}
%%   \graphic[0.49]{misc/cc/cern-from-air.jpg}
%%   \caption[CERN From Above]{(left) Map of CERN, the LHC, and surrounding political boundaries. Taken from Ref~\cite{cern-map}. (right) View of CERN from a northwestern vantage point. Point 1 is visible on the upper right, while point 4 is on the lower left. Taken from Ref~\cite{cern-from-air}.}
%%   \label{fig:cern-above}
%% \end{cfig}
\begin{cfig}
  \graphic[0.8]{misc/cc/cern-from-air.jpg}
  \caption[LHC from above]{View of the LHC from a mountain to the northwest of the complex (Cr\^et de la Neige). Point 1 is visible on the upper right, while point 4 is on the lower left. Taken from Ref~\cite{cern-from-air}.}
  \label{fig:cern-from-air}
\end{cfig}

Construction of the LHC continued throughout the 2000's. While the LHC was undergoing commissioning in 2008, an electrical fault caused an explosion which delayed collisions by roughly one year~\cite{lhc-incident}. Repairs were swift, however, and the first successful proton-proton collisions took place in November of 2009.
