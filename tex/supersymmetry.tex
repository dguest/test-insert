\section{Supersymmetry}
\label{sec:susy}

Supersymmetry~\cite{sggs,duelfermions,susywes} extends the standard model by adding an additional symmetry between fermions and bosons. This creates \emph{superpartners} for each of the particles, which differ in spin by $1/2$, but are otherwise identical to their standard model counterparts. Reflecting this symmetry, each SM field and its superpartner is condensed into a \emph{supermultiplet}. The simplest realistic SUSY model is the \emph{Minimal Supersymmetric Standard Model} (MSSM), which contains one supermultiplet for each standard model field. For reasons that will be discussed below, the MSSM also introduces a second Higgs doublet.
A summary of these particles is given in \cref{tab:mssm}
\begin{table}
  \renewcommand{\arraystretch}{1.2}
  \begin{center}
    \begin{tabular}{|c|c|c|c|c|} \hline
      \multicolumn{2}{|c|}{Name} & spin 0 & spin 1/2 & spin 1 \\ \hline
      %% \vspace{0.5em}
      \multirow{3}{*}{quarks} & Q & $\tilde{u}_\Lc$, $\tilde{d}_\Lc$ & $u_\Lc$, $d_\Lc$ & --- \\
       & $u$ & $\tilde{u}_\Rc$ & $u_\Rc$ & --- \\
       & $d$ & $\tilde{d}_\Rc$ & $d_\Rc$ & --- \\ \hline
      %% \vspace{0.5em}
      \multirow{2}{*}{leptons} & L & $\tilde{\nu}$, $\tilde{e}_\Lc$ & ${\nu}$, ${e}_\Lc$ & --- \\
      & $e$ & $\tilde{e}_\Rc$ & $e_\Rc$ & --- \\ \hline
      \multirow{2}{*}{Higgs} & $H_u$ & $H^0_u$, $H^-_u$ & $\tilde{H}^0_u$, $\tilde{H}^-_u$ & --- \\
      & $H_d$ & $H^+_d$, $H^0_d$ & $\tilde{H}^+_d$, $\tilde{H}^0_d$ & --- \\ \hline
      \multicolumn{2}{|c|}{gluons} & --- & $\tilde{g}$ & $g$ \\
      \multicolumn{2}{|c|}{$A$ bosons} & --- & $\tilde{A}^i$ & $A^i$ \\
      \multicolumn{2}{|c|}{$B$ bosons} & --- & $\tilde{B}$ & $B$ \\ \hline
    \end{tabular}
    \caption[MSSM fields]{Fields in the MSSM.}
    \label{tab:mssm}
  \end{center}
\end{table}
\renewcommand{\arraystretch}{1}

The superpartners follow a naming convention which distinguishes them from their SM counterparts. The superpartners to the fermions are prefixed with an `s', leading to the sleptons (selectrons, smuons, staus), the sneutrinos, and the squarks (stop, scharm, etc.). The superpartners to the bosons are suffixed with `-ino', leading to the gluinos, the winos, binos and the higgsinos. A generic superpartner is referred to as a \emph{sparticle}. More simply, they are represented as the SM symbol with a tilde i.e.~$\tilde{t}$ for a stop quark.

The MSSM, as mentioned above, adds a second Higgs doublet. This field is added in response to several problems which arise when extending the SM to SUSY. First, the MSSM Yukawa couplings must be holomorphic; the $\tilde{\phi}$ field in \cref{eq:yuk} isn't allowed, which leaves no Yukawa couplings (and thus no masses) for up-type quarks. Secondly, the triple-gauge coupling ``anomalies''---which cancel in the standard model but would otherwise violate gauge invariance---no longer cancel when higgsinos are added. Adding a second Higgs doublet with opposite hypercharge solves both problems. Yukawa couplings for up-type quarks are supplied by the uncharged component of the new Higgs field, and the gauge anomalies from both Higgs fields cancel each other. These two Higgs fields mix to form three neutral scalar ``Higgs'' bosons: $h^0$, $H^0$, and $A^0$; and two charged bosons: $H^+$ and $H^-$. By convention, $h^0$ is the lightest Higgs; so far this particle is indistinguishable from the observed SM Higgs.

As in the standard model, where the manifestly gauge-invariant $A$ and $B$ fields mix to become the $W$, $Z$ and $\gamma$, the most useful SUSY fields at tree-level are mixtures those given in \cref{tab:mssm}. In the MSSM, the $\tilde{A}^1$ and $\tilde{A}^2$ fields mix with the charged higgsinos to form the \emph{charginos}, $\tilde{\chi}^\pm_1$ and $\tilde{\chi}^\pm_2$. In addition $\tilde{A}^3$ and $\tilde{B}$ mix with the neutral higgsino fields to form the \emph{neutralinos}, $\tilde{\chi}_1^0$, $\tilde{\chi}_2^0$, $\tilde{\chi}_3^0$, and $\tilde{\chi}_4^0$. By convention, the neutrilinos are labeled by ascending mass.

The lightest neutrilino, $\neut$, is stable under the assumptions that SUSY conserves \emph{$R$-parity}, i.e.~that terms which violate lepton or baryon number are forbidden, and that the $\neut$ is the lightest supersymmetric particle (LSP). The first assumption, $R$-parity conservation, is somewhat \emph{ad hoc} and not strictly needed in SUSY models. It is frequently included because the alternative---$R$-parity violation---must be carefully tuned to avoid unphysical predictions like proton decay.
As a consequence of $R$-parity conservation, interactions must include sparticles in pairs---a single sparticle can only decay to another sparticle and some number of SM particles.
This requirement stabilizes the LSP; after being produced, it is left with no kinematically allowed decays and wonders the universe indefinitely until it annihilates with another sparticle.

Whether the LSP is charged or neutral depends on free parameters of the MSSM. A charged LSP, however, would form bound states with SM particles.
Given that no such states have been observed~\cite{chargedlsp} a charged LSP seems unlikely. A neutral LSP, on the other hand, should interact only through the weak force and gravitation. SUSY enthusiasts tout this prediction of a \emph{weakly interacting massive particle} (WIMP) as a major success of the theory; as discussed in \cref{sec:dm} considerable astronomical evidence supports the existence of such a particle.

\begin{itemize}
\item should cite~\cite{light-squarks}
\end{itemize}

\subsection{Naturalness and ``Breaking'' SUSY}
As has been mentioned, the sparticles in the MSSM are an exact replica of the SM particles down to coupling strengths.
This assumption has an important consequence for the Higgs mass corrections and the unnatural tuning mentioned in \cref{sec:yukawa}. To better quantify this statement, assuming a fermion with Yukawa coupling $\Gamma_f$, the correction to the Higgs mass ($\Delta m_{h}$) is given by
\begin{equation}
  \Delta m_{h}^2 = - \frac{|\Gamma_f|^2}{8 \pi^2} (\cutoff^2 + \ldots)
\end{equation}
where $\cutoff$ is the cutoff scale for the theory.
At face value this correction isn't problematic; the standard model is mathematically consistent even if $\cutoff$ is taken to infinity, and under alternative regularization schemes $\cutoff$ won't appear at all.
TALK ABOUT WHAT HAPPENS WHEN WE ADD PARTICLES
%% Problems only start
\begin{itemize}
\item Give the $\Delta m_{H}$ equation that shows how SUSY makes SM more natural.
\item Mention that this suggests SUSY near the TeV scale
\end{itemize}
\subsection{Dark Matter and Unification}
\label{sec:dm}
The ``other merits of SUSY'' chapter
\begin{itemize}
\item Cosmologists take dark matter as fact.
\item Cite WMAP results
\item Grand unification is suggestive
\item point out that string theory relies on this
\end{itemize}
\subsection{LHC exclusions}
leaves the situation rather un-natural
\subsection{Supersymmetric Charm}
\label{sec:supercharm}
\begin{itemize}
\item this hasn't been checked before
\item include Feynman diagram of signal
\end{itemize}

\begin{figure}
  \begin{center}
    \includegraphics[width=0.5\textwidth]{%
      paper/figures/feynman_diagram/scsc-ccN1N1.pdf}
    \caption{Feynman diagram of the $\sctoc$ process.}
    \label{fig:sctocfeyn}
  \end{center}
\end{figure}
