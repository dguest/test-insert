\section{Supersymmetry}
\label{sec:susy}

Supersymmetry~\cite{sggs,duelfermions,susywes} extends the standard model by adding an additional symmetry between fermions and bosons. This creates \emph{superpartners} for each of the particles, which differ in spin by $1/2$, but are otherwise identical to their standard model counterparts. Reflecting this symmetry, each SM field and its superpartner is condensed into a \emph{supermultiplet}. The simplest realistic SUSY model is the \emph{Minimal Supersymmetric Standard Model} (MSSM), which contains one supermultiplet for each standard model field. For reasons that will be discussed below, the MSSM also introduces a second Higgs doublet.
A summary of these particles is given in \cref{tab:mssm}
\begin{table}
  \renewcommand{\arraystretch}{1.2}
  \begin{center}
    \begin{tabular}{|c|c|c|c|c|} \hline
      \multicolumn{2}{|c|}{Name} & spin 0 & spin 1/2 & spin 1 \\ \hline
      %% \vspace{0.5em}
      \multirow{3}{*}{quarks} & Q & $\tilde{u}_\Lc$, $\tilde{d}_\Lc$ & $u_\Lc$, $d_\Lc$ & --- \\
       & $u$ & $\tilde{u}_\Rc$ & $u_\Rc$ & --- \\
       & $d$ & $\tilde{d}_\Rc$ & $d_\Rc$ & --- \\ \hline
      %% \vspace{0.5em}
      \multirow{2}{*}{leptons} & L & $\tilde{\nu}$, $\tilde{e}_\Lc$ & ${\nu}$, ${e}_\Lc$ & --- \\
      & $e$ & $\tilde{e}_\Rc$ & $e_\Rc$ & --- \\ \hline
      \multirow{2}{*}{Higgs} & $H_u$ & $H^0_u$, $H^-_u$ & $\tilde{H}^0_u$, $\tilde{H}^-_u$ & --- \\
      & $H_d$ & $H^+_d$, $H^0_d$ & $\tilde{H}^+_d$, $\tilde{H}^0_d$ & --- \\ \hline
      \multicolumn{2}{|c|}{gluons} & --- & $\tilde{g}$ & $g$ \\
      \multicolumn{2}{|c|}{$A$ bosons} & --- & $\tilde{A}^i$ & $A^i$ \\
      \multicolumn{2}{|c|}{$B$ bosons} & --- & $\tilde{B}$ & $B$ \\ \hline
    \end{tabular}
    \caption[MSSM fields]{Fields in the MSSM.}
    \label{tab:mssm}
  \end{center}
\end{table}
\renewcommand{\arraystretch}{1}

The superpartners follow a waggish naming convention. The superpartners to the fermions are prefixed with an `s', leading to the sleptons (selectrons, smuons, staus), the sneutrinos, and the squarks (stop, scharm, etc.). The superpartners to the bosons are suffixed with `-ino', leading to the gluinos, the winos, binos and the higgsinos. More simply, they are represented as the SM symbol with a tilde i.e.~$\tilde{t}$ for a stop quark.

The MSSM, as mentioned above, adds a second Higgs doublet. This serves two purposes. First, the MSSM Yukawa couplings must be holomorphic; the $\tilde{\phi}$ field in \cref{eq:yuk} isn't allowed. Instead, the up-type quarks couple to a second Higgs doublet with hypercharge $Y = -1/2$.

As in the standard model, where the manifestly gauge-invariant $A$ and $B$ fields mix to become the $W$, $Z$ and $\gamma$, the most useful SUSY fields at tree-level are mixtures those given in \cref{tab:mssm}. %In the SUSY case,

\begin{itemize}
\item need to cite~\cite{light-squarks}
\item point out that string theory relies on this
\item Spectrum (before and after breaking)
\end{itemize}

\subsection{Breaking SUSY and Naturalness}
\begin{itemize}
\item Give the $\Delta m_{H}$ equation that shows how SUSY makes SM more natural.
\item Mention that this suggests SUSY near the TeV scale
\end{itemize}
\subsection{Dark Matter and Unification}
The ``other merits of SUSY'' chapter
\begin{itemize}
\item Cosmologists take dark matter as fact.
\item Cite WMAP results
\item Grand unification is suggestive
\end{itemize}
\subsection{LHC exclusions}
leaves the situation rather un-natural
\subsection{Supersymmetric Charm}
\label{sec:supercharm}
\begin{itemize}
\item this hasn't been checked before
\item include Feynman diagram of signal
\end{itemize}

\begin{figure}
  \begin{center}
    \includegraphics[width=0.5\textwidth]{%
      paper/figures/feynman_diagram/scsc-ccN1N1.pdf}
    \caption{Feynman diagram of the $\sctoc$ process.}
    \label{fig:sctocfeyn}
  \end{center}
\end{figure}
