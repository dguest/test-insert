\section{Supersymmetry}
\label{sec:susy}

Supersymmetry~\cite{sggs,duelfermions,susywes} extends the standard model by adding an additional symmetry between fermions and bosons. In the minimal case, this creates \emph{superpartners} for each of the 16 fermions and gauge bosons, which differ in spin by $1/2$, but are otherwise identical to their standard model counterparts. Reflecting this symmetry, the combined fields are condensed into \emph{supermultiplets}.

\begin{itemize}
\item Cite~\cite{susywes}, talk about supermultiplets
\item point out that string theory relies on this
\item Spectrum (before and after breaking)
\end{itemize}

\subsection{Breaking SUSY and Naturalness}
\begin{itemize}
\item Give the $\Delta m_{H}$ equation that shows how SUSY makes SM more natural.
\item Mention that this suggests SUSY near the TeV scale
\end{itemize}
\subsection{Dark Matter and Unification}
The ``other merits of SUSY'' chapter
\begin{itemize}
\item Cosmologists take dark matter as fact.
\item Cite WMAP results
\item Grand unification is suggestive
\end{itemize}
\subsection{LHC exclusions}
leaves the situation rather un-natural
\subsection{Supersymmetric Charm}
\label{sec:supercharm}
\begin{itemize}
\item this hasn't been checked before
\item include Feynman diagram of signal
\end{itemize}

\begin{figure}
  \begin{center}
    \includegraphics[width=0.5\textwidth]{%
      paper/figures/feynman_diagram/scsc-ccN1N1.pdf}
    \caption{Feynman diagram of the $\sctoc$ process.}
    \label{fig:sctocfeyn}
  \end{center}
\end{figure}
