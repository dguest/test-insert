\section{Supersymmetry}
\label{sec:susy}

Supersymmetry~\cite{sggs,duelfermions,susywes} extends the standard model by adding an additional symmetry between fermions and bosons. As a result, each standard model particle is given a new superpartner or \emph{sparticle}, which differs in spin by $1/2$, but is otherwise identical.
Supersymmetry cancels the unnatural tuning discussed in \cref{sec:naturalness} by introducing corrections which are identical in magnitude to those in \cref{eq:higgs-deltam}, but opposite in sign.

While this is a miraculous result, this simple model also has a fatal flaw; it predicts sparticles with masses identical to their SM counterparts.
To date, no such sparticles have been observed, despite a huge number of experiments which should have been sensitive to them---this simple model is clearly not enough.
SUSY can be salvaged by adding \emph{soft} supersymmetry breaking terms ($\lagr_{\text{soft}}$), such that the total lagrangian can be written as
\begin{equation}
  \lagr = \lagr_{\text{SUSY}} + \lagr_{\text{soft}},
\end{equation}
where $\lagr_{\text{SUSY}}$ is essentially $\lagr_{\text{SM}}$ augmented with superpartners.
Many theories connect $\lagr_{\text{soft}}$ to gravitation and other Planck-scale physics~\cite{olive}, but for the purposes of this thesis the details are less important than the overall effect. By coupling superpartners to additional spontaneously broken symmetries at higher energies, the soft terms add additional mass to the sparticles and push their masses above those of the SM particles.

From one perspective the $\lagr_{\text{soft}}$ terms verge on pseudoscience; with them the sparticles can remain just outside the reach of accelerators indefinitely, leaving an elegant theory which can't be falsified.
Fortunately, the soft terms don't completely void the naturalness problem---they merely mitigate it.
With the soft terms included, the new Higgs correction is
\begin{equation}
  \Delta m_{h}^2 = m_{\text{soft}}^2 \left[ \frac{\Gamma}{16 \pi^2} \ln(\cutoff / m_{\text{soft}}) + \ldots \right],
\end{equation}
where $\Gamma$ stands in for an arbitrary dimensionless coupling which should be of order one in a less contrived theory.
Introducing a realistic SUSY model thus replaces a $\Delta m_{h} \sim \Lambda_{\mathrm{GUT}}$ correction with one where $\Delta m_{h} \sim m_{\text{soft}}$; naturalness is preserved so long as SUSY particles are light.
While this doesn't completely solve the falsifiability issue, it does restrict the region of parameter space where SUSY is compelling.
By most standards, SUSY remains compelling so long as the $m_{\text{soft}}$ terms (and sparticle masses) aren't too far above the electroweak scale. Conveniently, this means $m_{\text{soft}} \lesssim 1\,\tev$, i.e. SUSY should be accessible at the LHC.

\subsection{The Minimal Supersymmetric Standard Model}

Reflecting the boson-fermion symmetry, each SM field and its superpartner is condensed into a \emph{supermultiplet}. The simplest realistic SUSY model is the \emph{Minimal Supersymmetric Standard Model} (MSSM), which contains one supermultiplet for each standard model field. For reasons that will be discussed below, the MSSM also introduces a second Higgs doublet.
A summary of these particles is given in \cref{tab:mssm}
\begin{table}
  \renewcommand{\arraystretch}{1.2}
  \begin{center}
    \begin{tabular}{|c|c|c|c|c|} \hline
      \multicolumn{2}{|c|}{Name} & spin 0 & spin 1/2 & spin 1 \\ \hline
      %% \vspace{0.5em}
      \multirow{3}{*}{quarks} & Q & $\tilde{u}_\Lc$, $\tilde{d}_\Lc$ & $u_\Lc$, $d_\Lc$ & --- \\
       & $u$ & $\tilde{u}_\Rc$ & $u_\Rc$ & --- \\
       & $d$ & $\tilde{d}_\Rc$ & $d_\Rc$ & --- \\ \hline
      %% \vspace{0.5em}
      \multirow{2}{*}{leptons} & L & $\tilde{\nu}$, $\tilde{e}_\Lc$ & ${\nu}$, ${e}_\Lc$ & --- \\
      & $e$ & $\tilde{e}_\Rc$ & $e_\Rc$ & --- \\ \hline
      \multirow{2}{*}{Higgs} & $H_u$ & $H^0_u$, $H^-_u$ & $\tilde{H}^0_u$, $\tilde{H}^-_u$ & --- \\
      & $H_d$ & $H^+_d$, $H^0_d$ & $\tilde{H}^+_d$, $\tilde{H}^0_d$ & --- \\ \hline
      \multicolumn{2}{|c|}{gluons} & --- & $\tilde{g}$ & $g$ \\
      \multicolumn{2}{|c|}{$A$ bosons} & --- & $\tilde{A}^i$ & $A^i$ \\
      \multicolumn{2}{|c|}{$B$ bosons} & --- & $\tilde{B}$ & $B$ \\ \hline
    \end{tabular}
    \caption[MSSM fields]{Fields in the MSSM.}
    \label{tab:mssm}
  \end{center}
\end{table}
\renewcommand{\arraystretch}{1}

The superpartners follow a naming convention which distinguishes them from their SM counterparts. The superpartners to the fermions are prefixed with an `s', leading to the sleptons (selectrons, smuons, staus), the sneutrinos, and the squarks (stop, scharm, etc.). The superpartners to the bosons are suffixed with `-ino', leading to the gluinos, the winos, binos and the higgsinos. More simply, they are represented as the SM symbol with a tilde i.e.~$\tilde{t}$ for a stop quark.

The MSSM, as mentioned above, adds a second Higgs doublet. This field is added in response to several problems which arise when extending the SM to SUSY. First, the MSSM Yukawa couplings must be holomorphic; the $\tilde{\phi}$ field in \cref{eq:yuk} isn't allowed, which leaves no Yukawa couplings (and thus no masses) for up-type quarks. Secondly, the triple-gauge-coupling ``anomalies''---which cancel in the standard model but would otherwise violate gauge invariance---no longer cancel when higgsinos are added. Adding a second Higgs doublet with opposite hypercharge solves both problems. Yukawa couplings for up-type quarks are supplied by the uncharged component of the new Higgs field, and the gauge anomalies from both Higgs fields cancel each other. These two Higgs fields mix to form three neutral scalar ``Higgs'' bosons: $h^0$, $H^0$, and $A^0$; and two charged bosons: $H^+$ and $H^-$. By convention, $h^0$ is the lightest Higgs; so far this particle is indistinguishable from the observed SM Higgs.

Squarks ($\tilde{q}$) are bosons with separate soft mass terms for the left-handed and right-handed fields. Within the first two generations, $\tilde{q}_\Lc$ and $\tilde{q}_{\Rc}$ are roughly mass eigenstates.
Within the $\tilde{t}$ and $\tilde{b}$ fields, Yukawa couplings are strong enough to mix the right- and left-handed fields, and the mass eigenstates become $\tilde{t}_1$, $\tilde{t}_2$, $\tilde{b}_1$, and~$\tilde{b}_2$.

As in the standard model, where the manifestly gauge-invariant $A$ and $B$ fields mix to become the $W$, $Z$ and $\gamma$, the most useful SUSY fields at tree-level are mixtures those given in \cref{tab:mssm}. In the MSSM, the $\tilde{A}^1$ and $\tilde{A}^2$ fields mix with the charged higgsinos to form the \emph{charginos}, $\tilde{\chi}^\pm_1$ and $\tilde{\chi}^\pm_2$. In addition $\tilde{A}^3$ and $\tilde{B}$ mix with the neutral higgsino fields to form the \emph{neutralinos}, $\tilde{\chi}_1^0$, $\tilde{\chi}_2^0$, $\tilde{\chi}_3^0$, and $\tilde{\chi}_4^0$. By convention, the neutrilinos are labeled by ascending mass.

The lightest neutrilino, $\neut$, is stable under the assumptions that SUSY conserves \emph{$R$-parity}, i.e.~that terms which violate lepton or baryon number are forbidden, and that the $\neut$ is the lightest supersymmetric particle (LSP). The first assumption, $R$-parity conservation, is somewhat \emph{ad hoc} and not strictly needed in SUSY models. It is frequently included because the alternative---$R$-parity violation---must be carefully tuned to avoid unphysical predictions like proton decay.
As a consequence of $R$-parity conservation, interactions must include sparticles in pairs---a single sparticle can only decay to another sparticle and some number of SM particles.
This requirement stabilizes the LSP; after being produced, it is left with no kinematically allowed decays and wonders the universe indefinitely until it annihilates with another sparticle.

Whether the LSP is charged or neutral depends on free parameters of the MSSM. A charged LSP, however, would form bound states with SM particles.
Given that no such states have been observed~\cite{chargedlsp} a charged LSP seems unlikely. A neutral LSP, on the other hand, should interact only through the weak force and gravitation. SUSY enthusiasts tout this prediction of a \emph{weakly interacting massive particle} (WIMP) as a major success of the theory; as discussed in \cref{sec:dm} considerable astronomical evidence supports the existence of such a particle.

%% \subsection{``Breaking'' SUSY}
%% In the simplest case, sparticles in the MSSM are exact replicas of their SM superpartners down to coupling strengths.

%% %% Problems only start
%% \begin{itemize}
%% \item Give the $\Delta m_{H}$ equation that shows how SUSY makes SM more natural.
%% \item Mention that this suggests SUSY near the TeV scale
%% \end{itemize}

\subsection{SUSY Serendipity}

SUSY's ability to alleviate the naturalness problem is often viewed as its best single feature, but the theory has become especially popular because it seems to solve so many additional problems by accident.
Ignoring aesthetic merits,
%% \footnote{I'm not a theorist nor or a connoisseur of mathematical beauty, but personally I find SUSY no more ``beautiful'' than the standard model.}
SUSY could partially explain at least two additional (seemingly unrelated) phenomena: dark matter, and grand unification.

\subsubsection{Dark Matter}
\label{sec:dm}

From the perspective of cosmology, physics beyond the standard model of particle physics has already been discovered in the form of dark matter.
The so-called ``standard model of cosmology'', $\smcosmo$, combines a cosmological constant ($\Lambda$) with cold dark matter (CDM), and describes the post-inflation universe almost perfectly~\cite{planck2013,planck2015,wmap}.
More importantly, dark matter plays no small part of this model; the vast majority of all gravitating matter, 84\%, is dark~\cite{planck2013overview}.

\begin{figure}
  \subfigure[The cosmic microwave background, taken from ref.~\cite{planck-picture}.%
]{\includegraphics[width=0.5\textwidth]{%
misc/cc/Planck_CMB_Mollweide_wallpaper.jpg}\label{fig:cmb}}
  \subfigure[The bullet cluster, taken from ref.~\cite{bullet-picture}.%
]{\includegraphics[width=0.5\textwidth]{%
misc/cc/bulletcluster_comp_f2048.jpg}\label{fig:bullet}}
  \caption[Recent astrophysical evidence for dark matter]{
    Recent astrophysical evidence for dark matter. Fits to the cosmic microwave background~\subref{fig:cmb} are consistent with $\smcosmo$, which requires a universe composed of 84\% dark matter.
Collisions between galaxy clusters~\subref{fig:bullet}, such as the so-called ``bullet cluster'', suggest that the clusters are composed of two components: dark matter (blue) which only interacts gravitationally and is imaged with lensing, and visible x-ray emitting matter (red).}
\end{figure}

On scales smaller than cosmological fits, dark matter is supported by galaxy cluster dynamics. Fits to the mass distribution of these clusters, measured with x-ray telescopes~\cite{chandra}, generally supports the dark matter hypothesis.
Adding gravitational lensing maps to these x-ray profiles~\cite{bullet} demonstrates the non-interaction of dark matter more dramatically;
\cref{fig:bullet} shows the ``bullet cluster'' where a pair of gravitating clouds have passed through each other while the associated visible gas clouds collide inelastically.
On yet smaller scales, dark matter is strongly supported by galaxy rotation curves.
These curves provided some of the first evidence of invisible gravitating matter~\cite{andromeda}, and have since been confirmed by much larger surveys~\cite{universal-rotation}.

While the density of dark matter is well established, particle properties such as mass and interactions are much less certain.
Thermodynamic models of the early universe predict that weakly-interacting $\sim 100\,\gev$ particles, initially in equilibrium with the early universe, would persist to the present day with a density remarkably close to that of dark matter~\cite{wimps}. This ``WIMP miracle'' can be cast as another credit to $R$-parity conserving SUSY; the $\neut$ assigns a concrete particle to the otherwise amorphous gravitating WIMP clouds.

\subsubsection{Grand Unification}

The gauge couplings for the $\mathrm{U}(1)$, $\su{2}$, and $\su{3}$ internal symmetries of the standard model are not constant in the practical renormalized theory. Instead, they ``run'' with the energy of the interaction; the strength of the $\mathrm{U}(1)$ coupling increases with energy, while the $\su{2}$ and $\su{3}$ couplings decrease. This running suggests that all three interactions are in fact low energy relics of a single force that appears near the GUT scale, in the neighborhood of $10^{15}\,\gev$.

As shown on the left side of \cref{fig:unification}, within the standard model the three lines fail to intersect at a single point. In fact, taking experimental and theoretical uncertainties into account, the discrepancy is roughly $12 \sigma$~\cite{gutsusy}. Adding supersymmetry shrinks this discrepancy to $3 \sigma$~\cite{pdg2014}, as shown on the right of \cref{fig:unification}. While the unification isn't perfect the considerable improvement is yet another convincing argument for SUSY.

\begin{figure}
  \includegraphics[width=\textwidth]{misc/cc/unification-ugly.jpg}
  \caption[Unification of coupling constants with SUSY]{%
Unification of coupling constants without SUSY (left), and after SUSY has been introduced (right). The constants $\alpha_1$, $\alpha_2$, and $\alpha_3$ correspond to the $\mathrm{U}(1)$, $\su{2}$, and $\su{3}$ gauge symmetries of the  standard model, respectively. Taken from ref.~\cite{unification-ugly}.}
  \label{fig:unification}
\end{figure}


%% \begin{itemize}
%% \item point out that string theory relies on this
%% \end{itemize}
\subsection{Supersymmetric Charm}
\label{sec:supercharm}

The arguments above, in addition to the promise of a rich spectroscopy at colliders, have driven a huge number of SUSY searches at the LHC. None of these searches have returned any evidence for SUSY.
The subset that focus on squarks look for one of two signatures: third-generation squarks and generic ``light'' quark superpartners in the first two generations~\cite{cms-inclusive,atlas-inclusive}.
The first case is motivated by relatively large Yukawa coupling for the $\tilde{t}$, which increases the sensitivity of the Higgs potential to $m_{\tilde{t}}$ and leads to a tight upper-bound~\cite{susymasses,susyoneloop}.
In the second case, the light squark masses are generally considered eight-fold degenerate: $2 \text{ generations} \times (\text{up-type} + \text{down-type}) \times (\tilde{q}_\Lc + \tilde{q}_\Rc)$.

In the general case the squark mixing matrix can be populated by completely  arbitrary soft terms, but this can easily yield unphysical results by introducing unobserved flavor-changing neutral currents (FCNC). These currents can be removed completely by requiring that the mixing matrix be diagonal, in which case the model is said to enforce ``flavor-blindness'' or ``universality''.
While this will suppress FCNC it is arguably overkill; a more nuanced treatment known as \emph{minimal flavor violation} (MFV) can prevent FCNC at an observable level while allowing off-diagonal soft terms in the mixing matrix~\cite{mfveft}. Such off-diagonal elements seem reasonable based on observed flavor and neutrino physics; off-diagonal terms are small in the CKM matrix, but much larger in the cause of neutrino mixing.

Within the framework of MFV, it is possible to break the eight-fold degeneracy of the light quark superpartners~\cite{light-squarks}, such that only a single light squark is observable at the LHC. In both ATLAS and CMS, searches which target degenerate first- and second-generation squarks can be reinterpreted to the non-degenerate case, since the signature is identical. The sensitivity of these reinterpreted searches is reduced, however, because only one of the eight quarks is accessible. Considering each squark separately improves the situation: first generation squarks, $\tilde{u}$ and $\tilde{d}$, as superpartners to the proton constituents, are produced with a higher frequency and have been ruled out up to $\sim 1\,\tev$; second generation quarks can be targeted with designated searches.

This thesis concerns the first designated search for a second generation squark at the LHC. In our model, $\tilde{c}$ quarks are pair produced and subsequently decay to a pair of $\neut$ and two $c$ quarks, as shown in \cref{fig:sctocfeyn}. The two $\neut$, as LSPs, are weakly interacting and aren't detected directly by the detector. The $c$-quarks, by contrast, hadronize into jets which are measured. From the perspective of the detector, the process thus presents a two-part signature: a pair of $c$-quark initiated jets; and an overall imbalance in the momentum of measurable particles.

\begin{figure}
  \begin{center}
    \includegraphics[width=0.5\textwidth]{%
      paper/figures/feynman_diagram/scsc-ccN1N1.pdf}
    \caption{Feynman diagram of the $\sctoc$ process.}
    \label{fig:sctocfeyn}
  \end{center}
\end{figure}
