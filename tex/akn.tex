
%% The best part of being in physics is that you never have to grow up.
%% You're constantly learning, and collared shirts are only as common as b'nei mitzvah.

%% Most graduate students only get one adviser. Somehow I got three.
I owe a great deal to Keith Baker, who gave me my first summer job at CERN in 2009.
Without him my experience in Europe would still be limited to connecting flights.
Thanks, Keith, for introducing me to high energy physics.
In my first several years at CERN I also had the privilege to work with Moustapha Thioye.
His knowledge in all aspects of collider physics, ATLAS, and the French Republic/bureaucracy laid the foundations of my life in the greater Geneva area.

%% The work described in this thesis
My later work as a graduate student was jointly supervised by Paul Tipton and Tobias Golling.
Tobias's unbridled enthusiasm was a driving force behind everything I accomplished.
%% As an enthusiastic dispenser of ideas, Tobias was a driving force behind everything I've accomplished.
At the same time, Paul's calm and holistic perspective has kept my priorities in check.
I owe them both endless gratitude for their guidance.
 %% Both are sharp as physicists and cordial as human beings; both were better advisers than I deserve.

Thanks also to Witold Skiba, the fourth member of my thesis committee, for explaining everything.

Many pieces of the $\sctoc$ analysis described here were contributed by the Oxford SUSY group, especially by Will Kalderon, my fellow workhorse, and Alan Bar, who shepherded us through the \atlas\ approval process much faster than I thought possible.
Alex Dafinca, Anna Henrichs, and Clare Gwenlan also contributed considerable work.
Jan Sch\"affer and Mark Hothfeld deserve recognition for their contribution to $\sttoc$ reinterpretation.

Several colleagues from the flavor tagging group were instrumental to the charm tagging effort.
%% The charm-tagging which forms a cornerstone of this thesis only reached fruition through the efforts of the flavor-tagging group.
The project began only thanks to Giacinto Piacquadio, and reached fruition largely thanks to Tim Scanlon, Frank Filhaut, and Fabrizio Parodi's leadership.
%% I was lucky to work with
My student Luke de Oliveira brought in a refreshing new perspective, and ended up teaching me more about machine learning than I taught him about collider physics.
Thanks to Luke for showing that new ideas can solve our old problems, and to Tim for not dismissing our ideas as crazy.
%% Tim, in particular, was among the first to recognize that my ideas were't all crazy, and for that he has my gratitude.

%% The requisite knowledge for experim
%% The wealth of knowledge that makes an experiment like ATLAS possible is contained in th experimental particle physics came through hundreds of small conversations with colleagues around CERN.
I was lucky to share a large office with some very knowledgeable researchers.
Between Eliot, Ryan, John, Kunkle, and Liz, there was usually someone more senior to explain the finer points of ATLAS analysis.
The office became more amiable with Chris Lester's arrival, and started to feel like a home---for better or worse---when Mia arrived with a full kitchen set.
Thanks to Brett, my fellow corner-dweller, and Tuna, my fellow dissident.

Thanks to the entire Yale group, especially Jahred and Jennifer for the early mentoring, Ford and Johannes for many useful conversations, and Larry for mostly useless conversations. Xiaoxiao, thanks for all the pictures.

%%  but , and  moved in picked up the mood of the office
%% Thanks to my officemates: Kunkle, Mia, Chris, Tuna, Brett, John, Ryan, Liz, and sometimes Kurt
%% I was lucky to share an office with one detachment of the Penn army, and a few members of the Duke militia. () and one member of the Duke militia (Mia).

I was lucky to have friends who reminded me that the best part of living near Geneva was getting away from Geneva.
Thanks Arno, Gerog, Hari, and Rob, for all the belays, and to the larger Austrian ski team for all the tours. I look forward to many more.

%% As much as I appreciated my friends at CERN, nothing beat visits from old friends and family.
%% As much as I appreciated my friends at CERN, some of my greatest memories involve visits from friends and family.
%% Thanks to Cabot, Meg, and Elva, for Sarajevo
Thanks to all my friends from Norwich, especially Matt for living close enough to come skiing. Those who didn't make it---Ivan, Avo, Will, Eric, and Justin---there's still time.
Visits from American friends were the highlights of my time in Europe. Thanks to anyone who made it, especially Cabot, Meg, and Elva, for dragging me to Sarajevo, and to my brother and sister for meeting us there. Ben and Lita, thanks for not falling off a cliff on the way to the Schreckhornh\"utte, and for keeping the bar high in our family.
Long, Day, Ben, and Lita---you've been my inspiration every step of the way, I hope I make you proud.

%% Many others deserve thanks here as well.
%% To spare my readers any further delays I won't list them now.
%% These colleagues and friends will be acknowledged in the coming days.




%% I was amazed by their differences outside their 
%% Both cordial and smart physicists, they seem to 
%% While completing the work described in this thesis, I had the fortune to be coadvised by .
%% Without Tobias Golling's inexhaustible drive I never would have tried so many things; without Paul Tipton's calm and thoughtful perspective I probably would have lost my mind.


%% Almost commically adverse in their styles---Paul with his clam and careful , tobias on a rampage---thier only commonality is their deep of colider physics and amicable 


%% As uncertain 
%% Thanks for 
%%   allowed a single advisor, I've been lucky enough to stumble My dissertation committee happens to include of three of my advisers. Thanks to Keith Baker, for 
%% Thanks to my advisers: between Tobias's intense 
